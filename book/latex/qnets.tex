\section{Definitions for Queuing Networks}\label{sec-qn}

Realistic models of information and communications systems
involve interconnected systems, which can be captured by a
queuing networks. In general, not much can be said about a
queuing network. Even the stability conditions are not known in
general, and there is no equivalent of Loynes'theorem for
networks. Indeed, the natural condition that the utilization
factor is less than 1 is necessary for stability but may not be
sufficient -- see \cite{bramson1994instability} for an example
of a multi-class queuing network, with FIFO queues, Poisson
arrivals and exponential service times, which is unstable with
arbitrarily small utilization factor.

Fortunately, there is a broad class of queuing
networks, the so called \nt{multi-class product
form queuing networks} for which there are simple
and exhaustive results, given in this and the
following section. These networks have the
property that their stationary probability has
product form. They were developed as \nt{BCMP
networks}
 in reference to the authors of \cite{baskett1975open}
or \nt{Kelly networks} in reference to
\cite{kelly1979reversibility}. When there is only one class of
customers they are also called \nt{Jackson} networks in the
open case \cite{jackson1963jobshop} and Gordon and Newell
networks in the closed case \cite{gordon1967closed}. For a
broader perspective on this topic, see the recent books
\cite{serfozo1999introduction} and \cite{chao1999queueing}.
This latter reference presents in particular extensions to
other concepts, including the ``negative customers" introduced
in \cite{gelenbe1991product}. A broad treatment, including
approximate analysis for non product form queuing networks can
also be found in \cite{van1993queueing}.
%
%
%\subsection{
%Definition of Multi-Class Product Form Queuing Networks}
%

We now give the common assumptions required by
multi-class product form queuing networks (we
defer a formal definition of the complete process
that describes the network to
\sref{sec-q-form-def}).

\subsection{Classes, Chains and Markov Routing}
\label{q-qnet-routing}
We consider a network of queues, labeled $s=1,
...,S$, also called \nt{stations}. Customers
visit stations and queue or receive service
according to the particular station service
discipline, and once served, move to another
station or leave the network. Transfers are
instantaneous (delays must be modeled explicitly
by means of delay stations, see below).

Every customer has an attribute called
\nt{class}, in a finite set $\lc 1,...,C\rc $. A
customer may change class in transit between
stations, according to the following procedure
(called \nt{Markov routing} in
\cite{walrand1988introduction}).There is a fixed
non-negative \nt{routing matrix} $Q=\lp
q^{s,s'}_{c,c'}\rp_{s,s',c,c'}$ such that for all
$s,c$: $\sum_{s',c'}q^{s,s'}_{c,c'}\leq 1$. When
a class$-c$ customer leaves station $s$ (because
her service time is completed), she does a random
experiment such that: with probability
$q^{s,s'}_{c,c'}$ she joins station $s'$ with
class $c'$; with probability
$1-\sum_{s',c'}q^{s,s'}_{c,c'}$ she leaves the
network. This random experiment is performed
independently of all the past and present states
of the network. In addition, there are fresh
independent Poisson arrivals, also independent of
the past and present states of the network;
$\nu_c^s$ is the intensity of the Poisson process
of arrivals of class$-c$ customers at station
$s$. We allow $\nu_c^s= 0$ for some or all $s$ and
$c$.

We say that two classes $c, c'$ are \nt{chain
equivalent} if $c=c'$ or if it is possible for a
class $c$-customer to eventually become a class
$c'$ customer, or vice-versa. This defines an
equivalence relation between classes, the
equivalence classes are called \nt{chains}. It
follows that a customer may change class but
always remains in the same chain.

A chain $\calC$ is called \nt{closed} if the
total arrival rate of customers $\sum_{c \in
\calC, s}\la^s_c$ is $0$. In such a case we
require that the probability for a customer of
this chain to leave the network is also $0$, i.e.
$\sum_{c',s'}q^{s,s'}_{c,c'}=1$  for all $c\in
\calC$ and all $s$. The number of customers in a
closed chain is constant.

A chain that is not closed is called \nt{open}. We assume that
customers of an open chain cannot cycle forever in the
network, i.e. every customer of this chain eventually leaves
the network.

A network where all chains are closed is called a
\nt{closed  network}, one where all chains are
open is called an \nt{open network} and otherwise
it is a \nt{mixed network}.

We define the numbers $\theta^s_c$ (\nt{visit
rates}) as one solution to
  \be
  \theta^s_{c} = \sum_{s',c'}\theta^{s'}_{c'}
  q^{s',s}_{c',c} + \nu^s_c
  \label{eq-q-visitRatios}
  \ee
If the network is open, this solution is unique
and $\theta^s_{c}$ can be
interpreted\footnote{This interpretation is valid
when the network satisfies the stability
condition in \thref{theo-q-pf}} as the number of
arrivals per time unit of class-$c$ customers at
station $s$. If $c$ belongs to a closed chain,
$\theta^s_c$ is determined only up to one
multiplicative constant per chain. We assume that
the array $\lp \theta^s_c\rp_{s,c}$ is one non
identically zero, non negative solution of
\eref{eq-q-visitRatios}.

Chains can be used to model different customer
populations while a class attribute may be used
to model some state information, as illustrated
in \fref{fig-q-chains}.

\begin{figure}[htbp]
  \insfig{pfqn}{0.9}
 \mycaption{A Simple Product Form queuing network with 2 chains of
  customers, representing a machine with dual core
  processor. Chain 1 consists of classes 1, 2 and 3. Chain 2 consists of class 4.}
  \mylabel{fig-q-chains}
\end{figure}

It is possible to extend Markov routing to state-dependent
routing, for example, to allow for some forms of capacity
limitations; see \sref{sec-q-qnets-blocking}.

%
%\paragraph{Networks with Blocking}
%It is possible to extend Markov routing to
%state-dependent routing, In particular, it is
%possible to allow for some (limited) forms of
%blocking, as follows. Assume that there are some
%constraints on the network state, for example,
%there may be an upper limit to the number of
%customers in one station. A customer finishing
%service, or, for an open chain, a customer
%arriving from the outside, is denied access to a
%station if accepting this customer would violate
%any of the constraints. Consider the following
%two cases:
%  \begin{description}
%    \item[Transparent Stations with Capacity Limitations]
%    The constraints on the network state are
%    expressed by $L$
%capacity limitations of the form
%  \be
%  \sum_{(s,c) \in \calH_{\ell}}n^s_c \leq
%  \Gamma_{\ell},
%  \;\; \ell=1...L
%  \label{eq-q-qnet-caplim}
%  \ee
%where $n^s_c$ is the number of class $c$
%customers present at station $c$, $ \calH_{\ell}$
%is a subset of $\lc 1,...,S \rc  \times
%  \lc 1,...C\rc $ and $\Gamma_{\ell}\in \Nats$.
%In other words, some stations or groups of
%stations may put limits on the number of
%customers of some classes or groups of classes.
%
%
%    If a customer is
%    denied access to station $s$, she continues
%    her journey through the network, using Markov
%    routing with the fixed matrix $Q$, until she
%    finds a station that accepts her or until she leaves the
%    the network.
%    \item[Partial Blocking with Arbitrary Constraints] The constraints can
%    be of any type. Further, If a customer finishes service
%    and is
%    denied access to station $s$, she stays
%    blocked in service. More precisely,
%    we assume that service
%    distributions are mixtures of exponentials, and the
%    customer resumes the last completed service stage.
%    If the customer was arriving from the
%    outside, she is dropped.
%
%    Further, we need to assume that Markov
%    routing is \nt{reversible}, which means that
%    \be
%       \theta^s_c q^{s,s'}_{c,c'}
%       =
%       \theta^{s'}_{c'}q^{s',s}_{c',c}
%       \label{eq-q-qnet-rev}
%    \ee for all $s,s',c,c'$. Reversibility is a
%    constraint on the topology; bus and star
%    networks give reversible routing, but ring
%    networks do not.
%  \end{description}
%These two forms of routing with blocking are
%acceptable for multi-class product form queuing
%networks, i.e. the results in this chapter
%continue to apply
%\cite{pittel1979closed,leny,kelly1979reversibility}.
%There are other cases, too, see
%\cite{balsamo2000product} and references therein.

\subsection{Catalog of Service Stations}
\label{sec-q-qnet-cat} There are some constraints
on the type of service stations allowed in
multi-class product form queuing networks.
Formally, the service stations must satisfy the
property called ``local balance in isolation"
defined in \sref{sec-q-form-def}, i.e., the stationary
probability of the station in the configuration of \fref{sec-isolation}
must satisfy \eref{eq-q-qnet-lb1} and \eref{eq-q-qnet-lb2}.

In this section we give a catalog of
station types that are known to satisfy this
property. There are only two categories of
stations in our list (``insensitive", and
``MSCCC"), but these are fairly general
categories, which contain many examples such as
Processor Sharing, Delay, FIFO, Last Come First
Serve etc. Thus, in practice, if you have to
determine whether a given station type is allowed
in multi-class product form queuing networks, a
simple solution is to look up the following
catalog.

We use the following definitions. Every station
type is defined by
%
\begin{itemize}
   \item a \nt{discipline}: this specifies how arriving
       customers are queued, and which customers receive
       service at any time.
    We also assume that there is a \nt{station buffer}: this is where
    customers are placed while waiting or
    receiving service, and is represented with
    some form of data structure such that
    every position in the station buffer can be
    addressed by an index $i\in\calI$ where
    $\calI$ is some enumerable set. If $\calB$ is the state of
    the station buffer at a given time, $\calB_i$ is the class of the
    customer present at position $i$ (equal to
    $-1$ if there is no customer present).
    Further we will make use of
    two operations.

    $\calB' = \mathrm{add}(\calB,i,c)$ describes the
    effect of adding a customer of class
    $c$ at position indexed by $i$ into the station buffer described by
    $\calB$.

     $\calB' = \mathrm{remove}(\calB,i)$ describes the
    effect of removing the customer present at
    position $i$, if any (if there is no customer
    at position $i$,
    $\mathrm{remove}(\calB,i)=\calB$).


    For example, if the service discipline is
    FIFO: the data structure is a linear list such as
    $\calB=(c_1, c_2, ...,c_n)$ where $c_i$ is
    the class of the $i$th customer (labeled in
    arrival order); the index set is
    $\calI=\Nats$; $\mathrm{add}(\calB,i,c)=
    (c_1, ...c_{i-1}, c, c_i,..., c_n)$ and
    $\mathrm{remove}(\calB,i) = (c_1, ...c_{i-1}, c_{i+1},...,
    c_n)$.

   We call $\abs{\calB}$ the number of customers present in
   the buffer; we assume that it is always finite (but
unbounded).
  \item a \nt{service requirement}, also called \nt{service
      time}. For example, if a customer is a job, the
      service requirement may be the number of CPU cycles
      required; if it is a packet or a block of data, it
      may be the time to transmit it on a link. We assume
      that service requirements are random and drawn
      independently from anything else when a customer
      joins the service station. Unless otherwise
      specified, the distribution of service requirements
      may depend on the station and the class. Allowing
      service requirements to depend on the class is very
      powerful: it allows for example to model service
      times that are correlated from one visit to the next.
  \item a \nt{service rate}: this is the speed at which the server operates,
  which may depend on the customer class. If the service rate is $1$, the service duration is equal to the service requirement (but the response time may be larger, as it includes waiting time).
  %In \sref{sec-q-sq} the
%      service rate was always $1$. For multi-class
%      networks, there may be interest in allowing state
%      dependent service rates.
%For example, in a processor sharing service station, the
%     service rate is
%      inversely proportional to the number of customers
%present in the queue.
The service rate may be
used to model how resources are shared between
classes at a station.
\end{itemize}


\paragraph{Category 1: \nt{Insensitive
Station} or \nt{Kelly-Whittle} Stations}~\\

This category of stations is called
``insensitive" or ``Kelly-Whittle", for reasons
that become clear below. We first give a formal,
theoretical definition, then list the most
frequent instances.

\textbf{\sc Formal Definition.} \noitemsep
\begin{enumerate}
  \item The service requirement may be any phase type
      distribution; in practice, this may approximate any
      distribution, see \sref{sec-q-ph}. The service
      distribution may be dependent on the class.

    \item (\nt{Insertion Probability}) There is an array of numbers
$\gamma(i,\calB)\geq 0$  defined for any index $i
\in \calI$ and any station buffer state $\calB$,
such that: when a class $c$ customer arrives and
finds the station buffer in state $\calB$ just
before arrival, the position at which this
customer is added is drawn at random, and the
probability that this customer is added at
position indexed by $i$ is
  \be
  \gamma\lp i, \mathrm{add}\lp \calB,i,c\rp\rp
 \label{eq-q-q-net-ins-hjkdf}
  \ee
The same happens whenever a customer finishes a
service phase (from the phase type service
distribution), at which time the customer is
treated as a new arrival.

We assume to avoid inconsistencies that $\sum_{i
\in \calI}\gamma\lp i, \mathrm{add}\lp
\calB,i,c\rp\rp=1$ and $\gamma(i,\calB)=0$ if
there is no customer
    at position $i$ in $\calB$.
    \item (\nt{Whittle Function}) There is a function
        $\Psi()$, called the Whittle Function, defined over
        the set of feasible station buffer states, such
        that $\Psi(\calB)> 0$ and the service
    rate allocated to a user in position $i$ of
    the station buffer is
    \be
    \gamma(i,\calB)
    \frac{
    \Psi\lp \mathrm{remove}\lp \calB, i\rp\rp
    }
    {
    \Psi\lp \calB\rp
    }
    \label{eq-def-sr}
    \ee if there is a customer present at
    position $i$, and $0$ otherwise.
%
Note that any positive function may be taken as
Whittle function; the converse is not true, i.e.
any rate allocation algorithm does not
necessarily derive from a Whittle function.

One frequently considers the case where  \be
    \Psi(\calB) = \Phi(\vn) \ee where $\vn =(n_1,...,n_C)$
with $n_c$ the number of class-$c$ customers in $\calB$,
and $\Phi()$ is an arbitrary positive function defined on
$\Nats^C$. In other words, the Whittle function in such
cases depends on the state of the station only through the
numbers of customers (not their position in the buffer).
The function $\Phi$ is called the \nt{balance function};
the quantity $\frac{\Phi(\vn - \vone_c)}{\Phi(\vn)}$ is the
rate allocated to class $c$. As with Whittle function, any
positive $\Phi$ may be taken as balance function, but the
converse is not true, any rate allocation does not
necessarily derive from a balance function.
%


  \item We assume that for any index $i$, class $c$ and
  station buffer state $\calB$
  \be
  \bracket{
  \mathrm{remove}
       \lp
         \mathrm{add}
         \lp
           \calB, i, c
         \rp,
         i
       \rp
       =
       \calB
       \\
       \mbox{if }\calB_i \mbox{ is not empty: }
       \mathrm{add}
         \lp
           \mathrm{remove}(\calB,i), i, \calB_i
         \rp = \calB
         }
         \label{eq-q-sym-whi}
  \ee
%
%  \be
%  \lp \mathrm{remove}(\calB',i) = \calB \mand \calB'_i=c \rp \Leftrightarrow
%  \lp\calB' = \mathrm{add}(\calB,i,c)\rp \label{eq-q-sym-whi}
%  \ee
%which is equivalent to $\mathrm{remove}
%       \lp
%         \mathrm{add}
%         \lp
%           \calB, i, c
%         \rp,
%         i
%       \rp
%       =
%       \calB
%      $ and $\mathrm{add}
%         \lp
%           \mathrm{remove}(\calB,i), i, \calB_i
%         \rp = \calB, $
i.e. a state remains unchanged if one adds a customer
 and immediately removes it, or vice-versa.
 %Furthermore, we assume that there is only state $\calB'$ such
% that $\mathrm{remove}(\calB',i)=\calB$ and $\calB'_i=c$ (this state is necessarily
% $\mathrm{add}(\calB,i,c)$).
\end{enumerate}

This formal definition may seem fairly appalling,
but, as we show next, it is rarely necessary to
make use of the formal definition. Instead, it
may be easier to look up the next list of
examples.

\textbf{\sc Examples of Insensitive Stations.}

For each of these examples, the service
requirement distribution may be any phase type
distribution, and may be class dependent.

\begin{description}
    \item[\nt{Global PS (Processor Sharing)}] The station
        is as in the Processor Sharing queue of
        \sref{sec-q-ps}. All customers present in the
        station receive service, at a rate equal to
        $\frac{1}{n}$ when there are $n$ customers of any
        class present at the station.

This is a Kelly-Whittle station by taking as
    station buffer the ordered list of
    customer classes $\calB=(c_1, ..., c_n)$. Adding
    a customer at position $i$ has the effect that existing
    customers at positions $\geq i$ are shifted by one
position, thus \eref{eq-q-sym-whi} holds. When a
customer arrives, it is added at any position
$1$ to $n+1$  with equal probability
$\frac{1}{n+1}$, i.e. $
    \gamma(i,\calB)=\frac{1}{\abs{\calB}} $ (recall that
    $\abs{\calB}$ is the total number of customers in when
    the buffer state is $\calB$).
The Whittle function is simply $\Psi(\calB)=1$
for every $\calB$. Thus the
    service rate allocated to a customer is
    $\frac{1}{n}$, as required.


\item[\nt{Global LCFSPR}] This service station is \nt{Last
    Come First Serve, Preemptive Resume} (\nt{LCFSPR}).
    There is one global queue; an arriving customer is
    inserted at the head of the queue, and only this
    customer receives service. When an arrival occurs, the
    customer in service is preempted (service is
    suspended); preempted customers resume service where
    they left it, when they eventually return to service.


This is a Kelly-Whittle station by taking as
station buffer the ordered list of customer
classes $\calB=(c_1, ..., c_n)$ as in the
previous example. When a
    customer arrives, it is added at position $1$, i.e.
    $
    \gamma(i,\calB)=\ind{i=1}
    $.
The Whittle function is also $\Psi(\calB)=1$ for
every $\calB$. Thus the service rate allocated to
a customer is $1$ to the customer at the head of
the queue, and $0$ to all others, as required.


\item[\nt{Per-Class Processor Sharing}] This is a variant
    of the Processor Sharing station, where the service rate is divided
    between customers of the same class, i.e. a customer receives service
    at rate $\frac{1}{n_c}$, where $n_c$ is the number of class $c$ customers
    present in the system.

This is a Kelly-Whittle station by taking as
station buffer a collection of
    $C$ lists, one per class. Only customers of class $c$
    may be present in the $c$th list. An index is a
    couple $i=(c,j)$ where $c$ is a class index
    and $j$ an integer. Adding
    a customer at position $i=(c,j)$ has the effect that existing
    customers in the $c$th list at positions $\geq j$ are shifted by one
    position, and others do not move thus \eref{eq-q-sym-whi} holds.

    When a class $c$ customer arrives, it is
    inserted into the $c$th list, at any position
    $1$ to $n_c+1$, with equal probability. Thus
    $\gamma((c,j),\calB)=0$ if the customer at
    position $(c,j)$ is not of class $c$, and
    $\frac{1}{n_c}$ otherwise.
    We take as Whittle function
    $\Psi(\calB)=1$ for every $\calB$. It follows
    that the service rate allocated to a
    customer of class $c$ is $\frac{1}{n_c}$ as
    claimed above.


    \item[\nt{Per-Class LCFSPR}] This is a variant
    of the LCFSPR station, where one
    customer per class may be served, and this customer is the last arrived
    in this class.

    This is a Kelly-Whittle station by taking as
station buffer a collection of $C$ lists, one per class as
    for per-class PS. When a class $c$ customer arrives, it
    is added at the head of the $c$th queue, thus
    $\gamma(i,\calB)=1$ if $i=(c,1)$ and the class at the
    head of the $c$th queue in $\calB$ is $c$, otherwise
    $0$. It follows that the service rate allocated to a
    customer is $0$ unless it at the head of a queue, i.e.
    this customer is the last arrived in its class. We take
    as Whittle function $\Psi(\calB)=1$ for every $\calB$.
    It follows that this station is equivalent to a
    collection of $C$ independent LCFSPR service stations,
    one per class, with unit service rate in each.

\item[\nt{Infinite Server} (\nt{IS}) or \nt{Delay}
    station] There is no queuing, customers
    start service immediately.

This is a Kelly-Whittle station by taking the
same  station buffer and insertion probability as
for Global PS, but with Whittle function
     $
     \Psi(\calB)=\frac{1}{n!}
     $ where $n= \abs{\calB}$ is the total number
     of customers present in the station.
     It follows that the service rate allocated to any
     customer present in the station is $1$, as
required.
%%%%%%%%%%%%%%%%%%%%%%


\item[PS, LCFSPR and IS with \nt{class dependent
    service rate}]

Consider any of the previous examples, but assume
that the service rate is class dependent, and
depends on the number of customers of this class
present in the station (call $r_c(n_c)$ the
service rate for class $c$).

Thus, for Global PS, the service rate allocated
to a class $c$ customer is $\frac{r_c(n_c)}{n}$;
for Per-Class PS, it is $\frac{r_c(n_c)}{n_c}$.
For Global LCFSPR, the service rate allocated to
the unique customer in service is $r_c(n_c)$; for
Per Class LCFSPR, the service rate allocated to
the class $c$ customer in service is $r_c(n_c)$.
For IS the rate allocated to every class $c$
customer is $r_c(n_c)$.

This fits in the framework of Kelly-Whittle
stations as follows. For PS and LCFSPR (per-class
or global) replace the Whittle function by:
 \ben
   \Psi(\calB)=
   \prod_{c=1}^C \frac{1}{r_c(1)r_c(2)... r_c(n_c)}
 \een
 so that
 \ben
      \frac{
    \Psi\lp \mathrm{remove}\lp \calB, i\rp\rp
    }
    {
    \Psi\lp \calB\rp
    } = r_c(n_c)
      \een as required.
For IS, replace $\Psi$ by
$
   \Psi(\calB)=
   \frac{1}{n!}
   \prod_{c=1}^C \frac{1}{r_c(1)r_c(2)... r_c(n_c)}
$ in order obtain the required service rate.


\item[PS, LCFSPR and IS with \nt{queue size dependent
    service rate}]
Consider any of the first five previous examples,
but assume that the service rate is class
independent, and depends on the total number of
customers $n$  present in the station (call
$r(n)$ the service rate ). Thus for Global PS,
the service rate allocated to one customer is
$\frac{r(n)}{n}$ if this customer is of class
$c$; for Per-Class PS, it is $\frac{r(n)}{n_c}$.
For Global LCFSPR, the service rate allocated to
the unique customer in service is $r(n)$; for Per
Class LCFSPR, the service rate allocated to every
customer ahead of its queue is $r(n)$. For IS,
the service rate for every customer is $r(n)$.

This fits in the framework of Kelly-Whittle
stations as follows. For PS and LCFSPR (per-class
or global) replace the Whittle function by:
 \ben
   \Psi(\calB)= \frac{1}{r(1)r(2)... r(n)}
    \een
so that
 \ben
      \frac{
    \Psi\lp \mathrm{remove}\lp \calB, i\rp\rp
    }
    {
    \Psi\lp \calB\rp
    } = r(n)
 \een
as required. For IS, replace $\Psi$ by $
\Psi(\calB)= \frac{1}{n!}\frac{1}{r(1)r(2)...
r(n)}$ in order obtain the required service rate.

    \item \nt{Symmetric Station}, also called \nt{Kelly
    station}:
    This is a generic type introduced by Kelly in \cite{kelly1979reversibility} under the name of
        ``symmetric" service discipline.

    The station buffer is an ordered list as in the first
    example above. For an arriving
    customers who finds $n$ customers present in the station,
    the probability to join position $i$ is
    $p(n+1,i)$, where
    $\sum_{i=1}^{n+1}p(n+1,i)=1$ (thus
    $\gamma(\calB,i)=p(\abs{\calB},i)$). The rate
    allocated to a customer in position $i$ is
    $p(n,i)$ when there are $n$ customers
    present. The name ``symmetric" comes
    from the fact that the same function is used to
    define the insertion probability and the
    rate.

This fits in the framework of Kelly-Whittle stations, with
Whittle function equal to 1. The global PS and global
LCFSPR stations are special cases of Kelly stations.
%
 \item[\nt{Whittle Network}] This is a
    Per-Class Processor Sharing station where
   the Whittle function is a balance function, i.e.
   $\Psi(\calB) = \Phi(\vn)$.
%
It follows that the service rate for a class $c$
customer is
  \be
  \frac{1}{n_c}\frac{\Phi(\vn - \vone_c)}{\Phi(\vn)} \ee
 where $\vone_c=(0,..1,...0)$ with a
$1$ in position $c$%
\index{$\vone_c=(0,..1...0)$ }. This type of station is
used in \cite{bonald2003insensitive} to model resource
sharing among several classes.

A network consisting of a single chain of classes
and one single Whittle Station is called a
Whittle Network. In such a network, customers of
class $c$ that have finished service may return
to the station, perhaps with a different class.

A Whittle network can also be interpreted as a
single class, multi-station network, as follows.
There is one station per class, and customers may
join only the station of their class. However,
class switching is possible. Since knowing the
station at which a customer resides entirely
defines its class, there is no need for a
customer to carry a class attribute, and we have
a single class network.

In other words, a Whittle Network is a single
class network with PS service stations, where the
rate allocated to station $c$ is $\frac{\Phi(\vn
- \vone_c)}{\Phi(\vn)}$. The product form network
in \thref{theo-q-pf} implies that the stationary
probability that there are $n_c$ customers in
station $c$ for all $c$ is
 \be
  %P(\vn) = \frac{1}{\eta} \Phi(\vn)\prod_{c=1}^C\frac{\bar{S}_c^{n_c} \theta_c^{n_c}}{n_c!}
 P(\vn) = \frac{1}{\eta}
\Phi(\vn)\prod_{c=1}^C\bar{S}_c^{n_c} \theta_c^{n_c}
\label{eq-q-whi-sta}
 \ee
 where $\bar{S}_c$ is the expected service requirement
 at station $c$, $\theta_c$ the
 visit rate and $\eta$ a normalizing constant.

\end{description}
Note that the stationary probability in
\eref{eq-q-whi-sta} depends
 only on the traffic intensity $\rho_c=\bar{S}_c
 \theta_c$, not otherwise on the distribution of
 service times. This is the \nt{insensitivity}
 property; it applies not only to Whittle networks, but more
 generally to all service stations of Category 1,
 hence the name.


\paragraph{Category 2: \nt{MSCCC Station} }~\\

This second category of station contains as
special case the FIFO stations with one or any
fixed number of servers. It is called
\nt{Multiple Server with Concurrent Classes of
Customers} in reference to
\cite{chiola1988product,le1986bcmp,berezner1995quasi}.
A slightly more general form than presented here can be found in
\cite{adan98sum}.

The service requirement \imp{must be}
exponentially distributed with the same parameter
for all classes at this station (but the
parameter may be different at different
stations). If we relax this assumption, this
station is not longer admissible for multi-class
product form queuing networks. Thus, unlike for
category 1, this station type is \imp{sensitive}
to the distribution of service requirements.

The service discipline is as follows. There are
$B$ servers and $G$ \nt{token pools}. Every class
is associated with exactly one token pool, but
there can be several classes associated to the
same token pool. The size of token pool $g$ is an
integer $T_g\geq 1$.

A customer is ``eligible for service" when both one of the $B$
servers becomes available and there is a free token in the pool
$g$ that this customer's class is associated with. There is a
single queue in which customers are queued in order of arrival;
when a server becomes idle, the first eligible customer in the
queue, or to arrive, starts service, and busies both one server
and one token of the corresponding pool. The parameters such as
$G$, $B$ and the mapping $\calG$ of classes to token pools may
be different at every station.

The FIFO queue with $B$ servers is a special case
with $G=1$ token pool, and $T_1=B$.

In addition, this station may have a variable
service rate which depends on the total number of
customers in the station. The rate must be the
same for all classes  (rates that depend on the
population vector are not allowed, unlike for
Category 1 stations).

\begin{ex}{A Dual Core
Machine}\fref{fig-q-chains} illustrates a simple
model of dual core processor. Classes 1, 2 or 3
represent external jobs and class 4 internal
jobs. All jobs use the dual core processor,
represented by station 1. External jobs can cycle
through the system more than once. Internal jobs
undergo a random delay  and a variable delay due
to communication.

The processor can serve up to 2 jobs in parallel,
but some jobs require exclusive access to a
critical section and cannot be served together.
This is represented by an MSCCC station with 2
servers and 2 token pools, of sizes 1 and 2
respectively. Jobs that require access to the
critical section use a token of the first pool;
other jobs use tokens of the second pool (the
second pool has no effect since its size is as
large as the number of servers, but is required
to fit in the general framework of multi-class
product form queuing networks).

The delay of internal jobs is represented by
station 2 (an ``infinite server" station) and the
communication delay is represented by station 3
(a ``processor sharing" station, with a constant
rate server).

Internal jobs always use the critical section.
External jobs may use the critical section at
most once. This is modelled by means of the
following routing rules.
\begin{itemize}
    \item Jobs of classes 1, 2 or 3 are external
    jobs. Jobs of class 1 have never used the
    critical section in the past and do not use it ; jobs of class 2
    use the critical section; jobs of class 3
    have used the critical section in the past
    but do not use it any more.

    After service, a job of class 1
    may either leave or return immediately as class 1 or 2. A job of
    class 2 may either leave or return immediately as class 3. A
    job of class 3 may either leave or return immediately as class
    3.

    \item Jobs of class 4 represent internal jobs. They go
        in cycle through stations 1, 2, 3 forever.

    \item At station 1, classes 2 and 4 are
    associated with token pool 1 whereas classes 1
    and 3 are associated with token pool 2, i.e. $\calG(1)=2, \calG(2)=1, \calG(3)=2$ and
    $\calG(4)=1$. The constraints at station 1 are thus: there can be up to 2 jobs in
    service, with at most one job of classes 2 or
    4.
\end{itemize}

The routing matrix is
 \ben
 \bracket{
 \barr{lcrlcrlcr}
 q^{1,1}_{1,1}&=& \alpha_1; &
 q^{1,1}_{1,2}&=& \beta_1;
 \\
 q^{1,1}_{2,3}&=& \alpha_2;  &
 \\
 q^{1,1}_{3,3}&=& \alpha_3;  &
 \\
 q^{1,2}_{4,4}&=&1; &
 q^{2,3}_{4,4}&=&1; &
 q^{3,1}_{4,4}&=&1;
  \earr
  \\
 q^{s,s'}_{c,c'}=0\moth
 }
 \een
 where all numbers are positive, $\alpha_i \leq 1$ and $\alpha_1+\beta_1\leq
 1$.

There are two chains: $\lc 1, 2,3 \rc$ and $\lc
4\rc$. The first chain is open, the second is
closed, so we have a mixed network.

  Let $\nu$ be the arrival rate of external jobs
and $p_i$ the probability that an arriving job is
of class $i$. The visit rates are
 \ben
 \barr{lrclrclrclrcl}
\mbox{Class 1: } &
 \theta^1_1 &=& \nu \frac{p_1}{1-\alpha_1};
 &
  \theta^2_1 &=& 0 ;&
 \theta^3_1&=& 0;
 \\
 \mbox{Class 2: } &
 \theta^1_2 &=& \nu \lp p_2 + \beta_1 \frac{p_1}{1-\alpha_1}\rp ;&
 \theta^2_2 &=& 0 ;&
 \theta^3_2 &=& 0;
 \\
 \mbox{Class 3: } &
 \theta^1_3 &=& \nu \frac{1}{1-\alpha_3}
  \lp
  p_3 + \alpha_2 p_2 + \alpha_2 \beta_1 \frac{p_1}{1-\alpha_1}
  \rp;
&
 \theta^2_3 &=& 0 ;&
 \theta^3_3 &=& 0;
 \\
\mbox{Class 4: } &
 \theta^1_4&=&1 ;&
 \theta^2_4&=&1 ;&
 \theta^3_4&=&1.
 \earr
 \een
Note that the visit rates are uniquely defined
for the classes in the open chain(1, 2 and 3); in
contrast, for class 4, any constant can be used
(instead of the constant $1$).
\end{ex}
%
%
%
\subsection{The Station Function}
\begin{figure}[htbp]
  \insfig{qnetisol}{0.7}
  \mycaption{Station $s$ in isolation.}
  \mylabel{fig-q-st-isol}
\end{figure}
\label{sec-isolation}
\subsubsection{Station in Isolation}

The expression of the product form theorem uses
the \nt{station function}, which depends on the
parameter of the station as indicated below, and
takes as argument
 the vector $\vn =(n_1,...,n_C)$ where $n_c$ is the number of
class-$c$ customers at this station. It can be
interpreted as the stationary distribution of
numbers of customers in the station in isolation,
up to a multiplicative constant.

More precisely, imagine a (virtual) closed
network, made of this station and one external,
auxiliary Per Class PS station with mean service
time $1$ and service rate $1$ for all classes, as
in \fref{fig-q-st-isol}. In this virtual network
there is one chain per class and every class $c$
has a constant number of customers $K_c$. The
product form theorem implies that, for any values
of the vector $\vK=(K_1,...,K_C)$, this network
has a stationary regime, and the stationary
probability that there are $n_1$ customers of
class 1, ... $n_C$ customers of class $C$ is
  \be
  P^{\mathrm{isol}}(\vn) = \bracket{
  0 \mif n_c > K_c \mfs c
  \\
  f(\vn) \frac{1}{\eta(\vK) }\moth
  }
  \label{eq-q-def-stat-fun}
  \ee
where $\eta(\vK)$ is a normalizing constant
(independent of $\vn$).

It is often useful to consider the \nt{generating function}
$G()$ of the station function, defined as the \nt{$Z$
transform} \index{Z-transform} of the station function, i.e.
for $\vZ=(Z_1,...,Z_C)$:
 \be
 G(\vZ)=\sum_{\vn \geq
 0}f(\vn)\prod_{c=1}^C Z_c^{n_c}
 \label{eq-q-def-stat-fun-2}
 \ee
(Note that, in signal processing, one often uses $Z^{-1}$
instead of $Z$; we use the direct convention, called the
``mathematician's $z$-transform"). The following interpretation
of the generating function is quite useful. By
\thref{theo-q-pf-open}, $G(\vZ)$ is the normalizing constant
for the open network made of this station alone, fed by
independent external Poisson processes of rates $Z_c$, one for
each class $c$. Upon finishing service at this station,
customers leave the network and disappear.



In the rest of this section we give the station
functions for the different stations introduced
earlier.

\subsubsection{Station Function for Category 1}

Let $\mathrm{pop}(\calB)\eqdef (n_1, ...,
 n_C)$ where $n_c$ is the number of class $c$ customers at this
 station when the station buffer is in state
 $\calB$ (i.e. $n_c=\sum_{i \in \calI} \ind{\calB_i=c}$).
 The station function is
 \be
  f(\vn) =
  \sum_{\mathrm{pop}(\calB)=\vn}\Psi(\calB)
  \prod_{c=1}^C \bar{S}_c^{n_c}
 \ee where the summation is over all station
 buffer states $\calB$ for which the vector
of populations is $\vn$, $\bar{S}_c$ is the mean
service time for class $c$ at this station, and
$\Psi$ is the Whittle function of this station.

 Note that that the station function is \imp{independent of
 the insertion
 probabilities} $\gamma$. For example, the
 stationary probability is the same whether the
 station is PS or LCFSPR, since they differ only by
 the insertion probabilities.

 In the case where the Whittle function is a
 balance function, i.e. $\Psi(\calB)=\Phi(\vn)$,
 the summation may in some cases be computed.
 \begin{enumerate}
    \item If the station
 uses global queuing as in the Global PS and
 Global LCFSPR examples,
 there are $\frac{n!}{n_1! ... n_C!}$ station buffer states for
 a given population vector, with $n=\abs{\vn}=\sum_{c=1}^C n_c$.
 The station function is
   \be
  f(\vn) = \frac{n!}{\prod_{c=1}^C n_c!}
   \Phi(\vn)
  \prod_{c=1}^C  \bar{S}_c^{n_c}
  \label{eq-q-stf-t2ss}
  \ee
    \item If the station uses per class queuing as in the
        Per Class PS and Per Class LCFSPR examples, there
        is one station buffer state for one population
        vector and the station function is
  \be f(\vn) =
   \Phi(\vn)
  \prod_{c=1}^C  \bar{S}_c^{n_c}
  \label{eq-q-stf-t2sspc}
  \ee
 \end{enumerate}


\textbf{Global PS/Global LCFSPR/Kelly Station
with constant rate.} In these cases we can assume
that the service rate is 1; for all of these
disciplines the station function is given by
\eref{eq-q-stf-t2ss} with $\Phi(\vn)=1$. The
generating function is
 \be
G(\vZ)=\frac{1}{1-\sum_{c=1}^C \bar{S_c
}Z_c}\label{eq-q-stat-t2-z}
 \ee

 \textbf{Per Class PS/Per Class LCFSPR with
constant rate.} Here too we can assume that the
service rate is 1; the station function is given
\eref{eq-q-stf-t2sspc} with $\Phi(\vn)=1$. The
generating function is
 \be
G(\vZ)=\prod_{c=1}^C\frac{1}{1 - \bar{S_c
}Z_c}\label{eq-q-stat-t2pc-z}
 \ee

\textbf{IS with constant rate. } Here too we can
assume that the service rate is 1; the station
function is given by \eref{eq-q-stf-t2ss} with
$\Phi(\vn)=1/n!$. The generating function is
 \be
G(\vZ)=\mathrm{exp}\lp \sum_{c=1}^C \bar{S_c
}Z_c\rp \label{eq-q-stat-t3-z}
 \ee


\subsubsection{Station Function for
Category 2} For the general station in this category, the
station function is a bit complex. However, for the special
case of FIFO stations with one or more servers, it has a simple
closed form, given at the end of this section.

\textbf{General MSCCC Station} Recall that the
station parameters are:\noitemsep
\begin{itemize}
 \item$r(i)$: service rate when the total number of customers is $i$
 \item $\bar{S}$: the mean service time (independent of the class)
 \item $B$: number of servers
 \item $G$: number of token pools; $T_g$: size
            of token pool $g$; $\calG$: mapping of class to
            token pool, i.e. $\calG(c)=g$ when class $c$ is associated
            with token pool $g$.
 \end{itemize}
The station function is
 \be
 f(\vn)   =  d(\vx)
\frac{\bar{S}^{\abs{\vn}}}{\prod_{i=1}^{\abs{\vn}}r(i) }
\frac{
 \prod_{g=1}^G x_g !} {\prod_{c=1}^C n_c!
  }\label{eq-q-stat-t1}
  \ee
  %
with $\abs{\vn}=\sum_{c=1}^C
 n_c$, $\vx=(x_1,...,x_G)$ and $x_g=\sum_{c:
 \calG(c)=g}n_c$ (the number of customers associated with
 token pool $g$). The function $d$ is a
combinatorial function of $\vx \in\Ints^G$, recursively defined
by $d(\vx)  =  0$ if $x_g \leq 0$ for some $g$, $d(0,...,0)  =
1$ and
%
\be d(\vx) \times \mathrm{bs}(\vx) = \sum_{g =1}^G d(\vx -
\vone_g) \label{eq-q-fn-phi}\ee
%
where $\mathrm{bs}(\vx)\eqdef\min
      \lp
   B,\sum_{g=1}^G \min \lp x_g, T_g\rp \rp $ is the number
of busy servers and $\vone_g=(0,..1...0)$ with a $1$ in
position $g$. Note that \ben \mif \sum_{g}\min\lp
x_g,T_g\rp\leq B \mthen d(\vx)=\prod_{g=1}^G
\frac{1}{\prod_{i=1}^{x_g}\min(i,T_g)}\een

In general, though, there does not appear to be a closed form
for $d$, except when the station is a FIFO station (see below).


For the MSCCC station, the generating function cannot be
computed explicitly, in general, but when the service rate is
constant, i.e. $r(i)=1$ for all $i$, one may use the following
algorithm. Let $D$ be the generating function of $d$, i.e. \be
D(\vX)=\sum_{\vx \in \Nats^G}d(\vx) \prod_{g=1}^G X_g^{x_g}\ee
with $\vX=(X_1...X_G)$. For $\vtau \in \lc 0...T_1\rc \times
...\times \lc
  0...T_G\rc$, let
  \ben D_{\vtau}(\vX)\eqdef \sum_{\vx \geq 0, \min(x_g, T_g) = \tau_g, \forall g}d(\vx)
 \prod_{g=1}^G X_g^{x_g}\een so that
 $ D(\vX)= \sum_{\vtau \in \lc 0...T_1\rc \times ...\times
\lc 0...T_G\rc} D_{\vtau}(\vX)$.  One can
 compute $ D_{\vtau}()$ iteratively, using
 $ D_{\vzero}(\vX)=1$,
 $ D_{\vtau}(\vX) =  0$ if $\tau_g <0$ for some
 $g$ and the following, which follows from
 \eref{eq-q-fn-phi}:
 \be
  D_{\vtau}(\vX) =  \frac{1}{\mathrm{bs}(\vtau)- \sum_{g: \tau_g=T_g} X_g }
  \sum_{g: \tau_g >0} X_g  D_{\vtau-\vone_g}(\vX)
  \label{eq-comp-phidelta}
   \ee
   It is sometimes useful to note that
   \be
    D_{\vtau}(\vX)    =
  \prod_{g=1}^G
  \frac{X_g^{\tau_g}}
  {\tau_g!\lp 1 -\frac{X_g}{T_g}\ind{\tau_g=T_g}\rp}
%
  \mif \vtau \geq 0 \mand \mathrm{bs}(\vtau) < B
  \label{eq-q-ztr-simp}
   \ee
   The generating function of the MSCCC station with constant service rate is then given
   by
   \be
   G(\vZ)  =    D
   (X_1,...,X_G)
   \label{eq-q-stat-t1-z}
\ee with $X_g  =
   \bar{S}\lp\sum_{c \mst \calG(c)=g}Z_c\rp$ for all token pool
   $g$.

\textbf{FIFO with $B$ servers.} This is a special
case of MSCCC, with much simpler formulas than in
        the general case. Here the parameters are
    \begin{itemize}
         \item$r(i)$: service rate when the total number
             of customers is $i$
         \item $\bar{S}$: the mean service time
             (independent of the class)
             \item $B$: number of servers
\end{itemize}
The station function is derived from \eref{eq-q-stat-t1} with
$G=1$. One finds $\vx=(\abs{\vn})$ and $d(j)=
\frac{1}{\prod_{i=1}^j \min(B,i)}$ for $j\geq 1$. Thus:

\be f(\vn)   =
\frac{\bar{S}^{\abs{\vn}}}{\prod_{i=1}^{\abs{\vn}}\lb
r(i) \min(B,i)\rb } \frac{\abs{\vn} !}
{\prod_{c=1}^C
 n_c!
  }\label{eq-q-stat-fifo}
 \ee
In the constant rate case, the generating
function follows from \eref{eq-q-ztr-simp}:
 \be
G(\vZ) = 1 + X + \frac{X^2}{2!}
+...+\frac{X^{B-1}}{(B-1)!}+ \frac{X^B}{B!\lp
1-\frac{X}{B} \rp}\label{eq-q-stat-fifo-z} \ee
with $X=\bar{S}\sum_{c=1}^C Z_c$.

In particular, for the \imp{FIFO station with one
server
 and constant rate}, the station function is \be f(\vn) =
\frac{\bar{S}^{\abs{\vn}}\abs{\vn} !}
{\prod_{c=1}^C n_c! }\label{eq-q-stat-fifo1}\ee
and the generating function is
 \be
G(\vZ)=\frac{1}{1-\bar{S}\sum_{c=1}^C Z_c}
\label{eq-q-qnet-gen-f-fifo}
 \ee

\begin{ex}{Dual Core Processor in \fref{fig-q-chains}}
The station functions are (we use the notation
$n_i$ instead of $n^1_i$):
 \bearn
 f^1(n_1, n_2, n_3, n_4) & = &
 d(n_2+n_4, n_1+n_3)
 \frac{(n_1+n_3)!(n_2+n_4)!}{n_1!n_2!n_3!n_4!}(\bar{S}^1)^{n_1+n_2+n_3+n_4}
 \\
 f^2(n^2_4) & = & (\bar{S}^2)^{n^2_4}\frac{1}{n^2_4 !}
 \\
 f^3(n^3_4) & = & (\bar{S}^3)^{n^3_4}
 \eearn
In the equation, $d$ corresponds to the MSCCC station and is
defined by \eref{eq-q-fn-phi}. The generating functions for
stations 2 and 3 follow immediately from (\ref{eq-q-stat-t3-z})
and (\ref{eq-q-stat-t2pc-z}): \bearn
  G^2(Z_1, Z_2, Z_3, Z_4) & = & e^{\bar{S}^2 Z_4}
  \\
  G^3(Z_1, Z_2, Z_3, Z_4) & = & \frac{1}{1-\bar{S}^3 Z_4}
  \eearn

For station 1, we need more work.

First we compute the generating function $
D(X,Y)\eqdef\sum_{m\geq0, n\geq 0}d(m,n)X^mY^n$, using
\eref{eq-q-fn-phi}. One finds
  \bearn
     D_{0,0}(X,Y) & =  & 1
    \\
     D_{1,0}(X,Y) & =  & \frac{X}{1-X}
    \\
     D_{0,1}(X,Y) & =  & Y
    \\
     D_{1,1}(X,Y) & =  & \frac{1}{2-Y}\lp
                 X D_{0,1} +Y D_{1,0} \rp
                   =   \frac{XY}{1-X}
    \\
     D_{0,2}(X,Y) & =  & \frac{1}{2-X}Y D_{0,1}
                   =   \frac{Y^2}{2-Y}
    \\
     D_{1,2}(X,Y) & =  &  \frac{1}{2-X-Y}
     \lp X  D_{0,2} + Y D_{1,1} \rp
                   =   \frac{XY^2 (3-X-Y)}{(2-X-Y)(2-Y)(1-X)}
  \eearn
and $ D$ is the sum of these 6 functions. After some algebra:
 \be
    D(X,Y)  =
   \frac{1}{1-X} \lp
   1 + Y + \frac{Y^2}{2-X-Y}
     \rp
 \label{eq-q-ex-phi-skl}
 \ee
Using \eref{eq-q-stat-t1-z}, it follows that the
generating functions of station 1 is
  \be
  G^1(Z_1, Z_2, Z_3, Z_4)  =   D(\bar{S}^1
  (Z_2+Z_4), \bar{S}^1 (Z_1+Z_3))
  \label{eq-q-qnet-djjduetgsvdff}
  \ee

\label{ex-q-qnet-dslkjjdasflk}
\end{ex}



\mq{q-qn-aslkdksldoo}{Compare the station
function for an IS station with constant service
rate and equal mean service time for all classes
with a FIFO station with constant rate and $B\to
\infty$.}{Both are the same:
\eref{eq-q-stat-fifo} and \eref{eq-q-stf-t2ss}
with $\Phi(\vn)=1/n!$ give the same result:
$f(\vn)= \frac{\bar{S}^{\abs{\vn}}}{\prod_{c=1}^C
 n_c!}$.}
\mq{q-q-qnets-st-isol}{What is the station
function
 $f^{\mathrm{aux}}()$
  for the
 auxiliary station used in the
 definition of the station in isolation~?}{It is a Per Class PS station with
 $\bar{S}_c=1$ for all $c$ thus $f^{\mathrm{aux}}(\vn)=1$.
 The product form theorem
 implies that the stationary probability to see $n_c$
 customers
  in the station of interest is $\eta f(\vn)$.}
  %
 \mq{q-qnet-djjsuuo}
 {Verify  that $ D(X,0)$
 [resp. $ D(0,Y)$]
 is the generating function of a FIFO station with one server
 [resp. 2 servers] (where
 $ D()$ is given by \eref{eq-q-ex-phi-skl}); explain why.}
 {We find $\frac{1}{1-X}$ and $1+Y+\frac{Y^2}{2-Y}$ as
 given by \eref{eq-q-stat-fifo-z}.

 The generating function $ D(X,Y)$ is the $z$-transform
 of the station function with one class per token group, and
 is also equal to the normalizing constant for the station fed
  by a Poisson process with rate $X$ for group 1 and $Y$ for group 2.
If $Y=0$ we have only group 1 customers,
therefore the station is the same as a single
server FIFO station with arrival rate $X$; if
$X=0$, the station is equivalent to a FIFO
station with 2 servers and arrival rate $Y$.}


\section{The Product-Form Theorem}

\subsection{Product Form} The following theorem
gives the stationary probability of number of
customers in explicit form; it is the main
available result for queuing networks; the
original proof is in \cite{baskett1975open};
extension to any service stations that satisfies
the local balance property is in \cite{MR563738}
and \cite{kelly1979reversibility}; the proof that
MSCCC stations satisfy the local balance property
is in \cite{le1986bcmp,berezner1995quasi}. The
proof that all Kelly-Whittle stations satisfy the
local balance property is novel and is given in
\sref{sec-q-proofs} (see \sref{sec-q-form-def} for more details).
\begin{shadethm} Consider a multi-class network as defined
above. In particular, it uses Markov routing and
all stations are Kelly-Whittle or MSCCC. Assume that the
aggregation condition in \sref{sec-aggreg} holds.

Let $n^s_c$ be the number of class $c$ customers
present in station $s$ and
$\vn^s=(n^s_1,...,n^s_C)$. The stationary
probability distribution of the numbers of
customers, if it exists, is given by
  \be
  P(\vn^1, ...\vn^S) = \frac{1}{\eta} \prod_{s=1}^S \lp   f^s(\vn^s) \prod_{c=1}^C
   \lp \theta^s_c\rp^{n^s_c} \rp
   \label{eq-q-pf}
  \ee
  where $\theta^s_c$ is the visit rate in
  \eref{eq-q-visitRatios}, $f^s()$ is the station function
  and $\eta$ is a positive
   normalizing constant.

Conversely, let $\calE$ be the set of all feasible population
vectors $\vn=(\vn^1,...,\vn^S)$. If
 \be \sum_{\vn\in \calE} \prod_{s=1}^S \lp   f^s(\vn^s) \prod_{c=1}^C
   \lp \theta^s_c\rp^{n^s_c} \rp < \infty \label{eq-q-pf-eta}
   \ee
there exists a stationary probability.
 \label{theo-q-pf}
\end{shadethm}
In the open network case, any vector $(\vn^1, ...\vn^S)$ is
feasible, whereas in the closed or mixed case, the set of
feasible population vectors $\calE$ is defined by the
constraints on populations of closed chains, i.e.
 \ben
 \sum_{c \in \calC}\sum_{s=1}^S n^s_c = K_{\calC}
 \een
for any closed chain $\calC$, where $K_{\calC}$ is the
(constant) number of customers in this chain.

Note that the station function depends only on
the traffic intensities. In particular, the
stationary distribution is not affected by the
variance of the service requirement, for stations
of Category 1 (recall that stations of Category 2
must have exponential service requirement
distributions).


\mq{q-pf-qdsflkj}{What is the relationship between the sum in
\eref{eq-q-pf-eta} and $\eta$ ?}{They are equal.}

\subsection{Stability Conditions}

In the open case, stability is not guaranteed and
may depend on conditions on arrival rates.
However, the next theorem says that stability can
be checked at every station in isolation, and
correspond to the natural conditions. In
particular, pathological instabilities as
discussed in the introduction of \sref{sec-qn}
cannot occur for multi-class product form queuing
networks.
%
\begin{shadethm}[Open Case] Consider a multi-class
product form queuing network as defined above.
Assume that it is \imp{open}. For every station
$s$, $\vtheta^s=(\theta^s_1, ..., \theta^s_C)$ is
the vector of visit rates, $f^s$ the station
function and $G^s()$ its generating function,
given in Equations~ (\ref{eq-q-stat-t2-z}),
(\ref{eq-q-stat-t3-z}), (\ref{eq-q-stat-t1-z}),
and (\ref{eq-q-stat-fifo-z}).

The network has a stationary distribution if and only if for
every station $s$
  \be
    G^s(\vtheta^s) < \infty
  \ee

If this condition holds, the normalizing constant of
\thref{theo-q-pf} is $\eta =\prod_{s=1}^S G^s(\vtheta^s)$.
Further, let $P^s(\vn^s)$ be the stationary probability of the
number of customers in station $s$. Then
 \be P(\vn^1,...\vn^S)=\prod_{s=1}^S P^s(\vn^s)\label{eq-q-pf-ind}\ee
i.e. the numbers of customers in different stations are
independent. The marginal stationary probability for station
$s$ is :
 \be
 P^s(\vn^s) = \frac{1}{G^s(\vtheta^s)} f^s(\vn^s)
 \ee
 \label{theo-q-pf-open}
\end{shadethm}
%

The proof follows from the fact that the existence of an
invariant probability is sufficient for stability (as we assume
that the state space is fully connected, by the aggregation
condition). If the network is closed or mixed, then the
corollary does not hold, i.e. the states in different stations
are \imp{not independent}, though there is product-form. Closed
networks are always stable but it may not be as simple to
compute the normalizing constant; efficient algorithms exist,
as discussed in \sref{q-pf-algo}.

For mixed networks, which contain both closed and open chains,
stability conditions depend on the rate functions, and since
they can be arbitrary, not much can be said in general. In
practice, though, the following sufficient conditions are quite
useful. The proof is similar to that of the previous theorem.

\begin{shadethm} (\emph{Sufficient Stability Condition for Mixed Networks}.)
\label{theo-q-net-stabi-spec} Consider a
multi-class product form queuing network as
defined above. Assume that the network is mixed,
with $C_c$ classes in closed chains and $C_o$
classes in open chains. Let $\vm=(m_1, ...,
m_{C_c})$ be the population vector of classes in
closed chains, and $\vn=(n_1, ...,n_{C_o})$ the
population vector of classes in open chains. For
every station $s$ and $\vm$ define
  \be
  L^s(\theta|\vm) = \sum_{\vn \in \Nats^{C_o}}
  f^s(\vm,\vn) \prod_{c=1}^{C_o}\lp \theta^s_c\rp^{n^s_{c}}
  \ee
 where  $f^s(\vm,\vn)$ is the station function.

 If \ben L^s(\theta|\vm) < \infty, \; \forall \vm, \forall s \een
the network has a stationary distribution.
\end{shadethm}

In simple cases, a direct examination of \eref{eq-q-pf-eta}
leads to simple, natural conditions, as in the next theorem.
Essentially, it says that for the networks considered there,
stability is obtained when server utilizations are less than 1.
\begin{shadethm}[Stability of a Simple Mixed Network]
Consider a mixed multi-class product form queuing
network and assume all stations are either Kelly
stations (such as Global PS or Global LCFS), IS
or MSCCC with constant rates.

Let $\calC$ be the set of classes that belong to
\imp{open} chains. Define the utilization factor
$\rho^s$ at station $s$ by
  \bearn
  \rho^s =  \frac{ \bar{S}^s}{B^s }
  \sum_{c \in \calC}\theta^s_c
  & &\mbox{ if station } s \mbox{ is MSCCC } \mbox{ with } B^s \mbox{
  servers, and mean service time }\bar{S}^s
  \\
  \rho^s  =  \sum_{c \in \calC}\theta^s_c \bar{S}^s_c
   & & \mbox{ if station } s \mbox{ is a Kelly station with mean
   service time } \bar{S}^s_c \mbox{ for class} c.
  \eearn

The network has a stationary distribution if and only if
$\rho^s <1$ for every station Kelly or MSCCC station $s$. There
is no condition on IS stations.
\label{theo-q-net-stabi-spec-simp}.
\end{shadethm}

%
%\mq{q-q-nc}{What is the normalizing constant
%$\eta$ of \thref{theo-q-pf} in the case of
%\coref{coro-q-pf-open}~?}{$\prod_{s=1}^S
%G^s(\vtheta^s)$.}
%%
%
%Formule de ma th\`{e}se:
%
%\bearn
% f(\vn) & = &
% \frac{n!}{\mu(1)...\mu(n)}\prod_{c=1}^C
% \frac{\lp\theta_c w_c\rp^{n_c}}{n_c!}
% \\
% w_c & = & 1 \mfor \mathrm{ type 1}
% \\
% w_c & = & \mu/\mu_c \mand \mu(n) = \mu \mfor \mathrm{ type 2}
% \\
%w_c & = & \mu/\mu_c \mand \mu(n) = \mu n  \mfor
%\mathrm{ type 3} \eearn
%
%Type 1 FIFO:
%
%  \bearn
%  f(\vn) & = &
% \frac{n!}{\mu(1)...\mu(n)}\prod_{c=1}^C
% \frac{\lp\theta_c \rp^{n_c}}{n_c!}
% \\
% \mu(n) & = & \mu \; r(n)\;\min(B,n)
%\eearn
%
%Type 2 PS
% \bearn
%  f(\vn) & = & n! \prod_{c=1}^C
%\frac{\theta_c^{n_c}}{\mu_c(1)...
%\mu_c(n_c)}\frac{1}{n_c !}
% \\
% \mu_c(i) & = & \mu_c r(i)
%  \eearn
%
%  Type 3 IS
% \bearn
%  f(\vn) & = &  \prod_{c=1}^C
%\frac{\theta_c^{n_c}}{\mu_c(1)...
%\mu_c(n_c)}\frac{1}{n_c !}
% \\
% \mu_c(i) & = & \mu_c r(i)
%  \eearn
\begin{ex}{Dual Core Processor in \fref{fig-q-chains}}
Let $q\in (0,1]$ be the probability that an external job uses
the critical section and $r>0$ be the average number of uses of
the processor outside the critical section by an external job.
Thus $\theta^1_1+\theta^1_3= \nu r$ and $\theta^1_2=\nu q$. By
\thref{theo-q-net-stabi-spec-simp}, the stability conditions
are
  \bearn
  \nu (r +q ) \bar{S}^1 & \leq & 2 \\
  \nu q \bar{S}^1& \leq & 1
  \eearn
where $\bar{S}^1$ is the average job processing
time at the dual core processor. Note that we
need to assume that the processing time is
independent of whether it uses the critical
section, and of whether it is an internal or
external job. Thus the system is stable (has a
stationary regime) for $\nu < \frac{2}{\bar{S}^1
\lp q + \max\lp r,q\rp\rp}$. Note that the
condition for stability bears only on external
jobs.

Let $K$ be the total number of class $4$ jobs; it
is constant since class $4$ constitutes a closed
chain. A state of the network is entirely defined
by the population vector $(n_1, n_2, n_3,
n_4,n^2_4)$; the number of jobs of class $4$ in
station $3$ is $K-n_4-n^2_4$, and $n^s_c=0$ for
other classes. The set of feasible states is
 \ben
 \calE = \lc (n_1, n_2, n_3, n_4, n^2_4) \in \Nats^5 \mst
  n_4+ n^2_4 \leq K  \rc
 \een

 The
joint stationary probability is
  \bearn
  \lefteqn{P(n_1, n_2, n_3, n_4,n^2_4)=\frac{1}{\eta(K)}
 d(n_2+n_4, n_1+n_3)}\\
 &&
 \times \frac{(n_1+n_3)!(n_2+n_4)!}{n_1!n_2!n_3!n_4!}(
 (\theta^1_1)^{n_1}(\theta^1_2)^{n_2}(\theta^1_3)^{n_3}
 (\bar{S}^1)^{n_1+n_2+n_3+n_4}(\bar{S}^2)^{n^2_4}
 \frac{1}{n^2_4!}(\bar{S}^3)^{K-n_4-n^2_4}
   \eearn
where we made explicit the dependency on $K$ in the normalizing
constant. This expression, while explicit, is too complicated
to be of practical use. In \exref{ex-dcp-th} we continue with
this example and compute the throughput, using the methods in
the next section. \label{ex-dcp-s}
  \end{ex}
\section{Computational Aspects}
\label{q-pf-algo}

As illustrated in\exref{ex-dcp-s}, the product form theorem,
though it provides an explicit form, may require a lot of work
as enumerating all states is subject to combinatorial
explosion, and the normalizing constant has no explicit form
when there are closed chains. Much research has been performed
on providing efficient algorithms for computing metrics of
interest for multi-class product form queuing networks. They
are based on a number of interesting properties, which we now
derive. In the rest of this section we give the fundamentals
ideas used in practical algorithms; these ideas are not just
algorithmic, they are also based on special properties of these
networks that are of independent interest.

 In the rest of this
section we assume that the multi-class product form queuing
network satisfies the hypotheses of the product form
theorem~\ref{theo-q-pf} as described in \sref{sec-qn}, and has
a stationary distribution (i.e. if there are open chains, the
stability condition must hold -- if the network is closed there
is no condition).

\subsection{Convolution}

\begin{shadethm}(\nt{Convolution Theorem}.)

Consider a multi-class product form queuing
network with closed and perhaps some open chains,
and let $\vK$ be the \nt{chain population vector}
of the \imp{closed} chains (i.e. $K_{\calC}$ is
the number of customers of chain $\calC$; it is
constant for a given network).

Let $\eta(\vK)$ be the normalizing constant given
in the product form theorem \ref{theo-q-pf}. Let
$\vY$ a formal variable with one component per
chain, and define
 \ben
 F_{\eta}(\vY)\eqdef \sum_{\vK \geq \vzero}\eta(\vK)
\prod_{\calC}Y_{\calC}^{K_{\calC}}
 \een

Then
 \be
 F_{\eta}(\vY) = \prod_{s=1}^S G^s(\vZ^s)
 \label{eq-q-qnet-convol-zt}
 \ee
 where $G^s$ is the generating function of the
 station function for station $s$, and $\vZ^s$ is
 a vector with one component per class, such that
 \bearn
 Z^s_c &=& Y_{\calC} \theta^s_c  \mbox{ whenever }
 c \in\calC \mand \calC \mbox{ is closed}
 \\
 Z^s_c &=& \theta^s_c  \mbox{ whenever }
 c \mbox{ is in an open chain}
 \eearn
\label{theo-gf-eta}
\end{shadethm}

The proof is a direct application of the product form theorem,
using generating functions. \eref{eq-q-qnet-convol-zt} is in
fact a \nt{convolution equation}, since convolution translates
into product of generating functions. It is the basis for the
\nt{convolution algorithm}, which consists in adding stations
one after another, see for example \cite{balsamo2000product}
for a general discussion and \cite{le1988multibus} for networks
with MSCCC stations other than FIFO. We illustrate the method
in \exref{ex-dcp-th} below.

%
%The convolution theorem \ref{theo-gf-eta} is
%often interpreted in terms of the complement
%network property, defined as follows. To an
%arbitrary station $s_0$ we associate the
%\nt{complement network} \cite{reiser1981mean}, in
%which customers arriving at station $s_0$
%instantly proceeding to the next station,
%according to the routing matrix previously
%defined. The chains and classes are the same as
%in the in the original network; the visit rates
%$\theta^{s}_c$ are the same in both networks. Be
%careful that this is different from the procedure
%used in the station in isolation. Here, we are
%replacing station $s_0$ by a short-circuit, i.e.
%a station where the service requirement is $0$;
%in contrast, with the station in isolation, we
%replace a network part by a station with unit
%rate and unit service requirement.
%
%%\begin{shadethm}(\nt{Complement Network Theorem})
%With the same assumptions and notation as in
%\thref{theo-gf-eta}, fix some arbitrary station
%$s_0$ and let $\eta()$ be the normalizing
%constant for the original network and
%$\eta^{[s_0]}()$ for the complement network of
%station $s_0$.
%
%Then it follows immediately from
%\eref{eq-q-qnet-convol-zt} that
%  \be
%  F_{\eta}(\vY) = G^{s_0}(\vZ^{s_0})F_{\eta^{[s_0]}}(\vY)
%  \ee
%This can also be written explicitly as
% \be
% \eta(\vK) =
% \sum_{\vn^{s_0} \mst \vzero \leq\kappa(\vn^{s_0}) \leq \vK}
% f^{s_0}(\vn^{s_0})\prod_{c=1}^C\lp
% \theta^{s_0}_c\rp^{n^{s_0}_c} \eta^{[s]}\lp \vK-\kappa(\vn^{s_0})\rp
% \label{eq-q-qnet-convolkjsaj}
%  \ee
%  where
%$\kappa(\vn^{s_0})$ is the closed chain
%population vector that corresponds to state
%$\vn^{s_0}$, i.e.
%$\kappa(\vn^{s_0})_{\calC}=\sum_{c \in \calC}
%n_c$ (in particular, $\vK-\kappa(\vn^{s_0})$ is
%the closed chain population vector obtained when
%we remove $n^{s_0}_c$ customers of class $c$ for
%every $c$). The notation $\vzero \leq
%\kappa(\vn^{s_0}) \leq \vK$ means
%  $0 \leq \kappa(\vn^{s_0})_{\calC}\leq K_{\calC}$
%  for all chain $\calC$.
%
%The marginal probability distribution of station
%$s_0$ satisfies can be derived: for any $s_{0}$:
% \be
% P^{s_0}\lp\vn^{s_0}\left|\vK\rp\right. =
% f^{s_0}(\vn^{s_0})\prod_{c=1}^C\lp
% \theta^{s_0}_c\rp^{n^{s_0}_c}
% \frac{\eta^{[s_0]}\lp \vK-\kappa(\vn^{s_0})\rp}{\eta(\vK)}
% \label{eq-qnet-convol}
% \ee
% %\label{theo-q-qnet-compnet}
%%\end{shadethm}
%
%In particular the probability that station $s_0$
%is empty is \be
%P^{s_0}\lp\vzero\left|\vK\rp\right. =
% \frac{\eta^{[s_0]}\lp \vK \rp}{\eta(\vK)}
% \label{eq-q-convol-p0}
% \ee
\subsection{Throughput}
\label{sec-q-pcth}
Once the normalizing constants are computed, one
may derive throughputs for class $c$ at station
$s$, defined as the mean number of class $c$
arrivals at (or departures from) station $s$:
\begin{shadethm}(\nt{Throughput Theorem} \cite{buzen1973computational})
The throughput for class $c$ of the closed chain
$\calC$ at station $s$ is
 \be
 \la_c^s(\vK) = \theta_c^s \frac{\eta\lp \vK -\vone_{\calC}\rp}{\eta(\vK)}
 \label{eq-q-th-f}
 \ee
 \label{theo-q-qnet-th}
\end{shadethm}
It follows in particular that, for closed chains, the
throughputs at some station \imp{depend only on the throughput
per class and the visit rates}. Formally, choose for every
closed chain $\calC$ a station $s_0(\calC)$ effectively visited
by this chain (i.e. $\sum_{c \in \calC}\theta^{s_0}_{c}>0$);
define the \nt{per chain throughput} $\calC$ as the throughput
at this station $\la_{\calC}(\vK)\eqdef\sum_{c
\in\calC}\la^{s_0(\calC)}_{c}(\vK)$. Since for closed chains
the visit rates $\theta^s_c$ are determined up to a constant,
we may decide to let $\sum_{c \in
\calC}\theta^{s_0(\calC)}_{c}=1$, and then for all class~$c\in
\calC$ and station~$s$:
  \be
  \la^s_c(\vK) = \la_{\calC}(\vK)\theta^s_c
\label{eq-q-qnet-dskjsdjkjkdf}
  \ee
Also, the equivalent of \eref{eq-q-th-f} for the
per chain throughput is
 \be
 \la_{\calC}(\vK) = \frac{\eta\lp \vK -\vone_{\calC}\rp}{\eta(\vK)}
 \label{eq-q-th-f-ch}
 \ee
 (which follows immediately by summation on $c
 \in\calC$).
%for every chain $\calC$, define the per chain
%throughput at station $c$ by
%$\la^s_{\calC}\eqdef\sum_{c \in \calC}\la^s_c$
%and the per chain visit rate as
%$\theta^s_{\calC}\eqdef\sum_{c \in
%\calC}\theta^s_c$ station \be
% \la_c^s(\vK) = \theta_c^s \la^s_{\calC}
% \ee

Note that the throughput for a class $c$ of an
\emph{open} chain is simply the visit rate
$\theta^s_c$.

Last but not least, the throughput depends only
on the normalizing constants and not on other
details of the stations. In particular, stations
that are different but have the same station
function (such as FIFO with one server and
constant rate Kelly function with class
independent service time) give the same
throughputs.

The next example illustrates the use of the above
theorems in the study of a general case (a mixed
network with an MSCCC station). There are many
optimizations of this method, see \cite{359020}
and references therein.
\begin{figure}
\centering

 \subfigure[$x=0.7$, $y=0.8$]{\ifig{throughput-x=0.7y=0.8}{0.5}}
 \subfigure[$x=0.7$, $y=1$]{\ifig{throughput-x=0.7y=1}{0.5}}
 \subfigure[$x=0.5$, $y=1.2$]{\ifig{throughput-x=0.5y=1.2}{0.5}}
  \mycaption{Throughput $\la$ of internal jobs for the
 dual core processor in \fref{fig-q-chains}, in jobs per millisecond,
  as a function
 of the number of internal jobs. Dotted curve:
throughput that would be achieved if the internal
jobs would not use the critical section, i.e. any
job could use a processor when one is idle. $x$
is the intensity of external traffic that uses
the critical section and $y$ of other external
traffic. There are two constraints : $x + \la
\leq 1$ (critical section) and $x + y + \la \leq
2$ (total processor utilization). For the dotted
line only the second constraint applies. In the
first panel, the first constraint is limiting and
the difference in performance is noticeable. In
the last panel, the second constraint is limiting
and there is little difference. In the middle
panel, both constraints are equally limiting.
$\bar{S}^1=1, \bar{S}^2=5, \bar{S}^3=1$msec.
}\label{fig-q-qnet-ex-dcp-th}
\end{figure}


\begin{ex}{Dual Core Processor in \fref{fig-q-chains}, Algorithmic Aspect}
We continue \exref{ex-dcp-s}. Assume now that we
let all parameters fixed except the arrival rate
$\la$ of external jobs and the number $K$ of
internal jobs; we would like to evaluate the
throughput $\mu$ of internal jobs as a function
of $\la$ and $K$ as well as the distribution of
state of internal jobs.

We can use the throughput theorem and obtain that
the throughput $\la(K)$ for class $4$ is (we drop
the dependency on $\la$ from the notation)  \be
\la(K)
=\frac{\eta(K-1)}{\eta(K)}\label{eq-q-ex-mu}\ee

We now have to compute the normalizing constant
$\eta(K)$ as a function of $K$. To this end, we
use the convolution equation
\eref{eq-q-qnet-convol-zt}:
  \be
  F_{\eta}(Y) = G^1(\vZ^1)G^2(\vZ^2)G^3(\vZ^3)
  \ee
  with
  \bearn
  \vZ^1 & = & (\theta^1_1, \theta^1_2,
  \theta^1_3, Y)
  \\
  \vZ^2 & = & (0,0,0,Y)
  \\
  \vZ^3 & = & (0,0,0,Y)
    \eearn
The generating functions $G^1, G^2, G^3$ are
given in \exref{ex-q-qnet-dslkjjdasflk}. It
comes:
 \be
F_{\eta}(Y) =  D(Y\bar{S}^1+x,y) e^{\bar{S}^2Y}
      \frac{1}{1-\bar{S}^3 Y}
 \label{eq-q-qnet-sdjsduuepoowjhjdkjhfd}
 \ee
 %\bearn
% F_{\eta}(Y) &= &  D\lp\bar{S}^1 (Y + \la q),
% \bar{S}^1 \la r\rp
%      e^{\bar{S}^2Y}
%      \frac{1}{1-\bar{S}^3 Y}
%      \\
%      & = &
%\frac{1}{2-x-y-Y \bar{S}^1}
%  \lp
%  \frac{2-y}
%  {1-x-Y \bar{S}^1}+\frac{(2-x-Y \bar{S}^1)(2+y)}{2-y}-2
%  \rp
%  \frac{e^{\bar{S}^2Y}}{1-\bar{S}^3
% Y}
% \eearn
with $x=\nu q \bar{S}^1$, $y=\nu r \bar{S}^1$ and $D()$
defined in \eref{eq-q-ex-phi-skl}.

We can compute $\eta(K)$ by performing a power series expansion
(recall that $F_{\eta}(Y)=\sum_{K\in \Nats} \eta(K)Y^K$) and
find $\eta(K)$ numerically. Alternatively, one can interpret
\eref{eq-q-qnet-sdjsduuepoowjhjdkjhfd} as a convolution
equation $\eta = \eta_1 \star \eta_2 \star\eta_3$ with
$F_{\eta_1}(Y)=  \sum_{k \in \Nats}\eta_1(k) Y^k\eqdef
 D(Y\bar{S}^1+x,y)$, $F_{\eta_2}(Y) =e^{\bar{S}^2Y}$,
$F_{\eta_3}(Y) = \frac{1}{1-\bar{S}^3 Y}$ and use fast
convolution algorithms or the \pro{filter} function as in
\exref{ex-q-qnet-dsjkhuzewnnbasdio}. The throughput for
internal jobs follows from \eref{eq-q-ex-mu} and is plotted in
\fref{fig-q-qnet-ex-dcp-th}.
 \label{ex-dcp-th}
\end{ex}
%
%
\subsection{Equivalent Service Rate} This is a
useful concept, which hides away the details of a
station and, as we show in the next section, can
be used to aggregate network portions.
Consider some arbitrary station $s$, of any
category, with station function $f^s()$. We call
\nt{equivalent service rate} for class $c$
at station $s$ the quantity
 \be
  \mu^{*s}_c(\vn^s)\eqdef \frac{f^s(\vn^s -\vone_c)}{f^s(\vn^s)}
\label{eq-q-qnet-eqsrvrat}
   \ee
It can be shown that
$\mu^{*s}_c(\vn^s)$ is indeed the average rate at which customers of class $c$ depart from station $s$ when station $s$ is imbedded in a multi class queuing network and given that the numbers of customers at station $s$ is $\vn^s$, i.e.
  \ben
  \mu^{*s}_c(\vn^s)=\sum_{\ve \in \calE(s, \vn^s)}\sum_{\vf \in \calE'(s,\ve)}P(\ve) \mu(\ve,\vf)
  \een
where $\ve$ is a global micro state of the network (see \sref{sec-micro} for a definition), $\calE(s,\vn^s)$ is the set of global micro-states for which the population vector at station $s$ is $\vn^s$, $\calE'(s, \ve)$ is the set of of global micro-states such that the transition $\ve \to \vf$ is a departure from station $s$, $P()$ is the stationary probability of the network and $\mu(\ve,\vf)$ is the transition rate. This is true as long as the network satisfies the hypotheses of the product form theorem, and is a direct consequence of the local balance property.

To $s$
we associate a \emph{per class PS} station with unit service
requirement for all classes and with balance function $f^s(\vn^s)$.
 This virtual station
is called the \nt{equivalent station} of station
$s$. By construction, it is a category 1 station
and, by \eref{eq-q-stf-t2sspc} the station functions
of this virtual station and of $s$ are identical. Further, the rate of service allocated to customers of class $c$ is also $\mu^{*s}_c(\vn^s)$. Thus, as far as the stationary probability of
customers is concerned, using the original
station or the equivalent station inside a network make no
difference. We have an even stronger result.

\begin{shadethm}(Equivalent Station Theorem \cite{MR563738})
In a multi-class product form queuing network any station can
be replaced by its equivalent station, with equivalent service
rate as in \eref{eq-q-qnet-eqsrvrat} so that the stationary
probability and the throughput for any class at any station are
unchanged.
\end{shadethm}
Note that the equivalent station and the
equivalent service rate depend only on the
station, not on the network in which the station
is imbedded. It is remarkable that it is thus
possible to replace \emph{any} station by a per
class PS station. Note however that the
equivalence is only for distributions of numbers
of customers and for throughputs, not for delay
distributions; indeed, delays depend on the
details of the station, and stations with same
station function may have different delay
distributions.

The equivalent service rates for a few frequently
used stations are given in
\tref{table-q-qnet-eqsrvrat}. For some stations
such as the general MSCCC station there does not
appear to be a closed form for the equivalent
service rate.
\begin{table}
  \centering
\begin{tabular}{|c|c|}
  % after \\: \hline or \cline{col1-col2} \cline{col3-col4} ...
  \hline
  \emph{Station $s$} & \emph{Equivalent Service Rate $\mu^{*s}_c(\vn^s)$} \\
  \hline
  \hline
 \begin{minipage}[m]{0.50\textwidth}~\\
  Kelly Stations with Class Dependent Service
  Rate.
    Recall that this contains as special cases Global PS and
    Global LCFSPR stations with constant rate. ~\\
 \end{minipage}
 & $r_c^s(n^s_c)
  \frac{n^s_c}{\abs{\vn^s}}\frac{1}{\bar{S}^s_c}$
 \\
 \hline
\begin{minipage}[m]{0.50\textwidth}~\\
Kelly Stations with Queue Size Dependent Service
Rate.~\\
 \end{minipage}
 &
 $r^s(\abs{\vn^s})
  \frac{n^s_c}{\abs{\vn^s}}\frac{1}{\bar{S}^s_c}$
\\
\hline
\begin{minipage}[m]{0.50\textwidth}~\\
IS station with Class Dependent Service Rate]
 ~\\
 \end{minipage}
 &
 $r_c^s(n^s_c)
  n^s_c\frac{1}{\bar{S}^s_c}$
\\
\hline
\begin{minipage}[m]{0.50\textwidth}~\\
IS station with Queue Size Dependent Service
Rate.~\\
 \end{minipage}
 &
 $r^s(\abs{\vn^s})
  n^s_c\frac{1}{\bar{S}^s_c}$
\\
 \hline
\begin{minipage}[m]{0.50\textwidth}~\\
FIFO
 station with $B$ servers and queue size dependent service
 rate. Recall that this is a
 station of Category 2 hence the service
 requirement is exponentially distributed and has
 the same mean $\bar{S}^s$ for all classes.
~\\
 \end{minipage}
 &
 $\frac{1}{\bar{S}^s}\min\lp B,
\abs{\vn^s}\rp
    r(\abs{\vn^s})$
\\
 \hline
\end{tabular}
  \mycaption{Equivalent service rates for frequently used
  stations.
Notation: $\vn^s=(n^s_1, ...,n^s_C)$ with $n^s=$
number of class $c$ customers at station $s$;
$\bar{S}^s_c$ is the mean service requirement;
$r^s_c(n^s_c)$ is the rate allocated to a class
$c$ customer when the service rate is class
dependent; $r^s(\abs{\vn^s})$ is the rate
allocated to any customer when the service rate
depends on queue size; $\abs{\vn^s}$ is the total
number of customers in station $s$. For a
constant rate station, take $r^s_c()=1$ or
$r^s()=1$.}\label{table-q-qnet-eqsrvrat}
\end{table}
%
The equivalent service rate is used in the
following theorem.
\begin{shadethm}(\cite{reiser1975queuing})

Consider a multi-class product form queuing
network with closed and perhaps some open chains,
and let $\vK$ be the \nt{chain population vector}
of the \imp{closed} chains. For any class $c$ of
the closed chain $\calC$ and any station $s$, if
$n^s_c\geq 1$:
 \be
   P^s\lp\vn^s\left|\vK\rp\right.  =
   P^s\lp\vn^s-\vone_c\left|\vK-\vone_{\calC}\rp\right.
   \frac{1}{\mu^{*s}_c(\vn^s)}
   \la^s_c(\vK) \label{eq-q-mva-bas}
  \ee

  where $P^s\lp.\left|\vK\rp\right.$ is the
  marginal probability at station $s$ and
  $\la^s_c(\vK)$ is the throughput for class $c$
  at station $s$.
  \label{theo-q-qnet-adsjkhfuierbhfv}
\end{shadethm}

This theorem is useful if the equivalent service rate is
tractable or is numerically known. It can be used if one is
interested in the marginal distribution of one station; it
requires computing the throughputs $\la(\vK)$, for example
using convolution or MVA. \eref{eq-q-mva-bas} can be used to
compute $P^s\lp\vn^s\left|\vK\rp\right.$ iteratively by
increasing the populations of closed chains
\cite{reiser1981mean}. Note that it does not give the
probability of an empty station; this one can be computed by
using the fact that the sum of probabilities is $1$.

\begin{figure}
\centering
 \Ifignc{rsp-x=0.7y=0.8}{0.75}{0.4}
 \Ifignc{pdf-x=0.7y=0.8}{0.95}{0.4}
  \mycaption{First panel:
Mean Response time for internal jobs at the dual
core processor, in millisecond, as a function of
the number $K$ of internal jobs. Second panel:
stationary probability distribution of the number
of internal jobs at stations 1 to 3, for $K=10$.
(Details of computations are in
Examples~\ref{ex-q-qnet-dsjkldsjkfl}
and~\ref{ex-q-qnet-dsjkhuzewnnbasdio};
$\bar{S}^1=1, \bar{S}^2=5, \bar{S}^3=1$msec,
$x=0.7$, $y=0.8$.)}\label{fig-q-qnet-ex-dcp-eqsr}
\end{figure}

\begin{ex}{Dual Core Processor in \fref{fig-q-chains}, continued}
We now compute the stationary probability that
there are $n$ jobs in station 2 given that there
are $K$ internal jobs in total. By
\eref{eq-q-mva-bas}:
 \be
 P^2(n|K) = P^2(n-1|K-1) \la(K)
 \frac{\bar{S}^2}{n}
 \ee
since the equivalent service rate for station 2
(which is an IS station) is
 $\frac{n}{\bar{S}^2}$ when there are $n$ customers in the
 station.
  This gives $P^2(n|K)$ for
 $1 \leq 1 \leq K$ if we know $P^2(.|K-1)$;
 $P(0|K)$ is obtained by the normalizing
 condition\ben \sum_{n=0}^K P^2(n|K)=1\een
We compute $P^2(.|K)$ by iteration on $K$,
starting from $P^2(0|0)=1$ and using the previous
two equations. The mean number of jobs is station
2 follows:
 \be \bar{N}^2(K) = \sum_{n=0}^K n P(n|K)
 \ee
Similarly for station 3, with
 \be
 P^3(n|K) = P^3(n-1|K-1) \la(K) \bar{S}^3
 \ee
since the equivalent service rate for station 3
(which is a PS station) is
 $\frac{1}{\bar{S}^3}$. The mean number of
 internal jobs in station $1$ follows:
 $\bar{N}^1_4(K)=K-\bar{N}^2(K)-\bar{N}^3(K)$.

We derive the mean response times for internal
jobs in stations 1 to 3  by using Little's law:
$\bar{R}^s_4(K) =
  \frac{\bar{N}^s(K)}{\la(K)}$ for $s=1,2,3$.

By Little's law, $(R^1_4+R^2_4+R^3_4)\la =K$; for
large $K$, $\la \approx
\theta_{\max}=\min(1-x,2-x-y)$ and $R^2_4\approx
\bar{S}^2$, $R^3_4\approx \bar{S}^3$ (most of the
queuing is at station 1), thus $\bar{R}^1_4(K)
\approx
\frac{K}{\theta_{\max}}-\bar{S}^2-\bar{S}^3$ for
large $K$. The results are shown in
\fref{fig-q-qnet-ex-dcp-eqsr}.
\label{ex-q-qnet-dsjkldsjkfl}
  \end{ex}

%
%
%the of category 1 and a class $c$, the
%\nt{equivalent service rate} for class $c$ by
% \be
%   \mu^{*s}_c(\vn^s)\eqdef \frac{f^s(\vn^s -\vone_c)}{f^s(\vn^s)}
%   \ee
%where $f^s()$ is the station function. The
%intuition for the equivalent service rate is the
%rate at which station $s$ outputs a class $c$
%customer. The equivalent service rate can be
%readily derived from the station function, and
%has a simple form for a few stations:
%\begin{description}
%    \item[Kelly Stations with Constant or Class Dependent Service Rate]
%    Recall that this contains as special cases Global PS and
%    Global LCFSPR stations with constant rate.
%The equivalent service rate is directly derived
%from the station function and is
% \be
%  \mu^{*s}_c(\vn^s) = r_c^s(n^s_c)
%  \frac{n^s_c}{\abs{\vn^s}}\frac{1}{\bar{S}^s_c}
% \ee where $r^s_c(n^s_c)$ is the rate allocated in
% station $s$ to
% a class $c$ customer when there are $n_c$
% customers in station $s$, $\abs{\vn^s}$ is the total
% number of customers in station $s$, and
% $\bar{S}^s_c$ is the mean service requirement
% for class $c$ at station $s$.
%    \item[Kelly Stations with Queue Size Dependent Service Rate]
%    This is similar, with
%    \be
%  \mu^{*s}_c(\vn^s) = r^s(\abs{\vn^s})
%  \frac{n^s_c}{\abs{\vn^s}}\frac{1}{\bar{S}^s_c}
% \ee
%
%    \item[IS Station] If the service rate is class dependent
% then \be
%  \mu^{*s}_c(\vn^s) = r_c^s(n^s_c)
%  n^s_c\frac{1}{\bar{S}^s_c}
% \ee
%if it is queue size dependent then \be
%  \mu^{*s}_c(\vn^s) = r^s(\abs{\vn^s})n^s_c\frac{1}{\bar{S}^s_c}
% \ee
% \item[FIFO station] Recall that this is a
% station of Category 2 hence the service
% requirement is exponentially distributed and has
% the same mean $\bar{S}^s$ for all classes. For a FIFO
% station with $B$ servers and with queue size dependent service
% rate one finds
% \be \mu^{*s}_c(\vn^s)=\frac{1}{\bar{S}^s}\min\lp B,
%\abs{\vn^s}\rp
%    r(\abs{\vn^s})
%    \ee
%\end{description}
%
%
%
\subsection{Suppression of Open Chains} In an
open network, the product form theorem implies
that all stations are independent in the
stationary regime, and thus the network is
equivalent to a collection of stations in
isolation. In the mixed or closed case, this does
not hold anymore, and station states are mutually
dependent.

It is possible to simplify mixed networks by removing open
chains. In the modified network, there are only closed chain
customers, with the same routing matrix $q^{s,s'}_{c,c'}$ for
all $c,c'$ in closed chains; the stations are the same, but
with a modified station function. Let $G^s(\vZ)$ be the $z$
transform of the station function and $\theta^s_c$ the visit
rates in the original network with open chains. In the modified
network, the $z$ transform of the station function is
  \be
  G'^s(\vZ) =G^s(\vZ') \mwith
  \bracket{
  Z'_c = Z_c \mif c \mbox{ is in a closed
  chain}\\
  Z'_c = \theta^s_c \mif c \mbox{ is in an open chain}
  }
  \ee
In the above, $\vZ$ is a vector with one
component per class in a closed chain, whereas
$\vZ'$ has one component per class, in any open
or closed chain.
 \begin{shadethm}(\nt{Suppression of Open
 Chains})
Consider a mixed multi-class network that
satisfies the hypotheses of the product form
theorem~\ref{theo-q-pf}. Consider the network
obtained by removing the open chains as described
above. In the modified network the stationary
probability and the throughputs for classes of
closed chains are the same as in the original
network. \label{theo-q-qnet-rem-op}
 \end{shadethm}

The proof is by inspection of the generating
functions. Note that the modified stations may
not be of the same type as the original ones;
they are fictitious stations as in the equivalent
station theorem. Also, the equivalent service
rates of the modified stations depend on the
visit rates of the open chains that were removed,
as illustrated in the next example.

\begin{ex}{Dual Core Processor in \fref{fig-q-chains}, continued}
We now compute the stationary probability at station 1. We
suppress the open chains and compute the equivalent service
rate at station 1. We have now a single chain, single class
network, with only customers of class $4$. Stations 2 and 3 are
unchanged; station 1 is replaced by the station with generating
function:
  \ben
  G'^1(Z) = G^1(\theta^1_1, \theta^1_2, \theta^1_3,
  Z)
  \een where $G^1$ is given in
  \eref{eq-q-qnet-djjduetgsvdff}.
With the same notation as in \exref{ex-dcp-th}, $G'^1(z)=
 D(Z\bar{S}^1+x,y)$ with $ D$ given by
\eref{eq-q-ex-phi-skl}, and thus
 \be
  G'^1(Z) = \frac{1}{1-x-Z\bar{S}^1} \lp
   1 + y + \frac{y^2}{2-x-y- Z\bar{S}^1}\rp
 \ee
The station function $f'^1(n)$ of the modified station 1 is
obtained by power series expansion $G'^1(Z)=\sum_{n\geq
0}f'^1(n) Z^n$. modified as follows. Since $G'^1$ is a rational
function (quotient of two polynomials), its power series
expansion can be obtained as the impulse response of a filter
with rational $z$ transform (\sref{sec-z-tr}). Consider the
filter
  \be
 \frac{1}{1-x-B\bar{S}^1} \lp
   1 + y + \frac{y^2}{2-x-y- B\bar{S}^1}\rp
  \ee where $B$ is the backshift operator. The
  sequence $(f'^1(0), f'^1(1), f'^1(2)...)$ is
  the impulse response of this filter, and can be
obtained easily with the \pro{filter} function of
matlab. The equivalent service rate of station 1
for internal jobs is
  \be
  \mu'^1(n) = \frac{f'^1(n-1)}{f'^1(n)}
  \ee
Since we know the equivalent service rate, we can
obtain the probability distribution $P'^1(n)$ of
internal jobs at station $1$ using
\thref{theo-q-qnet-adsjkhfuierbhfv} as in
\exref{ex-q-qnet-dsjkldsjkfl}. The results are
shown in \fref{fig-q-qnet-ex-dcp-eqsr}.
 \label{ex-q-qnet-dsjkhuzewnnbasdio}
  \end{ex}

\subsection{Arrival Theorem and MVA Version 1}
%\subsubsection{Arrival Theorem}
%
%\subsubsection{Simple MVA for Networks With Single Server FIFO, IS or similar Stations}
\label{sec-mva} \nt{Mean Value Analysis} (MVA)\index{MVA} is a
method, developed in \cite{reiser1980mean}, which does not
compute the normalizing constant and thus avoids potential
overflow problems. There are many variants of it,
 %such as the
% Multibus algorithm \cite{le1988multibus} for networks with
% MSCCC stations other than FIFO;
see the discussion in \cite{balsamo2000product}.

In this chapter we give two versions. The former,
described in this section is very simple, but
applies only to some station types, as it
requires to be able to derive the response time
from the Palm distribution of queue size upon
customer arrival. The second, described in
\sref{sec-mva-2} is more general and applies to
all stations for which the equivalent service
rate can be computed easily.

MVA version 1 is based on the following theorem,
which is a consequence of the product form
theorem and the embedded subchain theorem of Palm
calculus (\thref{theo-incluse}).

\begin{shadethm}(\nt{Arrival Theorem})

Consider a multi-class product form queuing. The probability
distribution of the numbers of customers seen by customer just
before arriving at station $s$ is the stationary distribution
of
\begin{itemize}
    \item the same network
if the customer belongs to an open chain;
    \item the network with one customer less in
    its chain, if the customer belongs to a closed
    chain.
\end{itemize}\label{theo-arrival}
\end{shadethm}

Consider now a \emph{closed} network where all
stations are FIFO or IS with constant rate, or
are equivalent in the sense that they have the same
station function as one of these (thus have the same equivalent service rate). Indeed, recall
that stationary probabilities and throughput
depend only on the station function. For example,
a station may also be a global PS station with
class independent service requirement of any
phase type distribution, which has the same
station function as a FIFO station with one
server and exponential service time. In the rest
of this section we call ``FIFO" [resp. IS]
station one that has the same station function as
a single server, constant rate FIFO [resp. IS]
station. Recall that at a FIFO station we need to
assume that the mean service requirements are the
same for all classes at the same station; for the
IS station, it may be class dependent.

First we assume the FIFO [resp. IS] stations are
truly FIFO [resp.IS], not just equivalent stations
as defined above. We will remove this restriction
later. Let $\bar{N}^s_c(\vK)$ be the mean number
of class $c$ customers at station $s$ when the
chain population vector is $\vK$. The mean
response time for a class $c$ customer at a FIFO
station $s$ when the population vector is $\vK$
is
 \ben
 \bar{R}^s_c(\vK) = \lp 1+
 \sum_{c}\bar{N}^s_c(\vK-\vone_{\calC})
 \rp
 \bar{S}^s
 \een where $\calC$ is the chain of class $c$.
This is because of the exponential service
requirement assumption: an
 arriving customer has to wait for $\bar{S}^s$ multiplied by the
 number of customers present upon arrival; in average, this latter
 number is
 $\sum_{c}\bar{N}^s_c(\vK-\vone_{\calC})$
by the arrival theorem. By Little's formula:
 \ben
\bar{R}^s_c(\vK) \la^s_c(\vK)=\bar{N}^s_c(\vK)
 \een
Combining the two gives
  \be
 \bar{N}^s_c(\vK) = \la^s_c(\vK) \lp 1+
 \sum_{c}\bar{N}^s_c(\vK-\vone_{\calC})
 \rp \bar{S}^s
  \ee
which is valid for FIFO stations. For a delay
station one finds
 \be
 \bar{N}^s_c(\vK) = \la^s_c(\vK)\bar{S}^s_c
 \ee
This gives a recursion for $\bar{N}^s_c(\vK)$ if
one can get determine $\la^s_c(\vK)$. The next
observation is \eref{eq-q-qnet-dskjsdjkjkdf},
which says that if we know the throughput at one
station visited by a chain, then we know the
throughputs for all stations and all classes of
the same chain. The last observation is that the
sum of the numbers of customers across all
stations and all classes of chain $\calC$ is
equal to $K_{\calC}$. Combining all this gives:
for every chain $\calC$, if $K_{\calC}>0$ then
  \bear
  \frac{K_{\calC}}{\la_{\calC}( \vK)} &=&
\sum_{c \in \calC}
  \lb
   \sum_{s: \mathrm{FIFO}}\theta^s_c
   \lp 1+
 \sum_{c'}\bar{N}^s_{c'}(\vK-\vone_{\calC})
 \rp \bar{S}^s
  +
    \sum_{s: \mathrm{IS}}\theta^s_c
    \bar{S}^s_c
    \rb
    \label{eq-q-qnet-mva-dlsld}
  \eear
 and \be \la_{\calC}( \vK)= 0 \mif
 K_{\calC}=0 \ee
For every FIFO station $s$ and class $c$:
 \bear
 \bar{N}^s_c(\vK) &= &\theta^s_c\la_{\calC(c)}
 (\vK) \lp 1+
 \sum_{c'}\bar{N}^s_{c'}(\vK-\vone_{\calC(c)})
 \rp \bar{S}^s
 \mif K_{\calC(c)} >0  \label{eq-q-qnet-mva-dlsdsdld}
 \\
 & = & 0 \mif  K_{\calC(c)} =0
 \eear
%and for every IS station $s$ and class $c$
% \bear
% \bar{N}^s_c(\vK)& = &\theta^s_c\la_{\calC(c)}(\vK) \bar{S}^s_c
% \\
% & = & 0 \mif  K_{\calC(c)} =0
% \eear

Second, we observe that the resulting equations
depend only on the station function, therefore
they apply to equivalent stations as well.


The \nt{MVA algorithm version 1} iterates on the
total population, adding customers one by one. At
every step, the throughput is computed using
Equation~(\ref{eq-q-qnet-mva-dlsld}). Then the
mean queue sizes at FIFO queues are computed
using Equation~(\ref{eq-q-qnet-mva-dlsdsdld}),
which closes the loop. We give the algorithm in
the case of a single chain. For the multi-chain
case, the algorithm is similar, but there are
many optimizations to reduce the storage
requirement, see \cite{balsamo2000product}.
%
 \begin{algorithm}\mycaption{MVA Version 1: Mean Value Analysis for a single chain
 closed
 multi-class
 product form queuing network containing only constant rate
 FIFO and IS stations, or stations with same station functions.}
 \begin{algorithmic}[1]
%
  \State $K=$ population size
  \State $\la=0$ \Comment{throughput}
  %\State $N^s_c=0$ for all class $c$ and station
%  $s$ \Comment{mean numbers of customers}
  \State $Q^s=0$ for all station $s \in \mathrm{FIFO}$
  \Comment{total number of customers at station $s$, $Q^s=\sum_c \bar{N}^s_c$}
%
  \State Compute the visit rates $\theta^s_c$
  using \eref{eq-q-visitRatios} and
  $\sum_{c=1}^C\theta^1_c=1$
  \State $\theta^s=\sum_{c}\theta^s_c$ for every
  $s\in \mathrm{FIFO}$
  \State $h=\sum_{s \in \mathrm{IS}}\sum_{c}\theta^s_c
  \bar{S}^s_c  + \sum_{s \in
  \mathrm{FIFO}}\theta^s  \bar{S}^s$
  \Comment{constant term in \eref{eq-q-qnet-mva-dlsld}}
%
    \For{$k=1:K$}
    \State
    $\la=\frac{k}{h+\sum_{s \in \mathrm{FIFO}}
    \theta^s  Q^s \bar{S}^s}$
    \Comment{\eref{eq-q-qnet-mva-dlsld}}
    \State $Q^s=\la \theta^s
    \bar{S}^s (1 + Q^s)$ for all
    $s \in \mathrm{FIFO}$
  \EndFor
  \State The throughput at station $1$ is $\la$
  \State The throughput of class $c$ at station $s$ is $\la
 \theta^s_c$
  \State The mean number of customers of class $c$ at FIFO
  station
  $s$ is $Q^s\theta^s_c/\theta^s$
  \State The mean number of customers of class $c$ at
  IS
  station
  $s$ is $\la \theta^s_c \bar{S}^s_c$
\end{algorithmic}\label{algo-q-mva}
 \end{algorithm}
\begin{figure}[htbp]
   \Ifig{botana}{0.8}{0.4}
   \mycaption{Throughput in transactions per second versus number of users,
   computed with MVA for the network
   in \fref{fig-bottleneckAna}. The dotted lines are
   the bounds of bottleneck analysis in \fref{fig-bot-lamres}.}
  \mylabel{fig-q-qnet-botana}
\end{figure}


\begin{ex}{Mean Value Analysis of \fref{fig-bottleneckAna}}
We model the system as a single class, closed
network. The CPU is modelled as a PS station,
disks A and B as FIFO single servers, and think
time as an IS station. We fix the visit rate
$\theta^{\mathrm{think time}}$ to $1$ so that
$\theta^{\mathrm{CPU}}=V_{\mbox{CPU}}$,
$\theta^{\mathrm{A}}=V_{\mbox{A}}$ and
$\theta^{\mathrm{B}}=V_{\mbox{B}}$. Note that the
routing probabilities need not be specified in
detail, only the visit rates are required.

The CPU station is not a FIFO station, but is has
the same station function, therefore we may apply
MVA and treat it as if it would be FIFO.

\fref{fig-q-qnet-botana} shows the results, which
are essentially as prediced by bottleneck
analysis in \fref{fig-bot-lamres}.

\label{ex-q-qnet-djerzcbasvvajhqoejfz}
\end{ex}

\subsection{Network Decomposition}
%One often wishes to compute the stationary
%probability for \emph{one} station or for one
%group of stations. A key property of product
%form, i.e. the marginal of the joint
%distribution. In an open network, this is given
%by the station function. In a closed or mixed
%network, one may use \eref{eq-qnet-convol}. For
%simple stations, it is often simpler to use the
%equivalent service rate defined in the next
%section.
\begin{figure}[htbp]
\center
   \insfignc{netdecomp1}{0.4}
   \insfignc{netdecom2}{0.4}\\
   \insfignc{netdecom3}{0.4}
   \insfignc{netdecom4}{0.4}
  \mycaption{Decomposition procedure:
  original network $\calN$,
  with subnetwork $\calS$; simplified
  network $\tilde{\calN}$;
  equivalent station $\tilde{\calS}$; subnetwork
  in short-circuit
  $\tilde{\calN}_{\calS}$.}
  \mylabel{fig-q-st-aggreg}
\end{figure}


A key consequence of the product form theorem is
the possibility to replace an entire subnetwork
by an equivalent single station. This can be done
recursively and is the basis for many algorithms,
such as MVA version 2.

Consider a multi-class product form network
$\calN$ and a subnetwork $\calS$. The stations in
$\calS$ need not directly connected and the
network can be closed, mixed or open. If the
network is mixed or open, we consider that
outside arrivals are from some fictitious station
$0$, and $0\nin \calS$. We create two virtual
networks: $\tilde{\calN}$ and
$\tilde{\calN}_{\calS}$ and a virtual station
$\tilde{\calS}$ as follows
(\fref{fig-q-st-aggreg}).

The virtual station $\tilde{\calS}$, called the
\nt{equivalent station} of $\calS$, is obtained
by isolating the set of stations $\calS$ from the
network $\calN$ and collapsing classes to chains.
Inside $\tilde{\calS}$, there is only one class
per chain, i.e. a customer's attribute is its
chain $\calC$; furthermore, the station is a
``per class PS" station, with service rate to be
defined later%
\footnote{Observe that within one service station
customers cannot change class, therefore if we
aggregate a subnetwork into a single station, we
must aggregate classes of the same chain as
well.}.

$\tilde{\calN}$, called the \nt{simplified
network}, is obtained by replacing all stations
in $\calS$ by the equivalent station
$\tilde{\calS}$. In $\tilde{\calN}$, routing is
defined by the corresponding natural aggregation,
i.e. is the same as if the stations in $\calS$
were still present but not observable
individually. Thus the routing matrix $\tilde{q}$
is:
  \bearn
  \tilde{q}^{s,s'}_{c,c'}  & =  & q^{s,s'}_{c,c'}
 \mif s \nin \calS \mand s' \nin \calS
 \\
  \tilde{q}^{\calS,s'}_{\calC,c'}& = &
    \bracket{
    0 \mif c' \nin \calC
       \\
  \frac{1}{\tilde{\theta}_{\calC}}
      \sum_{s \in \calS, c \in \calC} \theta^s_c q^{s,s'}_{c,c'}
   \mif c' \in \calC}
  \\
  \tilde{q}^{s,\calS}_{c,\calC} & =&
  \bracket{
    0 \mif c \nin \calC
    \\
    \sum_{s' \in \calS, c' \in \calC} q^{s,s'}_{c,c'}  \mif c \in \calC
    }
 \\
  \tilde{q}^{\calS,\calS}_{\calC,\calC'}&  =&
  \bracket{0 \mif \calC \neq \calC'
  \\
   \frac{1}{\tilde{\theta}_{\calC}}
      \sum_{s, s' \in \calS, c,c' \in \calC}
      \theta^s_c q^{s,s'}_{c,c'} \mif \calC = \calC'
      }
   \\
  \tilde{\theta}_{\calC} & = &
      \sum_{s \in \calS, c \in \calC} \theta^s_c
   \eearn
where, for example,
$\tilde{q}^{\calS,s'}_{\calC,c'}$ is the
probability that a chain $\calC$ customer leaving
station $\tilde{\calS}$ joins station $s'$ with
class $c'$. If there are some open chains, recall
that $s=0$ represents arrivals and departures and
we assumed $0\nin\calS$; in such cases, the
external arrival rate of chain $\calC$ customers
to the virtual station $\tilde{\calS}$ is
  \ben
  \la^{\calS}_{\calC} = \sum_{s \in \calS, c \in \calC}
  \la^s_c
  \een
and the probability that a chain $\calC$
customers leaves the network after visiting
$\tilde{\calS}$ is
  \ben
  \frac{1}{\tilde{\theta}_{\calC}}
      \sum_{s \in \calS, c \in \calC}
      \theta^s_c q^{s,0}_{c}
    \een where  $q^{s,0}_c \eqdef 1- \sum_{s',
    c'}q^{s,s'}_{c,c'}$ is the probability that a
    class $c$ customer leaves the network after
    visiting station $s$.

The visit rates in $\tilde{\calN}$ are the same
as in $\calN$ for stations not in $\calS$; for
the equivalent station $\tilde{\calS}$, the visit
rate for chain $\calC$ is $
\tilde{\theta}_{\calC}$ given above.
 %\ben
% \bracket{
% \tilde{\theta}^s_c  =  \theta^s_c \mif s
% \nin \calS
% \\
% \tilde{\theta}^{\calS}_c= \sum_{s \in \calS}\theta^s_c
%   }
% \een
The station function of the equivalent station
$\tilde{\calS}$ is computed in such a way that
replacing all stations in $\calS$ by
$\tilde{\calS}$ makes no difference to the
stationary probability of the network. It
follows, after some algebra, from the product
form theorem; the precise formulation is a bit
heavy:
 \be
 f^{\calS}(\vk)=\sum_{(\vn^s)_{s \in \calS}
 \mst \sum_{s \in \calS, c \in \calC}n^s_c=k_{\calC}}\prod_{s \in
 \calS}
  \lb
  f^s(\vn^s)
   \prod_{c}
    \lp
      \frac{\theta^s_c}{\theta^{\calS}_c}
    \rp^{n^s_c}
  \rb
 \ee where $\vk$ is a population vector of closed
 or open
 chains. Note that it may happen that some chain
 $\calC_0$
be ``trapped" in $\calS$, i.e customers of this
chain never leave $\calS$. The generating
function of the virtual station $\calS$ has a
simple expression
  \be
 G^{\calS}(\vZ)  =  \prod_{s\in \calS}G^s(\vX^s)
 \;
 \mwith X^s_c  =
 Z_{\calC(c)}\frac{\theta^s_c}{\tilde{\theta}_{\calC}}
\label{eq-q-qnet-fgagllddd}
  \ee where $\calC(c)$ is the chain of class $c$.
  Here $\calC$ spans the set of all chains,
  closed or open. Thus, the equivalent station
$\tilde{\calS}$ is a per-class PS station, with
one class per chain, and with balance function
$f^{\calS}(\vk)$. In the next theorem, we will
give an equivalent statement that is easier to
use in practice.


The second virtual network, $\tilde{\calN}_{\calS}$ is called
the \nt{subnetwork in short-circuit}\index{short circuit}. It
consists in replacing anything not in $\calS$ by a
short-circuit. In $\tilde{\calN}_{\calS}$, the service times at
stations not in $\calS$ are $0$ and customers instantly
traverse the complement of $\calS$. This includes the virtual
station $0$ which represents the outside, so
$\tilde{\calN}_{\calS}$ is a closed network%
\footnote{Be careful that this is different from
the procedure used when defining the station in
isolation. In $\tilde{\calN}_{\calS}$, $\calS$ is
connected to a short-circuit, i.e. a station
where the service requirement is $0$; in
contrast, in the configuration called ``$\calS$
in isolation", $\calS$ is connected to a station
with unit rate and unit service requirement. }.
%The service stations in $\tilde{\calN}_{\calS}$
%are station $\tilde{\calS}'$ and the stations in
%$\calS$, and routing is defined by
%short-circuiting anything outside of $\calS$.
%More precisely, the routing matrix in
%$\tilde{\calN}_{\calS}$ is
% \bearn
%  \tilde{\tilde{q}}^{s,s'}_{c,c'}  & =  & q^{s,s'}_{c,c'}
% \mif s \in \calS \mand s' \in \calS
% \\
%  \tilde{\tilde{q}}^{\tilde{\calS}',s'}_{\calC,c'}& = &
%    \bracket{
%    0 \mif c' \nin \calC
%       \\
%  \frac{1}{\tilde{\tilde{\theta}}_{\calC}}
%      \sum_{s \nin \calS, c \in \calC} \theta^s_c q^{s,s'}_{c,c'}
%   \mif c \in \calC}
%  \\
%  \tilde{\tilde{q}}^{s,\tilde{\calS}'}_{c,\calC} & =&
%  \bracket{
%    0 \mif c \nin \calC
%    \\
%    \sum_{s' \nin \calS, c' \in \calC} q^{s,s'}_{c,c'}
%    }
% \\
%  \tilde{\tilde{q}}^{\tilde{\calS}',\tilde{\calS}'}_{\calC,\calC'}&  =&
%  \bracket{0 \mif \calC \neq \calC'
%  \\
%   \frac{1}{\tilde{\tilde{\theta}}_{\calC}}
%      \sum_{s, s' \nin \calS, c,c' \in \calC}
%      \theta^s_c q^{s,s'}_{c,c'}
%      }
%   \\
%  \tilde{\tilde{\theta}}_{\calC} & = &
%      \sum_{s \nin \calS, c \in \calC} \theta^s_c
%   \eearn
%Note that $\tilde{\calN}_{\calS}$ is always
%closed, even if there are some open chains in the
%original network.
The population vector $\vk$ remains constant in
$\tilde{\calN}_{\calS}$; the visit rates at stations in
$\calS$ are the same as in the original network for closed chains. For classes that belong to a chain that is open in the original network, we obtain the visit rates by setting arrival rates to $1$.

%It can be seen by direct inspection that the
%normalizing constant for the subnetwork in
%short-circuit is the same as the station function
%$f^{\calS}(\vk)$. Thus we have shown most of the
%following theorem:
 \begin{shadethm}(Decomposition
Theorem \cite{MR563738})

Consider a multi-class network that satisfies the
hypotheses of the product form
theorem~\ref{theo-q-pf}. Any subnetwork $\calS$
can be replaced by its equivalent station
 $\tilde{\calS}$, with one class per chain and
station function defined by
\eref{eq-q-qnet-fgagllddd}. In the resulting
equivalent network $\tilde{\calN}$, the
stationary probability and the throughputs that
are observable are the same as in the original
network.

Furthermore, if $\calC$ effectively visits
$\calS$, the equivalent service rate to chain
$\calC$ (closed or open) at the equivalent
station $\tilde{\calS}$ is
 \be
  \mu^{*\calS}_{\calC}(\vk) = \la^{*\calS}_{\calC}(\vk)
  \label{eq-q-qnet-sdlfkjfdlkii}
  \ee
 where
$\la^{*\calS}_{\calC}(\vk)$ is the throughput of
chain $\calC$ for the subnetwork in short-circuit
$\tilde{\calN}_{\calS}$ when the population
vector for all chains (closed or open) is $\vk$.
 \label{theo-q-qnet-aggreg}
\end{shadethm}
The phrase ``that are observable" means: the
numbers of customers of any class at any station
not in $\calS$; the total number of customers of
chain $\calC$ that are present in any station of
$\calS$; the throughputs of all classes at all
stations not in $\calS$; the throughputs of all
chains. Recall that the per chain throughput
$\la_{\calC}(\vK)$ (defined in
\eref{eq-q-qnet-dskjsdjkjkdf}) is the throughput
measured at some station $s_{\calC}$ effectively
visited by chain $\calC$. The station $s_{\calC}$
is assumed to be the same in the original and the
virtual networks, which is possible since the
visit rates are the same.

If $\calC$ does not effectively visit $\calS$
(i.e. if $\tilde{\theta}_{\calC} \eqdef
      \sum_{s \in \calS, c \in \calC}
 \theta^s_c=0$) then the equivalent service
rate  $\mu^{*\calS}_{\calC}$ is undefined, which
is not a problem since we do not need it.

By the throughput theorem,
\eref{eq-q-qnet-sdlfkjfdlkii} can also be written
$\mu^{*\calS}_{\calC}(\vk) =
\frac{\eta^*(\vk-\vone_{\calC})}{\eta^*(\vk)}$
where $\eta^*(\vk)$ is the normalizing constant
for the subnetwork in short-circuit
 $\tilde{\calN}_{\calS}$.

 If
 $\calS$ consists of a single station with one
 class per chain at this station, then the
 equivalent station is the same as the original station,
 as expected. Also, the theorem implies, as a byproduct,
 that the equivalent service rate for class $c$ at a
  station $s$, as defined in \eref{eq-q-qnet-eqsrvrat},
is equal to the throughput for class $c$ at the
network made of this station and a short circuit
for every class (i.e. every class $c$ customer
immediately returns to the station upon service
completion, with the same class).
 \begin{figure}
\centering
 \Ifignc{dcpAgg}{0.95}{0.5}~\\
 \vspace{1cm}
 \Ifignc{dcpAggSC}{0.95}{0.3}
 %\Ifignc{pdf-x=0.7y=0.8}{0.95}{0.4}
  \mycaption{Aggregation of stations applied to the dual core
  processor example of \fref{fig-q-chains}. First panel: stations
   2 and 3 are replaced by $\tilde{\calS}$. Bottom panel: the
   network in short-circuit $\tilde{\calN}_{\calS}$
   used to compute the equivalent service rate $\mu^*(n4)$ of $\tilde{\calS}$.}
  \label{fig-q-qnet-sdflfdlkjbnfiouzgb}
\end{figure}

\begin{ex}{Dual Core Processor in \fref{fig-q-chains},
continued} We replace stations 2 and 3 by one
aggregated station $\tilde{S}$ as in
\fref{fig-q-qnet-sdflfdlkjbnfiouzgb}. This
station receives only customers of class 4
(internal jobs). Its equivalent service rate is
 \be
 \mu^*(n_4) = \frac{\eta^*(n_4-1)}{\eta^*(n_4)}
 \ee
where $\eta^*(n_4)$ is the normalizing constant
for the network $\tilde{\calN}_{\calS}$ obtained
when replacing station 1 by a short-circuit as in
\fref{fig-q-qnet-sdflfdlkjbnfiouzgb}; the $z$
transform of $\eta^*$ is given by the convolution
theorem \ref{theo-gf-eta}:
 \be
 F_{\eta^*}(Y)=e^{\bar{S}^2Y}
      \frac{1}{1-\bar{S}^3 Y}
 \label{eq-q-qnet-sdjsduuepoowjfdgkjhhjdkjhfd}
 \ee
 One can compute a Taylor expansion and deduce $\eta^*(n)$ or use
 \pro{filter} as in the other examples, but here
 one can also find a closed form
  \be
  \eta^*(n) = \lp\bar{S}^3\rp^n \sum_{k=0}^n
  \lp\frac{\bar{S}^2}{\bar{S}^3}\rp^k \frac{1}{k!}
  \ee
 Note that for large $n$,  $\eta^*(n) \approx
 \lp\bar{S}^3\rp^n \exp\lp
 \frac{\bar{S}^2}{\bar{S}^3}\rp$ and thus
$\mu^*(n) \approx \frac{1}{\bar{S}^3}$, i.e. it
is equivalent to station 3 (but this is true only
for large $n$). We can deduce the equivalent
service rate $\mu^*(n)$ and obtain the
probability distribution $P^*(n)$ of internal
jobs at stations $2$ or $3$ using
\thref{theo-q-qnet-adsjkhfuierbhfv} as in
\exref{ex-q-qnet-dsjkldsjkfl}.

Note that internal jobs are either at station 1,
or at stations 2 or 3. Thus we should have
  \be
  P^*(n|K) = P'^1(K-n|K)
  \ee
where $P'^1(.|K)$ is the probability distribution
for internal jobs at station 1, already obtained
in \exref{ex-q-qnet-dsjkhuzewnnbasdio}, and we
can verify this numerically.
  \end{ex}
%
%decomposition procedures
%that aim at simplifying the network
%analysis. %There are different aggregation
%%methods, which differ in the definition of
%%auxiliary stations. A first approach is the same
%%as was used already in the definition of the
%%station in isolation.
%First we define the aggregation procedure.
%
%Consider a multi-class product form network
%$\calN$ and a subset $\calS$ of stations. The
%stations need not directly connected and the
%network can be closed, mixed or open. If the
%network is mixed or open, we consider that
%outside arrivals are from some fictitious station
%$0$, and $0\nin \calS$. We create two virtual
%networks: $\tilde{\calN}$ and
%$\tilde{\calN}_{\calS}$ and a virtual station
%$\tilde{\calS}$ as follows
%(\fref{fig-q-st-aggreg}).
%
%The virtual station $\tilde{\calS}$, called the
%\nt{equivalent station} of $\calS$, is obtained
%by isolating the set of stations $\calS$ from the
%network $\calN$ and collapsing classes to chains.
%Inside $\tilde{\calS}$, there is only one class
%per chain, i.e. a customer's attribute is its
%chain $\calC$; furthermore, the station is a
%``per class PS" station, with service rate to be
%defined later%
%\footnote{Observe that within one service station
%customers cannot change class, therefore if we
%aggregate a subnetwork into a single station, we
%must aggregate classes of the same chain as
%well.}.
%
%$\tilde{\calN}$, called the \nt{aggregated
%network}, is obtained by replacing all stations
%in $\calS$ by the equivalent station
%$\tilde{\calS}$. In $\tilde{\calN}$, routing is
%defined by the corresponding natural aggregation,
%i.e. is the same as if the stations in $\calS$
%were still present but not observable
%individually. More precisely, the routing matrix
%$\tilde{q}$ is:
%  \bearn
%  \tilde{q}^{s,s'}_{c,c'}  & =  & q^{s,s'}_{c,c'}
% \mif s \nin \calS \mand s' \nin \calS
% \\
%  \tilde{q}^{\calS,s'}_{\calC,c'}& = &
%    \bracket{
%    0 \mif c' \nin \calC
%       \\
%  \frac{1}{\tilde{\theta}_{\calC}}
%      \sum_{s \in \calS, c \in \calC} \theta^s_c q^{s,s'}_{c,c'}
%   \mif c \in \calC}
%  \\
%  \tilde{q}^{s,\calS}_{c,\calC} & =&
%  \bracket{
%    0 \mif c \nin \calC
%    \\
%    \sum_{s' \in \calS, c' \in \calC} q^{s,s'}_{c,c'}
%    }
% \\
%  \tilde{q}^{\calS,\calS}_{\calC,\calC'}&  =&
%  \bracket{0 \mif \calC \neq \calC'
%  \\
%   \frac{1}{\tilde{\theta}_{\calC}}
%      \sum_{s, s' \in \calS, c,c' \in \calC}
%      \theta^s_c q^{s,s'}_{c,c'}
%      }
%   \\
%  \tilde{\theta}_{\calC} & = &
%      \sum_{s \in \calS, c \in \calC} \theta^s_c
%   \eearn
%where, for example,
%$\tilde{q}^{\calS,s'}_{\calC,c'}$ is the
%probability that a chain $\calC$ customer leaving
%station $\tilde{\calS}$ joins station $s'$ with
%class $c'$. If there are some open chains, recall
%that $s=0$ represents arrivals and departures and
%we assumed $0\nin\calS$; in such cases, the
%external arrival rate of chain $\calC$ customers
%to the virtual station $\tilde{\calS}$ is
%  \ben
%  \la^{\calS}_{\calC} = \sum_{s \in \calS, c \in \calC}
%  \la^s_c
%  \een
%and the probability that a chain $\calC$
%customers leaves the network after visiting
%$\tilde{\calS}$ is
%  \ben
%  \frac{1}{\tilde{\theta}_{\calC}}
%      \sum_{s \in \calS, c \in \calC}
%      \theta^s_c q^{s,0}_{c}
%    \een where  $q^{s,0}_c \eqdef 1- \sum_{s',
%    c'}q^{s,s'}_{c,c'}$ is the probability that a
%    class $c$ customer leaves the network after
%    visiting station $s$.
%
%The visit rates in $\tilde{\calN}$ are the same
%as in $\calN$ for stations not in $\calS$; for
%the equivalent station $\tilde{\calS}$, the visit
%rate for chain $\calC$ is $
%\tilde{\theta}_{\calC}$ given above.
% %\ben
%% \bracket{
%% \tilde{\theta}^s_c  =  \theta^s_c \mif s
%% \nin \calS
%% \\
%% \tilde{\theta}^{\calS}_c= \sum_{s \in \calS}\theta^s_c
%%   }
%% \een
%The station function of the equivalent station
%$\tilde{\calS}$ is computed in such a way that
%replacing all stations in $\calS$ by
%$\tilde{\calS}$ makes no difference to the
%stationary probability of the network. It
%follows, after some algebra, from the product
%form theorem; the precise formulation is a bit
%heavy:
% \be
% f^{\calS}(\vk)=\sum_{(\vn^s)_{s \in \calS}
% \mst \sum_{s \in \calS, c \in \calC}n^s_c=k_{\calC}}\prod_{s \in
% \calS}
%  \lb
%  f^s(\vn^s)
%   \prod_{c}
%    \lp
%      \frac{\theta^s_c}{\theta^{\calS}_c}
%    \rp^{n^s_c}
%  \rb
% \ee where $\vk$ is a population vector of closed
% or open
% chains.
%The generating function of the virtual station
%$\calS$ has a simple expression
%  \be
% G^{\calS}(\vZ)  =  \prod_{s\in \calS}G^s(\vX^s)
% \;
% \mwith X^s_c  =
% Z_{\calC(c)}\frac{\theta^s_c}{\tilde{\theta}_{\calC}}
%\label{eq-q-qnet-fgagllddd}
%  \ee where $\calC(c)$ is the chain of class $c$.
%  Here $\calC$ spans the set of all chains,
%  closed or open. Thus, the equivalent station
%$\tilde{\calS}$ is a per-class PS station, with
%one class per chain, and with balance function
%$f^{\calS}(\vk)$. In the next theorem, we will
%give an equivalent statement that is easier to
%use in practice.
%
%
%The second virtual network,
%$\tilde{\calN}_{\calS}$ is called the
%\nt{subnetwork in short-circuit}. It results from
%the dual aggregation, where all stations not in
%$\calS$ are collapsed into one single station
%$\tilde{\calS}'$. In $\tilde{\calN}_{\calS}$, the
%service times at $\tilde{\calS}'$ are $0$ (so
%customers instantly traverse  $\tilde{\calS}'$;
%also, in $\tilde{\calS}'$, customers loose their
%class attribute, but retain their chain
%attribute)%
%\footnote{Be careful that this is different from
%the procedure used when defining the station in
%isolation. In $\tilde{\calN}_{\calS}$, $\calS$ is
%connected to a short-circuit, i.e. a station
%where the service requirement is $0$; in
%contrast, in the configuration called ``$\calS$
%in isolation", $\calS$ is connected to a station
%with unit rate and unit service requirement. }.
%The service stations in $\tilde{\calN}_{\calS}$
%are station $\tilde{\calS}'$ and the stations in
%$\calS$, and routing is defined by
%short-circuiting anything outside of $\calS$.
%More precisely, the routing matrix in
%$\tilde{\calN}_{\calS}$ is
% \bearn
%  \tilde{\tilde{q}}^{s,s'}_{c,c'}  & =  & q^{s,s'}_{c,c'}
% \mif s \in \calS \mand s' \in \calS
% \\
%  \tilde{\tilde{q}}^{\tilde{\calS}',s'}_{\calC,c'}& = &
%    \bracket{
%    0 \mif c' \nin \calC
%       \\
%  \frac{1}{\tilde{\tilde{\theta}}_{\calC}}
%      \sum_{s \nin \calS, c \in \calC} \theta^s_c q^{s,s'}_{c,c'}
%   \mif c \in \calC}
%  \\
%  \tilde{\tilde{q}}^{s,\tilde{\calS}'}_{c,\calC} & =&
%  \bracket{
%    0 \mif c \nin \calC
%    \\
%    \sum_{s' \nin \calS, c' \in \calC} q^{s,s'}_{c,c'}
%    }
% \\
%  \tilde{\tilde{q}}^{\tilde{\calS}',\tilde{\calS}'}_{\calC,\calC'}&  =&
%  \bracket{0 \mif \calC \neq \calC'
%  \\
%   \frac{1}{\tilde{\tilde{\theta}}_{\calC}}
%      \sum_{s, s' \nin \calS, c,c' \in \calC}
%      \theta^s_c q^{s,s'}_{c,c'}
%      }
%   \\
%  \tilde{\tilde{\theta}}_{\calC} & = &
%      \sum_{s \nin \calS, c \in \calC} \theta^s_c
%   \eearn
%Note that $\tilde{\calN}_{\calS}$ is always
%closed, even if there are some open chains in the
%original network. The population vector $\vk$
%remains constant in $\tilde{\calN}_{\calS}$.
%
%It can be seen by direct inspection that the
%normalizing constant for the subnetwork in
%short-circuit is the same as the station function
%$f^{\calS}(\vk)$. Thus we have shown most of the
%following theorem:
% \begin{shadethm}(Decomposition
%Theorem \cite{MR563738})
%
%Consider a multi-class network that satisfies the
%hypotheses of the product form
%theorem~\ref{theo-q-pf}. Any subset of stations
%$\calS$ can be replaced by its equivalent station
% $\tilde{\calS}$, with one class per chain and
%station function defined by
%\eref{eq-q-qnet-fgagllddd}. In the resulting
%equivalent network $\tilde{\calN}$, the
%stationary probability and the throughputs that
%are observable are the same as in the original
%network.
%
%Furthermore, the equivalent service rate to chain
%$\calC$ (closed or open) at the equivalent
%station $\tilde{\calS}$ is
% \be
%  \mu^{*\calS}_{\calC}(\vk) = \la^{*\calS}_{\calC}(\vk)
%  \label{eq-q-qnet-sdlfkjfdlkii}
%  \ee
%where $\la^{*\calS}_{\calC}(\vk)$ is the
%throughput of chain $\calC$ for the subnetwork in
%short-circuit $\tilde{\calN}_{\calS}$ when the
%population vector for all chains (closed or open)
%is $\vk$.
% \label{theo-q-qnet-aggreg}
%\end{shadethm}
%The phrase ``that are observable" means: the
%numbers of customers of any class at any station
%not in $\calS$; the total number of customers of
%chain $\calC$ that are present in any station of
%$\calS$; the throughputs of all classes at all
%stations not in $\calS$; the sum of all
%throughputs per class over classes in one chain
%and over all stations in $\calS$. The throughput
%for chain $\calC$ is measured on the outside loop
%of $\tilde{\calN}_{\calS}$ (see
%\fref{fig-q-st-aggreg}).
%
%By the throughput theorem,
%\eref{eq-q-qnet-sdlfkjfdlkii} can also be written
%$\mu^{*\calS}_{\calC}(\vk) =
%\frac{\eta^*(\vk-\vone_{\calC})}{\eta^*(\vk)}$
%where $\eta^*(\vk)$ is the normalizing constant
%for the subnetwork in short-circuit
% $\tilde{\calN}_{\calS}$.
%
% \begin{figure}
%\centering
% \Ifignc{dcpAgg}{0.95}{0.5}~\\
% \vspace{1cm}
% \Ifignc{dcpAggSC}{0.95}{0.3}
% %\Ifignc{pdf-x=0.7y=0.8}{0.95}{0.4}
%  \mycaption{Aggregation of stations applied to the dual core
%  processor example of \fref{fig-q-chains}. First panel: stations
%   2 and 3 are replaced by $\tilde{\calS}$. Bottom panel: the
%   network in short-circuit $\tilde{\calN}_{\calS}$
%   used to compute the equivalent service rate $\mu^*(n4)$ of $\tilde{\calS}$.}
%  \label{fig-q-qnet-sdflfdlkjbnfiouzgb}
%\end{figure}
%
%\begin{ex}{Dual Core Processor in \fref{fig-q-chains},
%continued} We replace stations 2 and 3 by one
%aggregated station $\tilde{S}$ as in
%\fref{fig-q-qnet-sdflfdlkjbnfiouzgb}. This
%station receives only customers of class 4
%(internal jobs). Its equivalent service rate is
% \be
% \mu^*(n_4) = \frac{\eta^*(n_4-1)}{\eta^*(n_4)}
% \ee
%where $\eta^*(n_4)$ is the normalizing constant
%for the network $\tilde{\calN}_{\calS}$ obtained
%when replacing station 1 by a short-circuit as in
%\fref{fig-q-qnet-sdflfdlkjbnfiouzgb}; the $z$
%transform of $\eta^*$ is given by the convolution
%theorem \ref{theo-gf-eta}:
% \be
% F_{\eta^*}(Y)=e^{\bar{S}^2Y}
%      \frac{1}{1-\bar{S}^3 Y}
% \label{eq-q-qnet-sdjsduuepoowjfdgkjhhjdkjhfd}
% \ee
% One can compute a Taylor expansion and deduce $\eta^*(n)$ or use
% \pro{filter} as in the other examples, but here
% one can also find a closed form
%  \be
%  \eta^*(n) = \lp\bar{S}^3\rp^n \sum_{k=0}^n
%  \lp\frac{\bar{S}^2}{\bar{S}^3}\rp^k \frac{1}{k!}
%  \ee
% Note that for large $n$,  $\eta^*(n) \approx
% \lp\bar{S}^3\rp^n \exp\lp
% \frac{\bar{S}^2}{\bar{S}^3}\rp$ and thus
%$\mu^*(n) \approx \frac{1}{\bar{S}^3}$, i.e. it
%is equivalent to station 3 (but this is true only
%for large $n$). We can deduce the equivalent
%service rate $\mu^*(n)$ and obtain the
%probability distribution $P^*(n)$ of internal
%jobs at stations $2$ or $3$ using
%\thref{theo-q-qnet-adsjkhfuierbhfv} as in
%\exref{ex-q-qnet-dsjkldsjkfl}.
%
%Note that internal jobs are either at station 1,
%or at stations 2 or 3. Thus we should have
%  \be
%  P^*(n|K) = P'^1(K-n|K)
%  \ee
%where $P'^1(.|K)$ is the probability distribution
%for internal jobs at station 1, already obtained
%in \exref{ex-q-qnet-dsjkhuzewnnbasdio}, and we
%can verify this numerically.
%  \end{ex}
%
%
%
\subsection{MVA Version 2}
\label{sec-mva-2} This is an algorithm which,
like MVA version 1, avoids computing the
normalizing constant, but which applies to fairly
general station types \cite{reiser1981mean}. We
give a version for single chain (but multi-class)
networks. For networks with several chains, the
complexity of this method is exponential in the
number of chains, and more elaborate
optimizations have been proposed; see
\cite{conway1986recal,conway1989mean} as well as
\cite{balsamo2000product} and the discussion
therein.

The starting point is the decomposition theorem,
which says that one can replace a subnetwork by a
single station if one can compute its throughputs
in short circuit. For example, using MVA version
1, one can compute the throughputs of a
subnetwork made of single server FIFO or IS
stations (or equivalent), therefore one can replace
the set of all such stations in a network by one
single station.

MVA version 2 does the same thing for general
stations in closed networks. This can be reduced
to the simpler problem of how to compute the
throughput of a network of 2 stations, with
numerically known service rates. If we can solve
this problem, we can replace the 2 stations by a
new one, the service rate is equal to the
throughput (by \thref{theo-q-qnet-aggreg}), and
we can iterate. This problem is solved by the
next theorem. It uses the concept of networks in
short-circuit.

\begin{shadethm}(\nt{Complement Network Theorem})
Consider a \emph{closed} multi-class product form
queuing network $\calN$. Let $\calS^1, \calS^2$
be a partition of $\calN$ in two subnetworks and
let $\calN_{\calS^1},\calN_{\calS^2}$ be the
corresponding subnetworks in short circuit (in
$\calN_{\calS^1}$, all stations in $\calS^2$ are
short circuited). Define:
\begin{itemize}
    \item $P^1(\vk| \vK)=$ the stationary probability
    that the number of customers of chain
    $\calC$ present in $\calS^1$ is $k_{\calC}$ for all
    $\calC$ when the total network
    population vector is $\vK$;
    \item $\eta^1(\vK)=$ [resp. $\eta^2(\vK)$, $\eta(\vK)$]
the normalizing constant of
    $\calN_{\calS^1}$ [resp. $\calN_{\calS^2}$, $\calN$] when the total network
    population vector is $\vK$;
    \item $\la^{*1}_\calC(\vK)=$
[resp. $\la^{*2}_\calC(\vK)$, $\la_\calC(\vK)$]
the per chain
    throughput of chain $\calC$ in
    $\calN_{\calS^1}$ [resp. $\calN_{\calS^2}$, $\calN$] when the total network
    population vector is $\vK$.
\end{itemize}
Then for $\vzero\leq \vk\leq \vK$:
  \bear
  P^1(\vk| \vK) & = &  \frac{\eta^1(\vk)
  \eta^2(\vK-\vk)}{\eta(\vK)}
  \label{eq-q-qnet-dec-dfsakjhfu1}
  \eear
  and for any chain $\calC$ such that
  $k_{\calC}>0$:
  \bear
  P^1(\vk|\vK) & = & P^1(\vk - \vone_{\calC}|(\vK - \vone_{\calC})
  \frac{\la_{\calC}(\vK)}{\la^{*1}_{\calC}(\vk)}
 \label{eq-q-qnet-dec-dfsakjhfu2}
  \\
  P^1(\vk|\vK) & = &  P^1(\vk |(\vK - \vone_{\calC})
  \frac{\la_{\calC}(\vK)}{\la^{*2}_{\calC}(\vK-\vk)}
 \label{eq-q-qnet-dec-dfsakjhfu3}
  \eear
 \label{theo-q-qnet-compnet}
\end{shadethm}
The inequalities $\vzero\leq \vk\leq \vK$ are
componentwise. The proof is by direct inspection:
recognize in \eref{eq-q-qnet-dec-dfsakjhfu1} the
convolution theorem;
\eref{eq-q-qnet-dec-dfsakjhfu2} and
\eref{eq-q-qnet-dec-dfsakjhfu3} follow from
\eref{eq-q-qnet-dec-dfsakjhfu1} and the
throughput theorem.

Note that \eref{eq-q-qnet-dec-dfsakjhfu2} is an
instance of the equivalent service rate formula
\eref{eq-q-mva-bas}, since
$\la^{*1}_{\calC}(\vk)=\mu^{*1}_{\calC}(\vk)$ is
also equal to the equivalent service rate of
$\calS^1$. \eref{eq-q-qnet-dec-dfsakjhfu3} is the
symmetric of \eref{eq-q-qnet-dec-dfsakjhfu2} when
we exchange the roles of $\calS^1$ and $\calS^2$
since $P^1(\vk|\vK)=P^2(\vK-\vk|\vK)$.

$\calS^2$ is called the complement network of
$\calS^1$ in the original work
\cite{reiser1981mean}, hence the name.

\paragraph{The MVA Composition Step}
In the rest of this section we consider that
there is only one chain, and drop index $\calC$.
Assume that we know the throughputs of the two
subnetworks $\la^{*1}(K), \la^{*2}(K)$; the goal
of the composition step is to compute $\la(K)$.
We compute the distribution $P^1(.|K)$ by
iteration on $K$, starting with $P^1(0|0)=1$,
$P^1(n|0)=0$, $n\geq 1$.
\eref{eq-q-qnet-dec-dfsakjhfu2} and
\eref{eq-q-qnet-dec-dfsakjhfu3} become
  \bear
  \mfor k=1...K\;:\;P^1(k|K) & = & P^1(k - 1|(K - 1)\frac{\la(K)}{\la^{*1}(k)}
   \\
\mfor k=0...K-1\;:\;  P^1(k|K) & = & P^1(k |(K -
1)\frac{\la(K)}{\la^{*2}(K-k)}
  \eear
None of the two equations alone is sufficient to
advance one iteration step, but the combination
of the two is. For example, use the former for
$k=1...K$ and the latter for $k=0$. $\la(K)$ is
then obtained by the condition
$\sum_{n=0}^KP^1(k|K)=1$.

\paragraph{MVA Version 2}
The algorithm works in two phases. In phase 1,
the throughput is computed. The starting point is
a network $\calN_0$; first, we compute the
throughput of the subnetwork $\calS^0$ made of
all stations to which MVA version 1 applies, as
this faster than MVA version 1. We replace
$\calS^0$ by its equivalent station; let
$\calN_1$ be the resulting network.

In one step we match stations 2 by 2, possibly
leaving one station alone. For every pair of
matched stations we apply the MVA Composition
Step to the network made of both stations in
short circuit (all stations except the two of the
pair are short-circuited); we thus obtain the
throughput of the pair in short-circuit. Then we
replace the pair by a single station, whose
service rate is the throughput just computed.
This is repeated until there is only one
aggregate station left, at which time the phase 1
terminates and we have computed the throughput
$\la(K)$ of the original network.

In phase 2, the distributions of states at all
stations of interest can be computed using the
equivalent service rate theorem
(\eref{eq-q-mva-bas}) and normalization to obtain
the probability of an empty station; there is no
need to use the complement network in this phase.

The number of steps in Phase 1 is order of
$\log_2(N)$, where $N$ is the number of stations;
the MVA Composition Step is applied in total
order of $N$ times (and not $2^N$ as wrongly
assumed in \cite{balsamo2000product}). The
complexity of one MVA Composition Step is linear
in $K$, the population size.

In \aref{algo-q-mva-cs} in
\sref{sec-q-qnet-qnaskxidzesdfsa} we give a
concrete implementation.

%This is a method, developed by
%Reiser \cite{reiser1981mean}, which avoids
%computing the normalizing constant and the
%corresponding overflow problems. There are many
%variants of it, such as the Multibus algorithm
%\cite{le1988multibus} for networks with MSCCC
%stations other than FIFO; see also the discussion
%in \cite{balsamo2000product}.
%
%In this section, we give two variants. The former
%is very simple, but applies only to some station
%types, as it requires to be able to derive the
%response time from the Palm distribution of queue
%size upon customer arrival. The second is more
%general and applies to all stations for which the
%service rate can be computed easily.
%
%networks where all
%stations are FIFO with one server or IS with
%constant rate, or have the same station function
%as one of these two stations. Indeed, recall that
%stationary probabilities and throughput depend
%only on the station function. It is based on the
%arrival theorem. The second variant is more
%general, and makes use of both the arrival
%theorem and the complement network theorem,
%presented next.

%There are many variants of it, and they are not
%all numerically stable. We give here the simplest
%form, which is  For other cases,  For an
%application to MSCCC stations other than the
%single server FIFO station, see the Multibus
%algorithm \cite{le1988multibus}. The method is
%based on the following theorem for Palm
%distributions at arrival times.
%

%
%\subsubsection{Simple MVA for Networks With Single Server FIFO, IS or similar Stations}
%This algorithm is based on the following theorem,
%which is a consequence of the product form
%theorem and the embedded subchain theorem of Palm
%calculus (\thref{theo-incluse}).
%
%\begin{shadethm}(\nt{Arrival Theorem})
%
%Consider a multi-class product form queuing,
%possibly with blocking. The probability
%distribution of the numbers of customers seen by
%customer just before arriving at station $s$ is
%the stationary distribution of
%\begin{itemize}
%    \item the same network
%if the customer belongs to an open chain;
%    \item the network with one customer less in
%    its chain, if the customer belongs to a closed
%    chain.
%\end{itemize}
%\end{shadethm}
%
%Consider now a \emph{closed} network where all
%stations are FIFO or IS with constant rate, or
%are similar in the sense that they have the same
%station function as one of these. Indeed, recall
%that stationary probabilities and throughput
%depend only on the station function. For example,
%a station may also be a global PS station with
%class independent service requirement of any
%phase type distribution, which has the same
%station function as a FIFO station with one
%server (and exponential service time). In the
%rest of this section we call ``FIFO" [resp. IS]
%station one that has the same station function as
%a single server, constant rate FIFO [resp. IS]
%station. Recall that at a FIFO station we need to
%assume that the mean service requirements are the
%same for all classes at the same station; for the
%IS station, it may be class dependent.
%
%First we assume the FIFO [resp. IS] stations are
%truly FIFO [resp.IS], not just similar stations
%as defined above. We will remove this restriction
%later. Let $\bar{N}^s_c(\vK)$ be the mean number
%of class $c$ customers at station $s$ when the
%chain population vector is $\vK$. The mean
%response time for a class $c$ customer at a FIFO
%station $s$ when the population vector is $\vK$
%is
% \ben
% \bar{R}^s_c(\vK) = \lp 1+
% \sum_{c}\bar{N}^s_c(\vK-\vone_{\calC})
% \rp
% \bar{S}^s
% \een where $\calC$ is the chain of class $c$.
%This is because of the exponential service
%requirement assumption: an
% arriving customer has to wait for $\bar{S}^s$ multiplied by the
% number of customers present upon arrival; in average, this latter
% number is
% $\sum_{c}\bar{N}^s_c(\vK-\vone_{\calC})$
%by the arrival theorem. By Little's formula:
% \ben
%\bar{R}^s_c(\vK) \la^s_c(\vK)=\bar{N}^s_c(\vK)
% \een
%Combining the two gives
%  \be
% \bar{N}^s_c(\vK) = \la^s_c(\vK) \lp 1+
% \sum_{c}\bar{N}^s_c(\vK-\vone_{\calC})
% \rp \bar{S}^s
%  \ee
%which is valid for FIFO stations. For a delay
%station one finds
% \be
% \bar{N}^s_c(\vK) = \la^s_c(\vK)\bar{S}^s_c
% \ee
%This gives a recursion for $\bar{N}^s_c(\vK)$ if
%one can get determine $\la^s_c(\vK)$. The next
%observation is \eref{eq-q-qnet-dskjsdjkjkdf},
%which says that if we know the throughput at one
%station visited by a chain class, then we know
%the throughputs for all stations and all classes
%of the same chain. The last observation is that
%the sum of the numbers of customers across all
%stations and all classes of chain $\calC$ is
%equal to $K_{\calC}$. Combining all this gives:
%for every chain $\calC$, if $K_{\calC}>0$ then
%  \bear
%  \frac{K_{\calC}}{\la_{\calC}( \vK)} &=&
%\sum_{c \in \calC}
%  \lb
%   \sum_{s: \mathrm{FIFO}}\theta^s_c
%   \lp 1+
% \sum_{c'}\bar{N}^s_{c'}(\vK-\vone_{\calC})
% \rp \bar{S}^s
%  +
%    \sum_{s: \mathrm{IS}}\theta^s_c
%    \bar{S}^s_c
%    \rb
%    \label{eq-q-qnet-mva-dlsld}
%  \eear
% and \be \la_{\calC}( \vK)= 0 \mif
% K_{\calC}=0 \ee
%For every FIFO station $s$ and class $c$:
% \bear
% \bar{N}^s_c(\vK) &= &\theta^s_c\la_{\calC(c)}
% (\vK) \lp 1+
% \sum_{c'}\bar{N}^s_{c'}(\vK-\vone_{\calC(c)})
% \rp \bar{S}^s
% \mif K_{\calC(c)} >0  \label{eq-q-qnet-mva-dlsdsdld}
% \\
% & = & 0 \mif  K_{\calC(c)} =0
% \eear
%%and for every IS station $s$ and class $c$
%% \bear
%% \bar{N}^s_c(\vK)& = &\theta^s_c\la_{\calC(c)}(\vK) \bar{S}^s_c
%% \\
%% & = & 0 \mif  K_{\calC(c)} =0
%% \eear
%
%Second, we observe that the resulting equations
%depend only on the station function, therefore
%they apply to similar stations as well.
%
%
%The \nt{Simple MVA} algorithm iterates on the
%total population, adding customers one by one. At
%every step, the throughput is computed using
%Equation~(\ref{eq-q-qnet-mva-dlsld}). Then the
%mean queue sizes at FIFO queues are computed
%using Equation~(\ref{eq-q-qnet-mva-dlsdsdld}),
%which closes the loop. We give the algorithm in
%the case of a single chain. For the multi-chain
%case, the algorithm is similar, but there are
%many optimizations to reduce the storage
%requirement, see \cite{balsamo2000product}.
%%
% \begin{algorithm}\mycaption{Mean Value Analysis for a single chain
% closed
% multi-class
% product form queuing network containing only constant rate
% FIFO and IS stations, or stations with same station functions.}
% \begin{algorithmic}[1]
%%
%  \State $K=$ population size
%  \State $\la=0$ \Comment{throughput}
%  %\State $N^s_c=0$ for all class $c$ and station
%%  $s$ \Comment{mean numbers of customers}
%  \State $Q^s=0$ for all station $s \in \mathrm{FIFO}$
%  \Comment{total number of customers at station $s$, $Q^s=\sum_c \bar{N}^s_c$}
%%
%  \State Compute the visit rates $\theta^s_c$
%  using \eref{eq-q-visitRatios} and
%  $\sum_{c=1}^C\theta^1_c=1$
%  \State $\theta^s=\sum_{c}\theta^s_c$ for every
%  $s\in \mathrm{FIFO}$
%  \State $h=\sum_{s \in \mathrm{IS}}\sum_{c}\theta^s_c
%  \bar{S}^s_c  + \sum_{s \in
%  \mathrm{FIFO}}\theta^s  \bar{S}^s$
%  \Comment{constant term in \eref{eq-q-qnet-mva-dlsld}}
%%
%    \For{$k=1:K$}
%    \State
%    $\la=\frac{k}{h+\sum_{s \in \mathrm{FIFO}}
%    \theta^s  Q^s \bar{S}^s}$
%    \Comment{\eref{eq-q-qnet-mva-dlsld}}
%    \State $Q^s=\la \theta^s
%    \bar{S}^s (1 + Q^s)$ for all
%    $s \in \mathrm{FIFO}$
%  \EndFor
%  \State The throughput at station $1$ is $\la$
%  \State The throughput of class $c$ at station $s$ is $\la
% \theta^s_c$
%  \State The mean number of customers of class $c$ at FIFO
%  station
%  $s$ is $Q^s\theta^s_c/\theta^s$
%  \State The mean number of customers of class $c$ at
%  IS
%  station
%  $s$ is $\la \theta^s_c \bar{S}^s_c$
%\end{algorithmic}\label{algo-q-mva}
% \end{algorithm}
%\begin{figure}[htbp]
%   \Ifig{botana}{0.8}{0.4}
%   \mycaption{Throughput in transactions per second versus number of users,
%   computed with MVA for the network
%   in \fref{fig-bottleneckAna}. The dotted lines are
%   the bounds of bottleneck analysis in \fref{fig-bot-lamres}.}
%  \mylabel{fig-q-qnet-botana}
%\end{figure}
%
%
%\begin{ex}{Mean Value Analysis of \fref{fig-bottleneckAna}}
%We model the system as a single class, closed
%network. The CPU is modelled as a PS station,
%disks A and B as FIFO single servers, and think
%time as an IS station. We fix the visit rate
%$\theta^{\mathrm{think time}}$ to $1$ so that
%$\theta^{\mathrm{CPU}}=V_{\mbox{CPU}}$,
%$\theta^{\mathrm{A}}=V_{\mbox{A}}$ and
%$\theta^{\mathrm{B}}=V_{\mbox{B}}$. Note that the
%routing probabilities need not be specified in
%detail, only the visit rates are required.
%
%The CPU station is not a FIFO station, but is has
%the same station function, therefore we may apply
%MVA and treat it as if it would be FIFO.
%
%\fref{fig-q-qnet-botana} shows the results, which
%are essentially as prediced by bottleneck
%analysis in \fref{fig-bot-lamres}.
%
%\label{ex-q-qnet-djerzcbasvvajhqoejfz}
%\end{ex}
%
%
%
%\subsubsection{General MVA}

%
% Be careful that this is different from
%the procedure used in the station in isolation.
%Here, we are replacing station $s_0$ by a
%short-circuit, i.e. a station where the service
%requirement is $0$; in contrast, with the station
%in isolation, we replace a network part by a
%station with unit rate and unit service
%requirement.
%
%
%With the same assumptions and notation as in
%
%
%Then it follows immediately from
%\eref{eq-q-qnet-convol-zt} that
%  \be
%  F_{\eta}(\vY) = G^{s}(\vZ^{s})F_{\eta^{[s]}}(\vY)
%  \ee
%This can also be written explicitly as
% \be
% \eta(\vK) =
% \sum_{\vn^{s} \mst \vzero \leq\kappa(\vn^{s}) \leq \vK}
% f^{s}(\vn^{s})\prod_{c=1}^C\lp
% \theta^{s}_c\rp^{n^{s}_c} \eta^{[s]}\lp \vK-\kappa(\vn^{s})\rp
% \label{eq-q-qnet-convolkjsaj}
%  \ee
%  where
%$\kappa(\vn^{s})$ is the closed chain population
%vector that corresponds to state $\vn^{s}$, i.e.
%$\kappa(\vn^{s})_{\calC}=\sum_{c \in \calC} n_c$
%(in particular, $\vK-\kappa(\vn^{s})$ is the
%closed chain population vector obtained when we
%remove $n^{s}_c$ customers of class $c$ for every
%$c$). The notation $\vzero \leq \kappa(\vn^{s})
%\leq \vK$ means
%  $0 \leq \kappa(\vn^{s})_{\calC}\leq K_{\calC}$
%  for all chain $\calC$.
%
%
%
%
% and the number of chain
%$\calC$ customers in $\vn^s$ is $0$ (i.e.
%$\lp\kappa(\vn^s)\rp_{\calC}=0$) then
%  \be
% P^s\lp \vn^s \left | \vK\right. \rp
% =\frac{\la_{\calC}(\vK)}{\la^{[s]}_{\calC}(\vK)}
%  P^s\lp \vn^s \left | \vK- \vone_{\calC}\right. \rp
% \mif
%  K_{\calC}\geq 1
%  \label{eq-q-mva-jduiewtgskxasdff}
% \ee
%
%states where station $s$ is empty are feasible,
%the probability that station $s$ is empty is \be
%P^{s}\lp\vzero\left|\vK\rp\right. =
% \frac{\eta^{[s]}\lp \vK \rp}{\eta(\vK)}
% \label{eq-q-convol-p0}
% \ee
% and, if $K_{\calC}\geq 1$:
%  \be
% P^s\lp \vzero \left | \vK\right. \rp
% =\frac{\la_{\calC}(\vK)}{\la^{[s]}_{\calC}(\vK)}
%  P^s\lp \vzero \left | \vK- \vone_{\calC}\right. \rp
%  \label{eq-q-mva-jduiewtgskxasdff}
% \ee
%
% \bear
% \mif n^s_c = 0 \mfa c \in \calC\; : p^*(\vn^s) & = &  \frac{1}{\la^{[s]}_{\calC}(\vK)}
%  P^s\lp \vn^s \left | \vK- \vone_{\calC}\right. \rp
%  \\
%  \melse \;  p^*(\vn^s) & = &
%   P^s\lp\vn^s-\vone_c\left|\vK-\vone_{\calC}\rp\right.
%   \frac{1}{\mu^{*s}_c(\vn^s)}
%   \label{eq-q-qnet-dskjlsdfjkfdjk1}
%   \\
%   && \mbox{where } c
%   \mbox{ is the smallest class label } \in \calC \mbox{ such that } n^s_c>0
%    \nonumber
%  \\
%  p^*(\vzero) & = &  \frac{1}{\la^{[s]}_{\calC}(\vK)}
%  P^s\lp \vzero \left | \vK- \vone_{\calC}\right. \rp
%  \label{eq-q-qnet-dskjlsdfjkfdjk2}
%  \eear
  %\bear
%  p^*(\vn^s) & = &
%   P^s\lp\vn^s-\vone_c\left|\vK-\vone_{\calC}\rp\right.
%   \frac{1}{\mu^{*s}_c(\vn^s)} \mif \vn^s \neq
%   \vzero, \label{eq-q-qnet-dskjlsdfjkfdjk1}
%   \\
%   && \mbox{where } c
%   \mbox{ is the smallest class label } \in \calC \mbox{ such that } n^s_c>0
%    \nonumber
%  \\
%  p^*(\vzero) & = &  \frac{1}{\la^{[s]}_{\calC}(\vK)}
%  P^s\lp \vzero \left | \vK- \vone_{\calC}\right. \rp
%  \label{eq-q-qnet-dskjlsdfjkfdjk2}
%  \eear
% \begin{algorithm}
% \mycaption{General MVA for a closed multi-class product form network.}
% \begin{algorithmic}[1]
%%
%  \State $K=$: population size
%  \State $p(n)$, $n=0...K$: probability
%  that there are $n$ customers at the FIFO
%  station
%  \State $\la$: throughput
%  \State $p(0)=1$, $p(n)=0$, $n=1...K$
%  \For{$k=1:K$}
%     \State $p^*(n) = p(n-1) \bar{Z} \;/\; \min\lp n
%     ,B\rp$,
%     $n=1...k$
%     \Comment{Unnormalized $p(n|k)$, \eref{eq-qnet-dklfsjfdkl1}}
%     \State $p^*(0) = p(0) \bar{Z}/ k$
%     \Comment{Unnormalized $p(0|k)$, \eref{eq-qnet-dklfsjfdkl2}}
%     \State $\la= 1 / \sum_{n=0}^k p^*(n)$
%     \State $p(n)=p^*(n)/\la$, $n=0...k$
%  \EndFor
%\end{algorithmic}\label{algo-q-mva-gen}
% \end{algorithm}

\section{What This Tells Us}

\subsection{Insensitivity} Multi-class product
form queuing networks are \imp{insensitive} to a
number of properties:
\begin{itemize}
    \item The distribution of service times is irrelevant
        for all insensitive stations; the stationary
        distributions of numbers of customers and the
        throughput depend only on traffic intensities (by
        means of the visit rates $\theta^s_c$) and on the
        station functions, which express how rates are
        shared between classes. The service distribution
        depends on the class, and classes may be used to
        introduce correlations in service times. The
        details of such correlations need not be modelled
        explicitly, since only traffic intensities matter.


        By Little's law, the mean response times are also
        insensitive (but not the distribution of response
        time, see \sref{sec-q-ps}).
    \item The nature of the service station plays a
    role only through its station function. Very
    different queuing disciplines such as FIFO or
    global PS, or global LCFSPR with class
    independent service times have the same
    station function, hence the same stationary
    distributions of numbers of customers,
    throughputs and mean response times
    also irrelevant as long
    \item The details of routings are also irrelevant, only
        the visit rates matter. For example, in
        \fref{fig-q-chains}, it makes no difference if we
        assume that external jobs visit station 1 only
        once, without feedback.
\end{itemize}
\begin{figure}
\Ifig{bonald}{1.0}{0.6}
 \mycaption{Product form queuing network used to model
 the Internet in \cite{bonald2003insensitive}.}
 \label{fig-q-qnet-bonald}
\end{figure}
\begin{ex}{Internet Model \cite{bonald2003insensitive}}
Internet users as seen by an internet provider
are modelled by Bonald and Prouti\`{e}re in
\cite{bonald2003insensitive} as follows (they use
a slightly different terminology as they do not
interpret a Whittle network as a product form
station as we do).

Users sessions arrive as a Poisson process. A
session alternates between active and think time.
When active, a session becomes a flow and
acquires a class, which corresponds to the
network path followed by the session (there is
one class per possible path). A flow of class $c$
has a service requirement drawn from any
distribution with finite mean $\bar{S}_c$. The
network shares its resources between paths
according to some ``bandwidth" allocation
strategy. Let $\mu_c(\vn)$ be the rate allocated
to class $c$ flows, where $\vn=(n_1, ..., n_C)$
and $n_c$ is the number of class $c$ flows
present in the network. We assume that it derives
from a balance function $\Phi$, i.e.
  \be
 \mu_c(\vn)=\frac{\Phi(\vn-\vone_c)}{\Phi(\vn)}
 \label{eq-q-qnet-bon-aal}
 \ee
All flows in the same
class share the bandwidth allocated to this class
fairly, i.e. according to processor sharing.

When a flow completes, it either leaves the
network, or mutates and becomes a session in
think time. The think time duration has any
distribution with a finite mean $S_0$. At the end
of its think time, a session becomes a flow.

This can be modelled as a single chain open
network with two stations: a Per-Class PS station
for flow transfers and an IS station for think
time, as in \fref{fig-q-qnet-bonald}.

A session in think time may keep the class it
inherited from the flow. This means that we allow
the classes taken by successive flows to be non
iid, as is probably the case in reality (for
example the next flow of this session might be
more likely to take the same path). In fact, we
may imagine any dependence, it does not matter as
long as the above assumptions hold, since we have
a product form queuing network; only the traffic
intensities on each flow path matter, as we see
next.

With the assumption in \eref{eq-q-qnet-bon-aal},
flow transfers are represented by means of a
per-class processor sharing station with Whittle
function $\Phi(\vn)$ (this is also called a
Whittle network); think times are represented by
a constant rate infinite server station; both are
category 1 stations, thus the network has product
form.

More precisely, let $\theta_c$ be the visit rate
at the Per-Class PS station, class $c$; it is
equal to the number of class $c$ flow arrivals
per time unit. Similarly, $\theta_0$ is the
number of arrivals of sessions in think time per
time unit. Let $n_0$ be the number of flows in
think time; the stationary probability
distribution of $(n_0, \vn)$ is, by the product
form theorem:
  \bear
  P(n_0, \vn) &=&\eta
  \Phi(\vn) \prod_{c=1}^C\lp \bar{S}_c \theta_c
  \rp^{n_c}
 \lp \theta_0 \bar{S}_0\rp^{n_0}
 \nonumber \\
 &=& \eta \Phi(\vn) \prod_{c=1}^C \rho_c^{n_c}\rho_0^{n_0}
\label{eq-q-qnet-dsafkjdefjhkds}
  \eear
where $\eta$ is a normalizing constant and
$\rho_c=\theta_c \bar{S}_c$, $ \rho_0=\theta_0
\bar{S}_0$ are the traffic intensities.

\eref{eq-q-qnet-dsafkjdefjhkds} is a remarkably
simple formula. It depends only on the traffic
intensities, not on any other property of the
session think times or flow transfer times. It
holds as long as bandwidth sharing (i.e. the rates
$\mu_c(\vn)$) derives from a
balance function. In \cite{bonald2003insensitive}
it is shown that this is also a necessary
condition.

This is used by the authors in \cite{bonald2003insensitive} to advocate that bandwidth sharing be performed using a balance function. Bandwidth sharing is the function, implemented by a network, which decides the values of $\mu_c(\vn)$ for every $c$ and $\vn$. The set $\calR$ of feasible rate vectors $(\mu_c(\vn))_{c=1...C}$ is defined by the network constraints. For example, in a wired network with fixed capacities, $\calR$ is defined by the constraints $\sum_{c \in \ell}\mu_c < R_l$ where $\ell$ is a network link, $R_l$ its rate, and ``$c\in \ell$" means that a class $c$ flow uses link $\ell$. The authors define \nt{balanced fairness} as the unique
allocation of rates to classes that (1) derives from a balance function and (2) is optimal in the sense that for any $\vn$, the rate vector $(\mu_c(\vn))_{c=1...C}$ is at the boundary of the set of feasible rate vectors $\calR$. They show that such an allocation is unique; algorithms to compute the balance function are given in \cite{bonald2004calculating}.
\end{ex}



%\subsubsection{The Multi-Class Paradigm}
%Ex PS analysis
%
%In a PS station, mean response time for a class
%$c$ customer is proportional to $\bar{S}_c$.

\subsection{The Importance of Modelling Closed Populations}
\begin{figure}
\Ifig{engset}{0.6}{0.4} \mycaption{Model used to
derive the Engset formula} \label{fig-engset}
\end{figure}

Closed chains give a means to account for feedback in the
system, which may provide a different insight than the single
queue models in \sref{sec-q-sq}; this is illustrated in
\sref{sec-q-ex}, where we see that the conclusion (about the
impact of capacity doubling) is radically different if we
assume an infinite population or a finite one.

Another useful example is the Engset formula, which we now
describe. The Erlang loss formula gives the blocking
probability for a system with $B$ servers, general service time
and Poisson external arrivals. If the population of tasks using
the system is small, there is a feedback loop between the
system and the arrival process, since a job that is accepted
cannot create an arrival. An alternative to the Erlang loss
formula is the model in \fref{fig-engset}, with a finite
population of $K$ jobs, a single class of customers, and two
stations. Both stations are IS; station 1 represents the
service center with $B$ resources, station 2 represents user
think time. If station $1$ has $B$ customers present, arriving
customers are rejected and instantly return to station 2 where
they resume service. Service requirements are exponentially
distributed. This is equivalent to the form of blocking called
partial blocking in \sref{sec-q-qnets-blocking}. This form of
blocking requires that routing be reversible; since there are
only two stations, the topology is a bus and the routing is
reversible, thus the network has product form.

It follows that the probability $P(n|K)$ that
there are $n$ customers in service, given that
the total population is $K \geq B$, is given by
the product form theorem and the station
functions for IS:
 \be
 P(n|K) = \frac{1}{\eta}
 \frac{\lp\bar{S}^1\rp^n}{n!}\frac{\lp\bar{S}^2\rp^{K-n}}{(K-n)!}
 \ee
where $\eta$ is a normalizing constant,
$\bar{S}^1$ is the average processing time and
$\bar{S}^2$ the average think time. Let
$\rho=\frac{\bar{S}^1}{\bar{S}^2}$; it comes:
  \ben
  \eta  =  \sum_{n=0}^B \frac{\rho^n}{n! (K-n)!}
  \een

The blocking probability $P^0(B|K)$ for is equal
to the Palm probability for an arriving customer
to find $B$ customers in station $1$. By the
arrival theorem, it is equal to $P(B|K-1)$. Thus
for $K > B$
  \be
  P^0(B|K) = \frac{\frac{\rho^B}{B! (K-B-1)!}}
  {\sum_{n=0}^{B} \frac{\rho^n}{n!
  (K-n-1)!}}
  \label{eq-q-qnet-engset}
  \ee
 and $P^0(B|K)=0$ for $K\leq B$.
\eref{eq-q-qnet-engset} is called the \nt{Engset
formula} and gives the blocking probability for a
system with $B$ resources and a population of
$K$. Like the Erlang-loss formula the formula is
valid for any distribution of the service time
(and of the think time). When $K\to \infty$, the
Engset formula is equivalent to the Erlang-loss
formula.


\begin{petit}
\section{Mathematical Details About Product-Form Queuing
Networks}
 \label{sec-q-form-def}

\subsection{Phase Type Distributions}
\label{sec-q-ph}
For insensitive stations, the
service time distribution is assumed to be a \nt{phase type} distribution; this is also called a \nt{mixture of exponentials} or a
\nt{mixture of gamma} distribution and is defined next.
Note that the product form theorem implies that the stationary
distribution of the network is insensitive to any property of the distribution
of service requirement other than its mean; thus its seems plausible to conjecture that the product form network continues to apply if we relax the phase type assumption. This is indeed shown for networks made of Kelly stations and of ``Whittle network" stations in \cite{barbour1976networks}.

A non negative random variable $X$ is
said to have a phase type distribution if there
exists a continuous time Markov chain with finite
state space $\lc 0,1, ..., I \rc$ such that $X$
is the time until arrival into state $0$, given
some initial probability distribution.

Formally, a phase type distribution with $n$
stages is defined by the non negative sequence
$(\alpha_j)_{j=1...n}$ with $\sum_j \alpha_j=1$
and the non negative matrix
$(\mu_{j,j'})_{j=1...n, j'= 0...n}$. $\alpha_j$ is
the probability that, initially, the chain is in
state $j$ and $\mu_{j,{j'}}\geq 0$ is the transition
rate from state $j$ to ${j'}$, for $j\neq {j'}$. Let
$F_j(s)$ be the Laplace-Stieltjes transform of
the time from now to the next visit to state $0$,
given that the chain is in state $j$ now. By the
Markov property, the Laplace-Stieltjes transform
of the distribution we are interested in is
$\esp{e^{-sX} } = \sum_{j\neq 0 }\alpha_j F_j(s)$
for all $s>0$. To compute $F_{j}(s)$ we use the
following equations, which also follow from the
Markov property:
 \be
 \forall j \in \lc 0,1, ..., j \rc: \lp s+\sum_{{j'}\neq j}\mu_{j,{j'}}\rp F_j(s)
 =
 \mu_{j,0}
 +
 \sum_{{j'} \neq j, {j'}\neq 0} \mu_{j,{j'}}F_{j'}(s)
 \label{eq-q-ph-type}
 \ee
\begin{figure}[htbp]
  \insfig{ph}{0.7}
  \mycaption{Mixtures of Exponential: a
  Phase Type distribution is the distribution of the time until absorption into state $0$
  (state $0$ represents the exit and is not shown).
  The Erlang and Hyperexponential are special cases.}
  \mylabel{fig-q-st-ph-type}
\end{figure}

Consider for example the \nt{Erlang-$n$} and
\nt{Hyper-Exponential} distributions, which
correspond to the Markov chains illustrated in
\fref{fig-q-st-ph-type}. The Laplace-Stieltjes
transform of the Erlang-$n$ distribution is
$F_1(s)$, which is derived from
\eref{eq-q-ph-type}:
 \bearn
(\la + s) F_1(s) & = & \la F_2(s)
\\
&...&
\\
(\la + s) F_{n-1}(s) & = & \la F_n(s)
\\
 (\la + s) F_{n}(s) & = & \la
\eearn
 and is thus $\lp \frac{\la}{\la + s}\rp^n$.  This could also be
obtained by noting that it is the convolution of
$n$ exponentials. (Note that this is a special
case of Gamma distribution). The PDF is
$f(x)=\la^n \frac{x^{n-1}}{(n-1)!}e^{-\la x}$.
The mean is $\bar{S}=\frac{n}{\la}$; if we set
the mean to a constant and let $n\to \infty$, the
Laplace Stieltjes transform converges, for every
$s>0$, to $e^{-s \bar{S}}$, which is the Laplace
Stieltjes transform of the constant concentrated
at $\bar{S}$. In other words, the Erlang-$n$
distribution can be used to approximate a
constant service time.

Similarly, the Laplace Stieltjes transform of the
Hyper-Exponential distribution follows
immediately from \eref{eq-q-ph-type} and is
$\sum_{{j}=1^n}\frac{\alpha_{j} \la_{j}}{\la_{j} +s}$
 and the PDF is $f(x)=\sum_{{j}=1^n}\alpha_{j} e^{-\la_{j'} x}$. This
 can be used to fit any arbitrary PDF.

\subsection{Micro and Macro States}
\label{sec-micro}
The state of every station is defined by a
\nt{micro-state}, as follows.
\begin{description}
\item[Insensitive Station] The micro state is
$(\calB, \calJ)$ where $\calB$ is the state of
the station buffer introduced in
\sref{sec-q-qnet-cat} and $\calJ$ is a data
structure with the same indexing mechanism, which
holds the service phase for the customer at this
position. In other words, for every index $i$ in
the index set of the buffer, $\calB_i$ is the
class of the customer present at this position,
and $\calJ_i$ is the service phase of the same
customer (if there is no customer present at this
position, both are $0$). A customer at position
$i$ receives a service rate $\rho_i(\calB)$ given
by \eref{eq-def-sr}. This means that the
probability that
 this customer moves from the service phase
 $j=\calJ_i$ to a next phase $j'$ in a time interval of duration
$dt$ is $\rho_i(\calB) \mu^c_{j,j'}dt + o(dt)$
where $c=\calB_i$ is this customer's class and
$\mu^c_{j,j'}$ is the matrix of transition rates
at this station for class $c$ customers, in the
phase type representation of service requirement.
If the next service phase is $j'=0$, this
customer will leave the station. When a class $c$
customer arrives at this station, it is inserted
at position $i$ in the buffer with probability
given in \eref{eq-q-q-net-ins-hjkdf}; the initial
stage is set to $j$  with probability
$\alpha^c_j$, the initial stage distribution
probability for customers of this class at this
station, and $\calJ_i$ is set to $j$.

\item[MSCCC Station] The micro-state is an
ordered sequence of classes $(c_1, c_2, ...,
c_M)$ where $M$ is the number of customers
present in the station. When a customer arrives,
it is added at the end of the sequence. The
customers in service are the first $B$
\emph{eligible} customers; a customer in position
$m$ is eligible if and only if there is a token
available
 (i.e. $\sum_{m'=0}^{m-1}\ind{\calG\lp c_{m'}\rp =g}<T_g$ with $g=\calG\lp c_{m}\rp$)
  and there
 is a server available (i.e.
 $\sum_g \min\lp T_g,\sum_{m'=0}^{m-1}\ind{\calG\lp c_{m'}\rp
 =g}\rp<B
 $). There is no state information about the
 service stage, since this category of station
 requires that the service times be exponentially
 distributed, hence memoryless. The probability
 that
 an eligible customer leaves the station in a time interval of duration
$dt$ is $\frac{1}{\bar{S}} r(M)dt + o(dt)$ where
$r(M)$ is the rate of this station when $M$
customers are present and $\bar{S}$ is the mean
service time (both are independent of the
class). Non eligible customers may not leave the station.
%\item[Type 2 or 3] The micro-state is an ordered
% sequence $(c_1, j_1, c_2, j_2, ..., c_M, j_M)$
% where $c_m$ is the class of the $m$th customer
% in the sequence and $j_m$ is the service phase
% for that customer (it can be any phase other than $0$; see above for the definition of phase type
% distribution).
%
% For a type 2 queue, an arriving customer is
% inserted into the list at position $i$ with probability
% $p^{m+1}(i)$ where $m$ is the queue size before
% arrival; the service rate for the $m$th customer in
% the queue is $\mu(m)=p^M(m) r_{c_m}(n_{c_m})$ where $n_{c_m}$ is the number
% of customers of the same class and $r_c$ is the service rate
% function. This means that the probability that
% this customers moves from the service phase
% $j_m$ to a next phase $j'$ in one time interval of duration
%$dt$ is $\mu(m) A^c_{j_m,j'}dt + o(dt)$. If the
%next service phase is $j'=0$, this customer will
%leave the station.
%
%  For a type 3 queue, the
% position is irrelevant so we may assume
% insertion is in order of arrival. The service rate for the $m$th customer in
% the queue is $\mu(m)= r_{c_m}$.
 \end{description}
 The \emph{global micro state} of the network is the
 sequence $(e_1, e_2, ..., e_S)$ where $e_s$ is
 the micro-state of station $s$. With the
 assumptions above, this defines a continuous
 time Markov chain. A network is defined by the population
 in closed chains, $K_{\calC}$.
 The \nt{global micro state space},
 $\calM$, is the set of all $(e_1, e_2, ...,
 e_S)$ that are possible given the rules of each
 station and
 \begin{enumerate}
    \item the total
 number of customers in chain $\calC$ present
 anywhere in the network is $K_{\calC}$, if $\calC$ is a closed chain, and is any
 non negative integer otherwise;
    \item  %routing has reached a stationary regime, more precisely,
if the visit rate $\theta^s_c$ is $0$ for some
station $s$ and class $c$, then there may not be
any customer of class $c$ at station $s$.
\end{enumerate}

 %We
%assume that this state of spaces is fully
%connected, which is generally true except in
%pathological
% cases where the order of customers is preserved throughout the network lifetime
% (for example a cyclic network with only type FIFO stations with only FIFO stations).
%The condition is true in particular, if every
%chain visits at least one type 2 or type 3
%station.

The \nt{macro-state} of station $s$ is the vector
$\vn^s=(n_1^s,...,n_C^s)$ where $n_c^s$ is the
number of class-$c$ customers present at this
station. The global macro-state is the collection
$(\vn^s)_{s=1...S}$; the global macro state does
not define a Markov chain as too much information
is lost (for MSCCC stations, we lost the order
of customers; for insensitive stations, we lost
the service phase). The micro-state description
is required to prove theorems, but most formulas
of interest are expressed in terms of
macro-states. The \nt{global macro state space},
$\calL$, is the set of all $(\vn^s)_{s=1...S}\geq
\vzero$ such that
\begin{enumerate}
    \item $\sum_{c \in \calC, s}n^s_c = K_{\calC}$ for
every closed chain $\calC$;
    \item  %routing has reached a stationary regime, more precisely,
if the visit rate $\theta^s_c$ is $0$ for some
station $s$ and class $c$, then $n^s_c=0$.
\end{enumerate}

\subsection{Micro to Macro: Aggregation
Condition} \label{sec-aggreg} All results in the
previous sections apply to the macro-state
description of the network. In the given form,
they require that the aggregation condition
holds, which says that
  \begin{quote}
  aggregation of state from
micro to macro does not introduce non feasible
micro states.
 \end{quote}
This is equivalent to saying that the set $\calM$
is fully connected, i.e. any global micro state
can be reached from any initial condition in a
finite number of transitions of the underlying
Markov chain. This is generally true except in
pathological cases where the order of customers
is preserved throughout the network lifetime.
Consider for example a cyclic network with only
FIFO stations and one customer per class. The
initial order of customers cannot be changed and
only states in $\calM$ that preserve the initial
ordering are feasible. In such a network, product
form does hold, but formulas for macro states
are different than given in this chapter as the numbers of microstates that
give one specific macro-state is smaller.


\subsection{Local Balance In Isolation}
%The product
%form theorem \ref{theo-q-pf} and the resulting
%independence for the open case in
%\thref{theo-q-pf-open} also hold for the
%micro-states; in particular, for an open network,
%the micro-states at different stations are
%independent. We do not give the details here, as
%this requires defining the station function for
%micro-state; the proof is in
%\cite{baskett1975open,le1986bcmp,MR563738,berezner1995quasi}
%for details.
The station function can be defined both at the micro and macro
levels. Formally, the \nt{station function at micro level} is a
function $F(e)$, if it exists, of the micro state $e$ of the
function in isolation, such that $F(\O)= 1$, where $\O$ is the empty state, and the
stationary probability of state $e$ in the station in isolation
is $\eta(\vK) F(e)$, where $\eta(\vK)$ is a normalizing
constant that depends on the total populations of customers
$K_c$ for every class $c$, in the station in isolation.

We say that a station satisfies the property of \nt{Local
Balance In Isolation} if the following holds.%
%
%
%The proof of the product form theorem is based on
%the analysis of stations in isolation; the key
%property is that, for a micro-state description
%of a station in isolation, the following holds
%(called \nt{partial balance})
For every
micro-state $e$ and class $c$:
 \be
 \barr{c}
 \mbox{departure rate out of state } e
 \mbox{ due to a class } c \mbox{ arrival}
 \\
 =
 \\
 \mbox{arrival rate into state } e
 \mbox{ due to a class } c \mbox{ departure}
 \earr
 \label{eq-q-qnet-lb1}
 \ee
In this formula, the rates are with respect to
the stationary probability of the station in
isolation, as defined earlier. It follows that
one must also have
 \be
 \barr{c}
 \mbox{departure rate out of state } e
 \mbox{ due to a departure or an internal transfer, of any class }
 \\
 =
 \\
 \mbox{arrival rate into state } e
 \mbox{ due to an arrival or an internal transfer, of any class }
 \earr
 \label{eq-q-qnet-lb2}
 \ee where an internal transfer is a change of state without arrival nor departure (this
is for insensitive stations, and is a change of phase for
one customer in service). The collection of all these equations
is the local balance in isolation. If one finds a station
function such that local balance in isolation holds, then this
must be the stationary probability of the station in isolation,
up to a multiplicative constant.

For example, consider a FIFO station with 1 server and
assume that there is one class per chain in the
network (i.e. customers do not change class). Let
$F(c_1, ..., c_M)$ be the stationary probability
for the station in isolation. Local balance
  here writes:
 \bearn
 F(c_1, ..., c_M)  \ind{\sum_{m=1}^M \ind{c_m=c} < K_c} &=& F(c,c_1,
 ..., c_M) \mu \mbox{ for all class } c
 \\
F(c_1, ..., c_M)\mu &= & F(c_1, ...,
  c_{M-1})\ind{\sum_{m=1}^{M-1} \ind{c_m=c_M} <  K_{c_M}}
 \eearn where $K_c$ is the number of class $c$ customers in the
 system and
 $\mu=\frac{1}{\bar{S}}$. The function $F(c_1, ...,
 c_M)= \bar{S}^M$ satisfies both of these types of equations,
 therefore it is equal to the stationary probability of the
 station in isolation, up to a multiplicative constant. $F(c_1, ...,
 c_M)= \bar{S}^M$ is the microscopic station
 function. The station function $f(\vn)$ given earlier
 follows by aggregation; indeed, let $\calE(n_1,
  ...,n_C)$ be the set of micro-states of the FIFO station with $n_c$
  customers of class $c$, for every $c$.
  \ben
  f(n_1, ...,n_C)= \sum_{e \in \calE(n_1,
  ...,n_C)}\bar{S}^{(n_1+ ...+n_C)}=
  \frac{(n_1+ ...+n_C)!}{n_1! ... n_C!}
  \bar{S}^{(n_1+ ...+n_C)}
  \een since $\frac{(n_1+ ...+n_C)!}{n_1! ...
  n_C!}$ is the number of elements of $\calE(n_1,
  ...,n_C)$. This is exactly the station function
  for the FIFO station described in \eref{eq-q-stat-fifo1}.

\subsection{The Product Form Theorem}
The product form theorem in \ref{theo-q-pf} is a direct
consequence of the following main result.
\begin{theorem}Consider a multi-class network with Markov
routing and $S$ stations. Assume all $S$ stations satisfy local
balance in isolation, and let $F^s(e^s)$ be the station
function at micro level for station $s$, where $e^s$ is the
micro state of station $s$. Then
  \be p(e^1, e^2, ...,e^S)\eqdef \prod_{s=1}^S F^s(e^s)
 \label{eq-q-pf-th}
   \ee
  is an invariant measure for the network.
  \label{theo-q-qnets-pff}
\end{theorem}
The theorem implies that, if appropriate stability conditions
hold, the product $p(e^1, e^2, ...,e^S)$ must be equal to a
stationary probability, up to a normalizing constant. The proof
can be found in \cite{MR563738}; see also
\cite{kelly1979reversibility,baskett1975open}. It consists in
direct verification of the balance equation. More precisely,
one shows that, in the network:
  \be
 \barr{c}
 \mbox{departure rate out of state } e
 \mbox{ due to a departure of any class }
 \\
 =
 \\
 \mbox{arrival rate into state } e
 \mbox{ due to an arrival of any class }
 \earr
 \label{eq-q-qnet-lb1n}
 \ee
In this formula, the rates are with respect to the joint
network probability of all stations at micro level, obtained by
re-normalizing $p()$. Note that the local balance property, as
defined in \eref{eq-q-qnet-lb1}, does not, in general, hold
inside the network at the micro level.

If the aggregation condition holds, then one can sum up
\eref{eq-q-pf-th} over all micro states for which the network
population vector is $\vn$  and obtain \eref{eq-q-pf}, which is
the macro level product form result. Note that, at the
macro-level, one has, in the network, and for any class $c$:
  \be
 \barr{c}
 \mbox{departure rate out of state } e
 \mbox{ due to a class } c \mbox{ departure}
 \\
 =
 \\
 \mbox{arrival rate into state } e
\mbox{ due to a class } c \mbox{ arrival}
 \earr
 \label{eq-q-qnet-lb1n-ag}
 \ee
In this formula, the rates are with respect to the joint
network probability of all stations at macro level. Note the
inversion with respect to local balance.

The resulting independence for the open case in
\thref{theo-q-pf-open} therefore also holds for micro-states:
in an open network, the micro-states at different stations are
independent.

The proof of the product form theorem \ref{theo-q-pf} follows immediately from
\thref{theo-q-qnets-pff} and the fact that all stations in our catalog
satisfy the property of local balance in isolation. The proof that MSCCC stations
satisfy the local balance property is in
\cite{le1986bcmp,berezner1995quasi}. For Kelly-Whittle stations, the
result was known before for some specific cases. For the general case, it is novel:
 \begin{theorem}
Kelly-Whittle stations satisfy local balance in isolation.
 \label{theo-kw}\end{theorem}
The
proof is in \sref{sec-q-proofs}.

\subsection{Networks with Blocking}
\label{sec-q-qnets-blocking} It is possible to extend Markov
routing to state-dependent routing, In particular, it is
possible to allow for some (limited) forms of blocking, as
follows. Assume that there are some constraints on the network
state, for example, there may be an upper limit to the number
of customers in one station. A customer finishing service, or,
for an open chain, a customer arriving from the outside, is
denied access to a station if accepting this customer would
violate any of the constraints. Consider the following two
cases:
  \begin{description}
    \item[Transparent Stations with Capacity Limitations]
        The constraints on the network state are expressed
        by $L$
capacity limitations of the form \be \sum_{(s,c) \in
  \calH_{\ell}}n^s_c \leq \Gamma_{\ell}, \;\; \ell=1...L
  \label{eq-q-qnet-caplim} \ee
where $n^s_c$ is the number of class $c$ customers present
at station $c$, $ \calH_{\ell}$ is a subset of $\lc 1,...,S
\rc  \times
  \lc 1,...C\rc $ and $\Gamma_{\ell}\in \Nats$. In other
words, some stations or groups of stations may put limits
on the number of customers of some classes or groups of
classes.


    If a customer is denied access to station $s$, she
    continues her journey through the network, using Markov
    routing with the fixed matrix $Q$, until she finds a
    station that accepts her or until she leaves the
    network.
    \item[Partial Blocking with Arbitrary Constraints] The
        constraints can be of any type. Further, If a
        customer finishes service and is denied access to
        station $s$, she stays blocked in service. More
        precisely, we assume that service distributions are
        of phase type, and the customer resumes
        the last completed service stage. If the customer
        was arriving from the outside, she is dropped.

    Further, we need to assume that Markov routing is
    \nt{reversible}, which means that \be
       \theta^s_c q^{s,s'}_{c,c'} =
       \theta^{s'}_{c'}q^{s',s}_{c',c}
       \label{eq-q-qnet-rev} \ee for all $s,s',c,c'$.
    Reversibility is a constraint on the topology; bus and
    star networks give reversible routing, but ring
    networks do not.
  \end{description}

Assume in addition that the service requirements are
exponentially distributed (but may be class dependent at
insensitive stations). Then the product form theorem continues
to apply for these two forms of blocking
\cite{pittel1979closed,leny,kelly1979reversibility}. There are
other cases, too, see \cite{balsamo2000product} and references
therein.

There is a more general result: if the service distributions
are exponential and the Markov routing is reversible, then the
Markov process of global micro-states is also reversible
\cite{le1987interinput}. Let $X_t$ be a continuous time Markov
chain with stationary probability $p()$ and state space
$\calE$. The process is called \nt{reversible} if
$p(e)\mu(e,e')=p(e')\mu(e',e)$ for any two states $e, e' \in
\calE$, where $\mu(e,e')$ is the rate of transition from $e$
to $e'$. Reversible Markov chains enjoy the following
\nt{truncation property} \cite{kelly1979reversibility}. Let
$\calE'\subset \calE$ and define the process $X'_t$ by forcing
the process to stay within $\calE'$; this is done by taking
some initial state space $e \in \calE'$ and setting to $0$ the
rate of any transition from $e \in \calE$ to $e'\in \calE'$.
Then the restriction of $p$ to $\calE'$ is an invariant
probability; in particular, if $\calE'$ is finite and fully
connected, the stationary probability of the truncated process
is the restriction of $p$ to $\calE'$, up to a normalizing
constant.

Note that setting to $0$ the rates of transitions from $e \in
\calE$ to $e'\in \calE'$ is equivalent to saying that the we
allow the transition from $e$ to $e'$ but then force an
immediate, instantaneous return to $e$; this explains why we
have product form for networks with partial blocking with
arbitrary constraints.

\end{petit}
