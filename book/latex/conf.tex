\begin{minipage}[b]{0.50\textwidth}
%LAISSER TOMBER VISUAL DISPLAY POUR L'INSTANT
%

In most measurements or simulations, we obtain large amounts of
data. Displaying the data correctly is important, and implies
to use some graphical packages, or maths packages with
graphical capabilities (Different tools have different
capabilities, and produce graphics of different aesthetic
value; but the most important is to use one tool that you know
well). Tools do not do everything and you need to know what to
represent. We discuss important and frequent summarizations
that can be used to display and compare data: the complete
distribution; summarized quantities such as mean, standard
deviation, median and tail quantiles; fairness indices.
\end{minipage}
%
\begin{minipage}[b]{0.50\textwidth}
\hspace{1cm} ~~~~
\insfig{confSigne}{0.9} ~\\
%~\\
%~\\
%~\\
\end{minipage}
We discuss some properties of these summarization and indices;
they are not all equivalent, and some, though less frequently
used, are preferable if one has a choice; for example, the
Lorenz curve gap is more robust than Jain's Fairness index
(which is essentially the same as the standard deviation
rescaled by the mean) and should be preferred.

Simulation and measurement usually contain some randomness,
therefore it is important to capture the uncertainty about
measured performance. This is done with confidence or
prediction intervals; we discuss the use and interpretation of
both. There are many different ways for defining a confidence
or prediction interval; some are robust and some not. We give
useful, and simple formulas and show that, if one has a choice,
intervals based on medians and quantiles should be preferred to
the more classical mean and standard deviation. We also give
useful, though little know results such as how to compute a
confidence interval for a success probability when has seen no
success.

\minitoc

\section{Summarized Performance Data}
\mylabel{sec-conf-sum} \subsection{Histogram and Empirical CDF}
Assume you have obtained a large set of results for the value
of a performance metric. This can be fully described by the
distribution of the data, and illustrated by a \nt{histogram}.
A histogram uses bins for the data values and plots on the
$y$-axis the proportion of data samples that fall in the bin on
the $x$ axis, see \fref{fig-conf-c1}.

\begin{figure}[htbp]
\begin{center}\nfs{sgbd}
\subfigure[Raw data, old]{\Ifignc{conf-raw-old}{0.23}{0.3}}
\subfigure[Raw data, new]{\Ifignc{conf-raw-new}{0.23}{0.3}}
\subfigure[Histogram, old]{\Ifignc{conf-hist-old}{0.23}{0.3}}
\subfigure[Histogram, new]{\Ifignc{conf-hist-new}{0.23}{0.3}}
  \end{center}
  \mycaption{Data for \exref{ex-conf-1}. Measured execution times,
   in ms, for 100
transactions with the old and new
code, with histograms.}\mylabel{fig-conf-c1}
\end{figure}


The \nt{empirical cumulative distribution function} (ECDF) is
an alternative to histograms which sometimes makes comparisons
easier. The ECDF of a data set $x_1, ...,x_n$ is the function
$F$ defined by
 \be
 F(x) = \frac{1}{n}\sum_{i=1}^n \ind{x_i \leq x}
  \ee
  so that $F(x)$ is the proportion of data samples that do not
  exceed $x$.

Data sets may be compared by means of their ECDFs. If one is
always above the other, then one may consider that it is
superior, even though some data points in the first set may be
less good (this is called \nt{stochastic majorization}). On
\fref{fig-conf-c1cdf} we see that the new data
  set (left)
  clearly outperforms the old one. Note that stochastic majorization is a
partial order, as is the comparison of multidimensional metrics
(\sref{metric}).

Assume the data samples come from
  a well defined probability distribution; the histogram can
  then be viewed as an estimate of the PDF of the distribution,
  and the ECDF as an estimate of the CDF\footnote{\index{CDF}The
  CDF of the random variable $X$ is the function defined by
 $F(x)=\P(X \leq x)$.}.

  \begin{figure}[htbp]
  \Ifig{sgbdCdf}{0.8}{0.3}\nfs{sgbdold.dat, sgbdnew.dat,
  sgbdCdf.m}
  \mycaption{Data of \exref{ex-conf-1}. Empirical
distribution functions for the old code (right
curve) and the new one (left curve). The new
outperforms the old, the improvement is
significant at the tail of the
distribution.}\mylabel{fig-conf-c1cdf}
\end{figure}



\subsection{Mean, Median and Quantiles}
Instead of considering entire histograms or ECDFs, one often
would like to summarize, i.e. compress the histogram into one
or a few numbers that represent both average and variability.
This is commonly done by either one of the following:

\imp{Median and Quantile}. A median is a value
that falls in
    the middle of the distribution, i.e. 50\% of the data is below
    and 50\% above.
A $p$\%-quantile leaves $p$\% of the observation below and
$(100-p)$\% above.
    The median gives some information about the average, while extreme
quantiles give information about the dispersion. A commonly
used plot is the \nt{Box Plot}. It shows the median, the 25\%
and 75\% quantiles (called ``quartiles") and the ``outliers",
defined as data points that are a fixed fraction away from the
quartiles (\fref{fig-conf-c1bp}).
\begin{petit}
The \nt{sample median} of a data set is defined
as follows. Assume there are $n$ data points
$x_1,..., x_n$. Sort the points in increasing
order and obtain $x_{(1)}\leq ...\leq x_{(n)}$.
If $n$ is odd, the median is
$x_{(\frac{n+1}{2})}$, else
$\frac{1}{2}(x_{(\frac{n}{2})}+x_{(\frac{n}{2}+1)})$.
More generally, the \nt{sample $q$- quantile} is
defined as $\frac{x_{(k')}+x_{(k")}}{2}$ with
$k'=\lfloor qn+(1-q)\rfloor$ and $k'=\lceil
qn+(1-q)\rceil$. $\lfloor x \rfloor$ is the
largest integer $\leq x$ and $\lceil x \rceil$ is
the smallest integer $\geq x$ \index{1@$\lfloor x
\rfloor$}\index{1@$\lceil x \rceil$}
\end{petit}

\imp{Mean and Standard Deviation}. The \nt{mean} $m$ of a data
set $x_1,..., x_n$ is $m=\frac{1}{n}\sum_{i=1}^n x_i$. It gives
some information about the center of the distribution. The
\nt{standard deviation $s$ of a data set} is defined by
$s^2=\frac{1}{n}\sum_{i=1}^n \left(x_i- m\right)^2$ or
$s^2=\frac{1}{n-1}\sum_{i=1}^n \left(x_i- m\right)^2$ (either
conventions are used -- see \sref{sec-conf-cimean} for an
explanation). It gives information about the variability. The
use of standard deviation is rooted in the belief that data
roughly follows a \nt{normal distribution}\label{def-normal},
also called \nt{gaussian distribution}. It is characterized by
a histogram with Bell shape (see wikipedia and
\tref{tab-distrib1} on \pgref{tab-distrib1}); the CDF of the
general normal distribution is denoted with $N_{\mu,\sigma^2}$,
where $\mu$ is the mean and $\sigma^2$ the variance. It is very
frequently encountered because of the central limit theorem
that says that an average of many things tends to be normal
(but there are some exceptions, called heavy tail in
\cref{ch-modfit}). If such a hypothesis is true, and if we had
$m\approx\mu$ and $\sigma\approx s$, then with 95\%
probability, the data sample would lie in the interval $m \pm
1.96 s$. This justifies the use of \nt{mean-variance} plots
like in \fref{fig-conf-c1bp}, which use as measure of
variability the interval $m \pm 1.96 s$. This is also called a
\nt{prediction interval} since it predicts a likely range for a
future sample (\sref{sec-conf-pi}).

\begin{figure}[htbp]
\nfs{sgbd}
\Ifig{conf-boxplots}{0.95}{0.45}
\mycaption{Box Plots for the data for \exref{ex-conf-1}.
Left: standard box plot commonly used by statisticians showing
median (notch) and
quartiles (top and bottom of boxes);
``dispersion" is an ad-hoc measure, defined here
as 1.5 times the inter-quartile distance; the
notch width shows the confidence interval for the median.
Right: same, overlaid with quantities commonly used in signal processing:
mean, confidence interval
for the mean ($=\mbox{ mean }\pm 1.96 \sigma / \sqrt{n}$, where $\sigma$
is the standard deviation and $n$ is the number of samples) and
prediction interval ($=\mbox{ mean }\pm 1.96 \sigma$).}
\mylabel{fig-conf-c1bp}
\end{figure}


\begin{ex}{Comparison of Two Options}\mylabel{ex-conf-1}
An operating system vendor claims that the new
version of the database management code
significantly improves the performance. We
measured the execution times of a series of
commonly used programs with both options. The
data are displayed in \fref{fig-conf-c1}. The raw
displays and histograms show that both options
have the same range, but it seems (graphically)
that the new system more often provides a smaller
execution time. The box plots are more
suggestive; they show that the average and the
range are about half for the new system.
\end{ex}


In \sref{sec-conf-mom} we discuss the differences
between these two modes of summarization.



\subsection{Coefficient of Variation and Lorenz Curve Gap}
\label{sec-var} Those are frequently used
measures of variation, rescaled to be invariant
by change of scale. They apply to a positive data
set $x_1, ..., x_n$.

\paragraph{Coefficient of Variation. }
It\index{Coefficient of Variation}\index{CoV} is
defined by
  \be
  \mbox{CoV} = \frac{s}{m}  \label{eq-cov}
  \ee
where $m$ is the mean and $s$ the standard
deviation, i.e. it is the standard deviation
rescaled by the mean. It is also sometimes called
\nt{Signal to Noise ratio}. For a data set
with $n$ values one always has%
\footnote{Consider the maximization problem:
maximize $\sum_i (x_i-m)^2$ subject to $x_i \geq
0$ and $\sum x_i= mn$. Since $x \mapsto \sum_i
(x_i-m)^2$ is convex, the maximum is at an
extremal point $x_{i_0}=mn,$ $x_i= 0$, $i\neq
i_0$.}
 \be
 0 \leq \mbox{CoV} \leq \sqrt{n-1}
 \label{eq-cv-bounds}
 \ee
where the upper bound is obtained when all $x_i$ have the same
value except one of them. The lower bound is reached when all
values are equal.

\paragraph{Lorenz Curve Gap.} It is an alternative measure of
dispersion, obtained when we replace the standard
deviation by the \nt{Mean Absolute Deviation}
(\nt{MAD}). The MAD is defined by
 \ben
 \mbox{MAD} = \frac{1}{n}\sum_{i=1}^n \abs{x_i-m}
 \een
 i.e. we compute the mean distance to the mean, instead of the
 square root of the mean square distance. Compared to the
 standard deviation, the MAD is less sensitive to a few very
 large values. It follows from the Cauchy-Schwarz inequality that it is always less than the
 standard deviation, i.e.
 \be
 0 \leq \mbox{MAD} \leq s
 \ee
with equality only if $x_i$ is constant, i.e.
$x_i=m$ for all $i$.

If $n$ is large and $x_i$ is iid from a gaussian distribution,
then
 \be
 \mbox{MAD} \approx \sqrt{\frac{2}{\pi}}s \approx 0.8
 s
 \ee
If in contrast, if $x_i$ comes from a heavy
tailed distribution with a finite mean $m$, then
$s\to {\infty}$ as $n$ gets large, whereas
$\mbox{MAD}$ converges to a finite limit.

The \nt{Lorenz Curve Gap}\index{gap} is a
rescaled version of MAD, defined by
 \be
 \mbox{ gap } = \frac{\mbox{MAD}}{2 m}
 \label{eq-def-gap}
 \ee
The reason for the factor $2$ is given in the
next section. We always have
  \be 0 \leq \mbox{ gap } \leq 1-\frac{1}{n}
  \ee
thus, contrary to $\mbox{CoV}$, $\mbox{ gap }$ is between $0$
and $1$. If $n$ is large and $x_i$ is iid from a gaussian
distribution, then $\mbox{ gap } \approx 0.4 \mbox{CoV}$; if it
comes from an exponential distribution, $\mbox{ gap } \approx
0.37$ and $\mbox{CoV}\approx 1$.

\begin{petit}
If $x_i$ is iid and comes from a distribution with PDF $f()$,
then, for large $n$, $\mbox{CoV}$ and $\mbox{MAD}$ converge to
their theoretical counterparts:
 \bearn
 \mbox{CoV} & \to &  \mbox{CoV}_{\mbox{th}} =
    \frac{\sqrt{\int_0^{\infty}(x-\mu)^2 f(x) dx}}{\mu}\\
 \mbox{MAD} & \to & \mbox{ gap }_{\mbox{th}}=
    \frac{\int_0^{\infty}\abs{x-\mu} f(x) dx}{2 \mu}
 \eearn
 with $\mu=\int_0^{\infty}x f(x) dx$.

If the distribution is gaussian $N_{\mu,\sigma^2}$ then
$\mbox{CoV}_{\mbox{th}}=\frac{\sigma}{\mu}$ and $\mbox{ gap
}_{\mbox{th}}=\sqrt{1 \over 2\pi}\frac{\sigma}{\mu}$; if it is
exponential then $\mbox{CoV}_{\mbox{th}}=1$ and $\mbox{ gap
}_{\mbox{th}}=\frac{1}{e}$.
\end{petit}
%
%\paragraph{Gini's Index}
%Define \nt{Mean Difference}
%  \be
%  \mbox{MD} = \frac{1}{n(n-1)}\sum_{i, j}\abs{x_i-x_j}
%  \ee
%\be
%  \mbox{Gini} = \frac{\mbox{MD}}{2 \mu}
%  \label{eq-def-gini}
%  \ee
\subsection{Fairness Indices}
Often one interprets variability as fairness, and several
fairness indices have been proposed. We review here the two
most prominent ones. We also show that they are in fact
reformulations of variability measures, i.e. they are
equivalent to $\mbox{CoV}$ and $\mbox{ gap }$, after proper
mapping (so that using these indices may appear superfluous).
Like in the previous section, the data set $x_i$ is assumed
here to be positive.

\paragraph{Jain's Fairness Index (JFI).}
\index{Jain's Fairness Index}\index{JFI}It is
defined as the square of the cosine of the angle
between the data set $x_i$ and the hypothetical
equal allocation (\fref{fig-jfi}). It is given by
 \be
 \mbox{JFI} =
 \frac{\lp\sum_{i=1}^n x_i\rp^2}{n \sum_{i=1}^n x_i^2}
\label{eq-jfi}
  \ee
A straightforward computation shows that the
fairness
 measure $\mbox{JFI}$ is a decreasing function of the
 variability measure $\mbox{CoV}$:
 \be
\mbox{JFI} = \frac{1}{1 + \mbox{CoV}^2}
 \ee
 so that, by \eref{eq-cv-bounds}, we conclude that
 $\mbox{JFI}$ ranges from $\frac{1}{n}$ (maximum
 unfairness) to $1$ (all $x_i$ are equal).

 \begin{figure}
 \insfig{jfi}{0.45}
 \mycaption{Jain's fairness index is $\cos^2 \theta$. $x_1, ..., x_n$
 is the data set and $m$ is the sample mean. The figure is for $n=2$.}
 \label{fig-jfi}
 \end{figure}

\paragraph{Lorenz Curve}
The \nt{Lorenz Curve} is defined as follows. A
point $(p,\ell)$ on the curve, with $p, \ell \in
[0,1]$, means that the bottom fraction $p$ of the
distribution contributes to a fraction $\ell$ of
the total $\sum_{i=1}^n x_i$.

More precisely, we are given a data set $x_i >0$,
$i=1...n$. We plot for all $i=1...n$ the points
$(p_i, \ell_i)$ with
 \be
 \bracket{
 p_i =  \frac{i}{n}
 \\
 \ell_i  =  \frac{\sum_{j=1}^n x_j\ind{x_j \leq x_i}}{\sum_{j=1}^n x_j}
 }\ee
See \fref{fig-lorenz} for examples. We can make
the Lorenz curve a continuous mapping $\ell=L(p)$
by linear interpolation and by setting $L(0)= 0$.
The resulting $L()$ is a continuous mapping from
$[0,1]$ onto $[0,1]$, monotone non decreasing,
convex, with $L(0)= 0$ and $L(1)=1$.

The Lorenz curve $\ell=L(p)$ can be interpreted
as a global measure of fairness (or variability).
If all $x_i$s are equal (maximum fairness) then
$L(p)= p$ and $L()$ is the diagonal of the square
$[0,1]\times [0,1]$ (called the ``line of perfect
equality"). In the worst case, the Lorenz curve
follows the bottom and right edges of the square
(called the ``line of perfect inequality")
(\fref{fig-gini}). In practice the Lorenz curve
is computed by sorting $x_i$ in increasing order
($x_{(1)} \leq x_{(2)}\leq \ldots \leq x_{(n)}$)
and letting
 \be
l_i=\frac{x_{(1)} + ...+ x_{(i)} }{nm}
\label{eq-calcul-lorenz}\ee where $m$ is the
sample mean. It follows that $0 \leq l_i \leq
\frac{i}{n}$, i.e.
  \ben
0\leq L(p) \leq p
  \een
i.e. and the Lorenz curve is always between the
lines of perfect equality and perfect inequality.

\begin{figure}\begin{center}
 \subfigure[Execution times in \fref{fig-conf-c1}, old code]
 {\Ifignc{lorenzsgbdold}{0.45}{0.45}}
 \subfigure[Execution times in \fref{fig-conf-c1}, new code]
 {\Ifignc{lorenzsgbdnew}{0.45}{0.45}}
 \subfigure[Ethernet Byte Counts ($x_n$ is the byte length of the $n$th packet of
 an Ethernet trace \cite{leland1994self})]
 {\Ifignc{lorenzbyte}{0.45}{0.45}}
 \begin{tabular}{|c||c|c|c|c|c|}
   \hline
   % after \\: \hline or \cline{col1-col2} \cline{col3-col4} ...
     %& \fref{fig-conf-c1}, old code & \fref{fig-conf-c1},
%     new code & Ethernet Byte Counts  \\ \hline
%   \hline
%   CoV & 0.779 & 0.720 & 1.84 \\
%   JFI & 0.622 & 0.658 & 0.228 \\
%   \hline
    & CoV & JFI & gap & Gini & Gini-approx\\ \hline \hline
     \fref{fig-conf-c1}, old code &
     0.779  & 0.622  & 0.321 & 0.434&0.430
     \\ \hline
     \fref{fig-conf-c1},
     new code &
     0.720   & 0.658 & 0.275 &0.386 & 0.375
     \\  \hline
     Ethernet Byte Counts
     & 1.84
     &   0.228
     & 0.594 &
     0.730&0.715
     \\
     \hline
 \end{tabular}


\end{center}
  \mycaption{Lorenz curves for three data sets, with proportion of users $p$ on $x$ axis and proportion of total sum $\ell$ on $y$ axis. The diagonal is the line of perfect
  equality. The maximum distance (plain line) is equal to $1\over \sqrt{2}$ times
  the maximum vertical deviation (dashed line), which is called the Lorenz curve gap.
  The Gini coefficient is the area between the diagonal and the Lorenz curve,
  rescaled by its maximum
  value $\frac{1}{2}$. The table gives the values of Coefficient of
  Variation, Jain's Fairness Index, Lorenz Curve Gap, Gini coefficient
  and the Gini coefficient approximation in \eref{eq-gini-approx}.}
  \label{fig-lorenz}
\end{figure}



\paragraph{Lorez Curve Gap, Again} A measure of fairness is the largest euclidian
distance (the gap) from the Lorenz curve to the
diagonal, rescaled by its maximum value
$\left(1\over \sqrt{2}\right)$. It is also equal to the largest
vertical distance, $\sup_{u \in [0,1]}\lp u- L(u)
\rp$ (\fref{fig-lorenz}). The gap can easily be
computed by observing that it is reached at index
$ i_0 = \max\{i: x_{(i)}\leq m\} $, i.e. at a
value $p_0=\frac{i_0}{n}$ such that the bottom
fraction $p_0$ of the data have a a value less
than the average. Thus
 \be
 \mbox{ gap } = \frac{i_0}{n}-\frac{x_{(1)}+...+x_{(i_0)}}{m n}
 \ee

We have already introduced the gap in
\eref{eq-def-gap}, so we need to show that the
two definitions are equivalent. This follows from
 \bearn
  \mbox{MAD} &=&  \frac{1}{n}\sum_{i=1}^n \abs{x_i-m} =    \frac{1}{n}\sum_{i=1}^n \abs{x_{(i)}-m}
 \\
 &= & \frac{1}{n}\lp \sum_{i=1}^{i_0}(m-x_{(i)})+ \sum_{i_0+1}^n(x_{(i)}-m)\rp\\
 &=&   \frac{1}{n}\lp i_0 m-\sum_{i=1}^{i_0} x_{(i)}
 +nm-\sum_{i=1}^{i_0}x_{(i)}-(n-i_0) m\rp
 \\
 & = & 2 m \;\mbox{ gap }
 \eearn which is the same as \eref{eq-def-gap}.

 \begin{petit}
 The theoretical Lorenz curve is defined for a probability distribution with
 cumulative distribution function  CDF
 $F()$ and finite mean $\mu$ by
  \be L(p) = \frac{1}{\mu}\int_0^p F^{-1}(q) dq
  \label{eq-def-l-th}\ee where $F^{-1}$ is the (right-continuous)
  pseudo-inverse
 \ben
 F^{-1}(q) = \sup \{x: F(x) \leq p\}=\inf\{x: F(x) > q\}
 \een
 If the CDF $F()$ is continuous and increasing,
 then $F^{-1}$ is the usual function inverse. In this case,
 the theoretical Lorenz curve gap is then equal to \ben
\mbox{gap}_{\mbox{th}}=p_0-L(p_0)\een with $p_0=F(\mu)$.


The theoretical Lorenz curve is the limit of the Lorenz curve
for an iid data sample coming from $F()$, when $n$ is large.
 \end{petit}

\paragraph{The Gini Coefficient}
\nt{Gini coefficient} This is yet another fairness index, very
widespread in economy, and, by imitation, in computer and
communication systems. Its definition is similar to the Lorenz
curve gap, with the mean average deviation replaced by the
\nt{Mean Difference}:
  \be
  \mbox{MD} = \frac{1}{n(n-1)}\sum_{i, j}\abs{x_i-x_j}
  \ee
The Gini coefficient is then defined as
  \be
  \mbox{Gini} = \frac{\mbox{MD}}{2 m}
  \label{eq-def-gini}
  \ee where $m$ is the empirical mean of the data set. It can
  be shown that it is equal to $2 \times $ the area between the line of perfect
equality and the Lorenz curve (the rescaling factor $2$ makes
it lie between $0$ and $1$). In practice the Gini coefficient
can be computed by using \eref{eq-calcul-lorenz}, which gives
 \bear
\mbox{Gini} &= & \frac{2}{m n^2}\sum_{i=1}^n i
x_{(i)}-1 - \frac{1}{n}
 \label{eq-gini-comp}
 \eear
%
 \begin{petit}
The theoretical Gini coefficient for a probability distribution
with CDF $F()$ is defined by
 \ben
 \mbox{Gini}_{\mbox{th}} = 2\int_0^{1} \lp q - L(q)\rp dq
 =1- 2\int_0^{1} L(q) dq
 \een where $L()$ is the theoretical Lorenz curve defined in \eref{eq-def-l-th}.
 \end{petit}

\begin{figure}
 \insfig{gini}{0.45}
 \mycaption{Lorenz curve (plain line). The line of perfect equality is $OD$, of perfect
 inequality $OBD$. The Lorenz curve gap is the maximum distance to the line of
 perfect equality, re-scaled by $\sqrt{2}$. The Gini coefficient is the area between the line of perfect
 equality and the Lorenz curve, re-scaled by 2.}
 \label{fig-gini}
 \end{figure}


Since the Lorenz curve is convex, it is straightforward to
bound the Gini coefficient by means of the Lorenz curve gap. On
\fref{fig-gini}, we see that the area between the Lorenz curve
and the diagonal is lower bounded by the triangle $OM_0D$ and
upper bounded by the trapeze $OACD$. It follows from this and
\eref{eq-gini-comp} that
 \bearn
 0 \leq \mbox{ gap } \leq &\mbox{Gini}&\leq 1-\frac{1}{n}\\
 &\mbox{Gini}&\leq \mbox{ gap }(2-\mbox{ gap })
 \eearn where the lower bound $0$ is reached at maximum fairness.

It follows that one can also approximate
$\mbox{Gini}$ by the arithmetic mean of the lower
and upper bounds:
  \be
   \mbox{Gini} \approx \mbox{ gap } (1.5 -0.5 \mbox{ gap })
  \label{eq-gini-approx}
  \ee


\paragraph{Summary}
Since there are so many different variability and
fairness indices, we give here a summary with
some recommendations.

First, since the Gini coefficient can be essentially predicted
from the Lorenz curve gap, we do not use it further in this
book. However, it may be useful to know the relationship
between the two since you may find that it is used in some
performance evaluation results.

Second, Jain's fairness index and the Lorenz curve gap are
fundamentally different and cannot be mapped to each other. The
former is essentially the same as the standard deviation or the
coefficient of variation. If the data comes from a heavy tailed
distribution, the theoretical coefficient of variation is
infinite, and $\mbox{CoV}\to \infty$ as the number of data
points gets large. Comparing different CoVs in such a case does
not bring much information. In contrast, the Lorenz curve gap
continues to be defined, as long as the distribution has a
finite mean. It should be preferred, if one has a choice.

We recall the main inequalities and bounds in
\tref{tab-fairness-indices} on \pgref{tab-fairness-indices}.
See also \fref{fig-lorenz} for some examples.

\begin{table}
  \centering
  \begin{tabular}{|c||c|c|c|}
  \hline
    % after \\: \hline or \cline{col1-col2} \cline{col3-col4} ...
      &
      Jain's Fairness Index&
       Lorenz Curve Gap&
       Gini Coefficient
        \\
        & (JFI) & (gap) & (Gini)\\
        \hline \hline
    Definition &
     $\frac{1}{1+ \mbox{ CoV }^2} $ &
        $\frac{\mbox{ MAD }}{2 m}$&
        $\frac{\mbox{ MD }}{2 m}$
       \\
& \eref{eq-cov}, \eref{eq-jfi}
& \eref{eq-def-gap}
&\eref{eq-def-gini}
               \\
       \hline
    Bounds&
       $\frac{1}{n}\leq \mbox{ JFI } \leq 1$ (MF)&
       (MF) $0\leq \mbox{ gap } \leq 1-\frac{1}{n}$ &
       (MF) $0\leq \mbox{ Gini } \leq 1-\frac{1}{n}$
       \\
       \hline
    Relations &
    $ \frac{1}{1+4\mbox{ gap }^2} \leq  \mbox { JFI }
    $&
    &
    $\mbox { gap } \leq \mbox{ Gini } \leq  \mbox { gap }
    ( 2- \mbox { gap })$
     \\
     &
     Equality only at MF&
     &
 $\mbox{Gini} \approx \mbox{ gap } (1.5 -0.5 \mbox{ gap })$
 \\
       \hline
    %$N_{\mu, \sigma^2}$ &
%    $\frac{1}{1+\sigma^2/\mu^2}$&
%      $\sqrt{2 \over \pi}\frac{\sigma}{\mu} \approx 0.798 \frac{\sigma}{\mu} $ &
%       TBD
%       \\ \hline
   Exp$(\lambda)$, $\la >0$&
     $0.5$ &
      $\frac{1}{e}\approx 0.368$ &
       $0.5$
       \\ \hline
       Unif$(a,b)$
       &
       $\frac{1}{1+\frac{(b-a)^2}{3(b+a)^2}}$
       &
       $\frac{b-a}{4(a+b)}$
       &
       $\frac{b-a}{3(a+b)}$
       \\
        $0\leq a <b$&&&\\ \hline
    Pareto ($p,x_0)$&
     $ \frac{p(p-2)}{(p-1)^2}
     $
      &
      $\frac{1}{p}\lp 1  - \frac{1}{p} \rp^{p-1}$&
       $\frac{1}{2 p -1}$
       \\
        $x_0 >0, p>1$&for $p>2$& & \\\hline
 \end{tabular}
  \mycaption{Relationships between different fairness indices of a data set
  with $n$ samples and empirical mean $m$ (MF = value when fairness is maximum, i.e
  all data points are equal).}\label{tab-fairness-indices}
\end{table}
\nfs{$gini=\frac{2}{m}\int_{-\infty}^{+\infty}x
f(x) F(x) dx -1$}

%\subsection{Receiver Operating Characteristics (ROC), Recall and Precision}
%(type I, II errors; MAP vs MLE; recall vs precision)
%
%
%\subsection{Summary}
%\bearn
% \mbox{AD} &=& 2 \lp \mu F(\mu) - \int_0^{\mu} x f(x) dx\rp
% \\
% \mbox{gaussian: } AD &=& \sqrt{2 \over \pi} \sigma \approx 0.80
% \sigma
% \\
% \mbox{ gap } &=& \frac{\mbox{AD}}{2  \mu} \mbox{rescaled to be in
% }[0,1]
% \\
% \mbox{maxgap}& = &\lp 1-{1 \over n}\rp
% \\
% u_{\mbox{ gap }}&=&F^{-1}(\mu)
% \\
% \mbox{gini index} &=& 2 \times \mbox{area between curve and diagonal}
% \\
% \mbox{Jain's fairness index}&=& \frac{1}{1 + \mbox{CoV}^2} \mbox{ ranges from } {1 \over
% n} \mbox{ to } 1
% \\
% \mbox{JYLB fairness index} &=& 1- \mbox{ gap } \mbox{ ranges from } {1 \over
% n} \mbox{ to } 1
% \eearn
%
% Interpolation by a quadratic function on each of the two intervals,
% with same slope at $f_0$ gives the approximation for Gini index
% ($f_0=F(\mu)$):
%
% \bearn
% \mbox{gini} &\approx& 2 (A_1 + A_2)
% \\
% A_1 & = & \frac{2}{3} f_0 \times \mbox{ gap }
% \\
% A_2 & = & \frac{2 \mbox{ gap }}{3}\lp 1-f_0 \rp
% \\
%\mbox{gini} &\approx&\frac{4}{3} \mbox{ gap }
% \eearn
%If data is gaussian
%\be
% \mbox{ gap } = \frac{1}{\sqrt{2 \pi}} \mbox{CoV}
%\ee%
%
%\section{Graphics}
%Pitfalls, voir Jain +
%
%\begin{enumerate}
%    \item ne pas mettre index en axe des x (avoid unnecessary graphical
%    elements)
%    \item watch the scale - put 0
%    \item misleading elements
%\end{enumerate}
%
%Display RoC
%
%Voir le livre recommend\'{e} par Vetterli.

\section{Confidence Intervals}
\mylabel{sec-conf-cimean}
\subsection{What is a Confidence Interval~?}
When we display a number such as the median or
the mean of a series of performance results, it
is important to quantify their accuracy (this is
part of the scientific method, \cref{ch-metho}).
\nt{Confidence intervals} quantify the
uncertainty about a summarized data that is due
to the randomness of the measurements.

\begin{ex}{Comparison of Two Options, continued} \mylabel{ex-conf-c2}%
We wish to quantify the improvement due to the
new system. To this end, we measure the reduction
in run time for the same sequence of tasks as on
\fref{fig-conf-c1} (both data sets on
\fref{fig-conf-c1} come from the same transaction
sequences -- statisticians say that this is a
\nt{paired experiment}). The differences are
displayed in \fref{fig-conf-c2}.

The last panel shows confidence intervals for the
mean (horizontal lines) and for the median
(notches in Box plot). For example, the mean of
the reduction in run time is $
\input{ex2.dat}$.%
\nfs{sgbd.m}%
The uncertainty margin is called the confidence interval for
the mean. It is obtained by the method explained in this
section. Here, the mean reduction is non negligible, but the
uncertainty about it is large.
\end{ex}
\begin{figure}
\Ifig{conf-bp2}{0.9}{0.3}%
\nfs{sgbdnew.dat, sgbdold.dat, sgbdplot.m}%
\mycaption{Data for \exref{ex-conf-c2}: reduction in
run time (in ms). Right: Box plot with mean and confidence
interval for mean.} \mylabel{fig-conf-c2}
\end{figure}

%\fref{fig-conf-c1} and \fref{fig-conf-c2} Note
%that the confidence interval is \imp{not the same
%as a measure of variability}, though it is
%related, as we discuss in \sref{sec-conf-mom}: on
%\fref{fig-conf-c1} the confidence interval for
%the mean is considerably smaller than the
%variability interval given by $m \pm 1.96 s$.
%There is a confidence interval for each of the
%summarized data given earlier: median, quantile,
%mean and standard deviation.

There is a confidence interval for every
summarized quantity: median, mean, quartile,
standard deviation, fairness index, etc. In the
rest of this section, we explain \emph{how} to
compute confidence intervals.

\subsection{Confidence Interval for Median and Other Quantiles}
We start with the median and other quantiles, as
it is both simplest and most robust; this section
also serves as an illustration of the general
method for computing confidence intervals.

%The confidence interval for the median is shown
%by notches on Box plots (\fref{fig-conf-c1}, (3)
%on Box plot). We start with the median and then
%extend it to other quantiles.

The main idea (which underlies all classical statistics
formulae) is to imagine that the data we have measured was in
fact generated by a simulator, whose program is unknown to us.
More precisely, we are given some data $x_1, ..., x_n$; we
imagine that there is a well defined probability distribution
with CDF $F()$ from which the data is sampled, i.e. we have
received one sample from a sequence of independent and
identically distributed (\nt{iid}) random variables $X_1, ...,
X_n$, each with common CDF $F()$. The assumption that the
random variables are iid is capital; if it does not hold, the
confidence intervals are wrong. We defer to \sref{sec-iid-cose}
a discussion of when we may or may not make this assumption.
For now we assume it holds.

The distribution $F()$ is non-random but is
unknown to us. It has a well defined median $m$,
defined by : for every $i$, $\P(X_i \leq m)=0.5$.
We can never know $m$ exactly, but we
\imp{estimate} it by $\hat{m}(x_1, ...,x_n)$,
equal to the sample median defined in
\sref{sec-conf-sum}. Note that the value of the
estimated median depends on the data, so it is
random: for different measurements, we obtain
different estimated medians. The goal of a
confidence interval is to bound this uncertainty.
It is defined relative to a \nt{confidence level}
$\gamma$; typically $\gamma=0.95$ or $0.99$:
\begin{definition}
A \imp{confidence interval} at level $\gamma$
for the fixed but unknown parameter $m$ is an
interval
$\left(u(X_1,...,X_n),v(X_1,..,X_n)\right)$ such
that
 \be \P\left(u(X_1,...,X_n)< m < v(X_1,...,X_n)\right) \geq \gamma
 \ee
\end{definition}
In other words, the interval is constructed from
the data, such that with at least $95\%$
probability (for $\gamma=0.95$) the true value of
$m$ falls in it. Note that \imp{it is the
confidence interval that is random, not the
unknown parameter $m$}.

A confidence interval for the median or any other
quantile is very simple to compute, as the next
theorem shows.
\begin{shadethm}[Confidence Interval for Median and Other Quantiles]
\label{theo-conf-ci-median} Let $X_1, ...,X_n$ be $n$ iid
random variables, with a common CDF $F()$. Assume that $F()$
has a density, and for $0<p<1$ let $m_p$\index{1mp@$m_p$ :
$p$-quantile} be a $p$-quantile of $F()$, i.e. $F(m_p)=p$.

 Let
$X_{(1)}\leq X_{(2)}\leq ...\leq X_{(n)}$ be the \nt{order
statistic}, i.e. the set of values of $X_i$ sorted in
increasing order. Let $B_{n,p}$ be the CDF of the binomial
distribution with $n$ repetitions and probability of success
$p$. A confidence interval for $m_p$ at level $\gamma$ is
\ben[X_{(j)},X_{(k)}]\een where $j$ and $k$ satisfy \ben
B_{n,p}(k-1)-B_{n,p}(j-1) \geq \gamma\een See the tables in
Appendix \ref{sec-ci} on \pgref{sec-ci} for practical values.
For large $n$, we can use the approximation
  \bearn
  j & \approx & \lfloor n p - \eta \sqrt{n p
  (1-p)} \rfloor\\
  k & \approx & \lceil n p + \eta \sqrt{n p
  (1-p)} \rceil +1
  \eearn
  where $\eta$ is defined by
  $N_{0,1}(\eta)=\frac{1+\gamma}{2}$ (e.g.
  $\eta=1.96$ for $\gamma=0.95$).
\end{shadethm}
The \nt{Binomial distribution} \index{$B_{n,p}$} $B_{n,p}$,
with $n$ repetitions and probability of success $p$, is the
distribution of $Z=\sum_{i=1}^n Z_i$ where $Z_i$ are iid random
variables such that$Z_i=0$ or $1$ and $\P(Z_i=1)=p$, i.e. it is
the distribution of the number of successes in an experiment
with $n$ trials and individual success probability $p$. (The
random variables $Z_i$ are called \nt{Bernoulli} random
variables. $N_{0,1}$ is the CDF of the gaussian distribution
with mean $0$ and variance $1$.)

For $n=10$, the theorem and the table in
\sref{sec-ci} say that a $95\%$-confidence
interval for the median (estimated as
$\frac{X_{(5)}+X_{(6)}}{2}$) is
$\left[X_{(2)},X_{(9)}\right]$. In other words,
we obtain a confidence interval for the median of
10 results by removing the smallest and the
largest. Could it be simpler~?

Note that, for small values of $n$, no confidence
interval is possible at the levels $0.95$ or
$0.99$. This is due to the probability that
the true quantile is outside any of the observed
data still being large.

For large $n$, the binomial distribution can be
approximated by a gaussian distribution, which
explains the approximation in the theorem.


\begin{petit}
The assumption that the distribution has a
density (also called PDF\index{PDF}, probability
density function) is for simplicity of
exposition. If $F()$ does not have a density
(e.g. because the numbers $X_i$ are integers) the
theorem hold with the modification that the
confidence interval is $[X_{(j)},X_{(k)})$
(instead of $[X_{(j)},X_{(k)}]$).
\end{petit}



%\begin{table}
%\footnotesize
%  \centering
%  \begin{tabular}{|c|c|c|c|c|c|c|c|c|c|c| } \hline 0 & 1 & 2 & 3 & 4 & 5 & 6 & 7 &
%8 & 9 & 10 \\ \hline
% 0.001 & 0.011 & 0.055 & 0.172 & 0.377 & 0.623 & 0.828 & 0.945 & 0.989 & 0.999 & 1.000 \\ \hline
% \end{tabular}
%  \mycaption{CDF of Binomial distribution $B_{n=10,p=0.5}$}\label{tab-conf-bin}
%\end{table}

\subsection{Confidence Interval for the Mean}
\label{sec-ci-mean} Here too there is a widely used result,
given in the next theorem. The proof is standard and can be
found in probability textbooks
\cite{grimmett2001probability,papoulis1965probability}.

\begin{shadethm}
\mylabel{theo-conf-ln} Let $X_1, ...,X_n$ be $n$ iid random
variables, the common distribution of which is assumed to have
well defined mean $\mu$ and a variance $\sigma^2$. Let
$\hat{\mu}_n$ and $s_n^2$ by
 \bear
 \hat{\mu}_n&= & \frac{1}{n}\sum_{i=1}^n X_i \mylabel{eq-conf-2-mu}\\
 s_n^2& =&\frac{1}{n}\sum_{i=1}^n \left(X_i-
 \hat{\mu}_n\right)^2
 \mylabel{eq-conf-2-sigma}
 \eear
The distribution of
 $\sqrt{n} \frac{\hat{\mu}_n-\mu}{s_n}$
converges to the normal distribution $N_{0,1}$
when $n\rightarrow +\infty$. An approximate
confidence interval for the mean at level
    $\gamma$ is
    \be
    \hat{\mu}_n\pm \eta \frac{s_n}{\sqrt{n}}
    \label{eq-ci-m}
    \ee
    where $\eta$ is the
$\frac{1+\gamma}{2}$ quantile of the normal
distribution $N_{0,1}$, i.e
$N_{0,1}(\eta)=\frac{1+\gamma}{2}$. For example,
$\eta=1.96$ for $\gamma=0.95$ and $\eta=2.58$ for
$\gamma=0.99$.
\end{shadethm}

Note that the amplitudes of the confidence
interval decreases like $\frac{1}{\sqrt{n}}$.

Also note however that some caution may be
required when using the theorem, as it makes 3
assumptions: \noitemsep
 \begin{enumerate}
    \item the data comes from an iid sequence
    \item the common distribution has a finite
    variance
    \item the number of samples is large
 \end{enumerate} each of these assumptions is worth screening, as there are
 realistic cases where they do not hold. Assumption 1 is the same as for all confidence intervals in this chapter, and
is discussed in \sref{sec-iid-cose}. Assumption 2 is true
unless the distribution is heavy tailed, see \sref{sec-ht}.
Assumption 3 is usually true even for small values of $n$, and
can be verified using the method in
\sref{sec-verify-asymptotic}.

%\begin{petit}
\subsubsection{Normal iid Case} \mylabel{sec-conf-normal}
The following theorem is a slight variant of
\thref{theo-conf-ln}. It applies only to the cases where we
know a priori that the distribution of the measured data
follows a common gaussian distribution $N_{\mu, \sigma^2}$,
with $\mu$ and $\sigma$ fixed but unknown. It gives practically
the same result as \thref{theo-conf-ln} for the confidence
interval for the mean; in addition it gives a confidence
interval for the standard deviation. This result is often used
in practice, perhaps not rightfully, as the gaussian
assumptions are not always satisfied.
\begin{theorem}
\mylabel{theo-conf-normal} Let $X_1, ...,X_n$ be a sequence of
iid random variables with common distribution $N_{\mu,
\sigma^2}$. Let
 \bear
 \hat{\mu}_n&=& \frac{1}{n}\sum_{i=1}^n X_i\\
 \index{1sigma@$\hat{\sigma}_n$}\hat{\sigma}_n^2&=&\frac{1}{n-1}\sum_{i=1}^n \left(X_i -
 \index{1mu@$\hat{\mu}_n$}\hat{\mu}_n\right)^2 \mylabel{eq-conf-sigma2}
 \eear
Then\begin{itemize}
    \item The distribution of $\sqrt{n}\frac{\hat{\mu}_n-\mu}{\hat{\sigma}_n}$
    is Student's $t_{n-1}$; a confidence interval for the mean at level
    $\gamma$ is
    \be
    \hat{\mu}_n\pm \eta
    \frac{\hat{\sigma}_n}{\sqrt{n}}\label{eq-ci-mn}
    \ee
    where $\eta$ is the
    $\left( \frac{1+\gamma}{2}\right)$ quantile of the student distribution $t_{n-1}$.
\item The distribution of
    $(n-1)\frac{\hat{\sigma^2}_n}{\sigma^2}$ is
    $\chi^2_{n-1}$.

    A confidence interval at level $\gamma$ for the standard
    deviation is
    \be
    [\hat{\sigma}_n\sqrt{\frac{n-1}{\xi}},\hat{\sigma}_n \sqrt{\frac{n-1}{\zeta}}]
    \ee
    where $\zeta$ and $\xi$ are quantiles of $\chi^2_{n-1}$:
    $\chi^2_{n-1}(\zeta)=\frac{1+\gamma}{2}$ and
    $\chi^2_{n-1}(\xi)=\frac{1-\gamma}{2}$.

%    A confidence interval at level $\gamma$ for the standard
%    deviation is
%    \be
%    [\hat{\sigma}_n\sqrt{\frac{\zeta}{n-1}},\hat{\sigma}_n \sqrt{\frac{\xi}{n-1}}]
%    \ee
%    where $\zeta$ and $\xi$ are quantiles of $\chi^2_{n-1}$:
%    $\chi^2_{n-1}(\zeta)=\frac{1-\gamma}{2}$ and
%    $\chi^2_{n-1}(\xi)=\frac{1+\gamma}{2}$.
    \end{itemize}
\end{theorem}

The distributions $\chi^2$ and $t_n$  are defined as follows.
\nt{Chi-Square} ($\chi^2_n$)\index{$\chi^2_n$} is the
distribution of the sum of the squares of $n$ independent
random variables with distribution $N_{0,1}$ (its expectation
is $n$ and its variance $2n$). \nt{Student}
($t_n$)\index{$t_n$} is the distribution of
    \ben
    Z=\frac{X}{\sqrt{Y/n}}
    \een
    where $X \sim N_{0,1}$, $Y \sim \chi^2_n$ and $X$ and $Y$ are
    independent.

Unlike in \thref{theo-conf-ln}, the magic numbers
$\eta, \zeta, \xi$ depend on the confidence level
$\gamma$ but also on the sample size $n$. For
instance, with $n=100$ and confidence level
$0.95$, we have $\eta=1.98$, $\zeta=73.4$, and
$\xi=128.4$. This gives the confidence intervals
for mean and standard deviation: $[\hat{\mu}_n-
0.198 \hat{\sigma}_n, \hat{\mu}_n+ 0.198
\hat{\sigma}_n]$ and
$[0.86\hat{\sigma}_n,1.14\hat{\sigma}_n]$.

\mq{q-conf-kasdj}{Does the confidence interval for the mean in
\thref{theo-conf-normal} depend on the estimator of the
variance~? Conversely~?}{Yes; No}


We can compare the confidence interval for the
mean given by this theorem in \eref{eq-ci-mn} and
by \thref{theo-conf-ln} in \eref{eq-ci-m}. The
latter is only approximately true, so we may
expect some small difference, vanishing with $n$.
Indeed, the two formulas differ by two terms.
\noitemsep
\begin{enumerate}
    \item The estimators of the variance $\hat{\sigma}_n^2= \frac{1}{n-1}\sum_{i=1}^n \left(X_i-
 \hat{\mu}_n\right)^2$ and $s_n^2=\frac{1}{n}\sum_{i=1}^n \left(X_i-
 \hat{\mu}_n\right)^2$ differ by the factor
$\frac{1}{n}$ versus $\frac{1}{n-1}$. The factor
$\frac{1}{n-1}$ may seem unnatural, but it is
required for \thref{theo-conf-normal} to hold
exactly. The factor $\frac{1}{n}$ appears
naturally from the theory of maximum likelihood
estimation (\sref{sec-conf-mle}). In practice, it
is not required to have an extreme accuracy for
the estimator of $\sigma^2$ (since it is a second
order parameter); thus using $\frac{1}{n-1}$ or
$\frac{1}{n}$ makes little difference. Both
$\hat{\sigma}_n$ and $s_n$ are called \nt{sample
standard deviation}.
    \item $\eta$ in \eref{eq-ci-mn} is defined by the student distribution, and by the normal
     distribution in \eref{eq-ci-m}. For large
     $n$, the student distribution is close to
     normal; for example, with $\gamma=0.95$ and
     $n=100$, we have $\eta=1.98$ in
     \eref{eq-ci-mn} and $\eta=1.96$ in
     \eref{eq-ci-m}.
\end{enumerate} See \fref{fig-conf-c3} for an
illustration.
%\end{petit}
%
%
\ifnfs Not sure whether we need this example
\begin{ex}{Exponential Random Variables}
 Un CI exact pour n=10 rv exponentielles est $[20/34.17 \bar{x},
20/9.59 \bar{x}]$=$[0.5853, 2.0855]$. Compare to normal ci:
sigma=barx donc couverture de ce ci est 1- 23.38%
 ie l'intervalle serait beaucoup trop petit si on supposait
 normal.

errf((2.0855 -1)sqrt(n))-errf((1-0.5853) sqrt(n))
\end{ex}
\fi
%
%

\begin{figure}
\Ifig{bsci}{0.5}{0.5}%
\nfs{sgbdold.dat sgbdnew.dat bs.m}%
\mycaption{Confidence intervals for both compiler
options of \exref{ex-conf-1} computed with three
different methods: assuming data would be normal
(\thref{theo-conf-normal}) (left); the general
method in and with the bootstrap method (right).
%Though the normal assumption is not valid, as
%shown in \sref{sec-conf-verify}, the result
%obtained with it correct as it does not differ
%significantly from the asymptotic result
}
\mylabel{fig-conf-c3}
\end{figure}
\subsection{Confidence Intervals for Fairness Indices and The Bootstrap}
There is no analytical general method, even when
$n$ is large (but see \cite{verrill2003cbn} for
some special cases, if the data is i.i.d normal
or log-normal). Instead, we use a generic,
computational method, called the bootstrap. It is
general and can be used for any estimator, not
just to fairness indices. It applies to all cases
where data is iid.

\paragraph{The Bootstrap}
 \mylabel{sec-conf-bootstrap}
Consider a sample $\vec{x}=(x_1,...,x_n)$, which we assume to
be a realization of an iid sequence $X_1, ..., X_n$. We know
nothing about the common distribution $F()$ of the $X_i$s. We
are interested in some quantity $t(\vec{x})$ derived from the
data, for which we want to find a confidence interval (in this
context $t(\vec{x})$ is called a \nt{statistic}). For example,
if the statistic of interest is the Lorenz curve gap, then by
\sref{sec-var}:
 \ben
 t(\vec{x}) = \frac{1}{2
  \sum_{i=1}^n x_i}\sum_{j=1}^n \abs{x_j-\frac{1}{n}\sum_{i=1}^n x_i}
 \een
The \nt{bootstrap method} uses the sample
$\vec{x}=(x_1,...,x_n)$ as an approximation of the true,
unknown distribution. It is justified by the Glivenko-Cantelli
theorem which says that the ECDF converges with probability 1
to the true CDF $F()$ when $n$ gets large.

The method is described formally in
\aref{algo-bs}.
\begin{algorithm}\mycaption{The Bootstrap, for computation of
confidence interval at level $\gamma$ for the statistic
$t(\vec{x})$. The data set $\vec{x}=(x_1, ...,x_n)$ is assumed
to be a sample from an iid sequence, with unknown distribution.
$r_0$ is the algorithm's accuracy parameter.}
 \begin{algorithmic}[1]
%
  \State $R=\lceil 2 \;r_0/(1-\gamma)\rceil-1$ \label{al-bs-l1}\Comment{For example
  $r_0=25$, $\gamma=0.95$, $R=999$}

    \For{$r=1:R$}
    \State draw $n$ numbers with replacement from
the list $(x_1, ...,x_n)$ and call them
$X^r_{1},...,X^r_{n}$ \label{line-draw-xr}
    \State let $T^r=t(\vec{X^r})$ \label{al-bs-l3}
  \EndFor
  \State $\left(T_{(1)},...,T_{(R)}\right)=\mbox{sort}\left(T^1 ,...,T^R\right)$
  \State Prediction interval is
  $[T_{(r_0)}\;;\;T_{(R+1-r_0)}]$ \label{al-bs-l10}
\end{algorithmic}\label{algo-bs}
 \end{algorithm}
The loop creates $R$ \nt{bootstrap replicates} $\vec{X}^r$,
$r=1,...,R$. Each bootstrap replicate
$\vec{X}^r=(X^r_1,...,X^r_n)$ is a random vector of size $n$,
like the original data. All $X^r_i$ are independent copies of
the same random variable, obtained by drawing from the list
$(x_1,...,x_n)$ \imp{with replacement}. For example, if all
$x_k$ are distinct, we have $\P(X^r_i=x_k)=\frac{1}{n}$,
$k=1,...,n$.

For each $r$, line~\ref{al-bs-l3} computes the
value of the statistic obtained with the $r$th
``replayed" experiment. The confidence interval
in line~\ref{al-bs-l10} is the \nt{percentile
bootstrap estimate} at level $\gamma$. It is
based on the order statistic
$(T_{(r)})_{r=1,...,R}$ of $(T^r)_{r=1,...,R}$.

The value of $R$ in line~\ref{al-bs-l1} needs to be chosen such
that there are sufficiently many points outside the interval,
and depends on the confidence level. A good value is
$R=\frac{50}{1-\gamma}-1$. For example, with $\gamma=0.95$,
take $R=999$ and the confidence interval in
line~\ref{al-bs-l10} is $\left[T_{(25)};T_{(975)}\right]$.

\begin{ex}{Confidence Intervals for Fairness Indices} The confidence
intervals for the left two cases on
\fref{fig-lorenz} were obtained with the
Bootstrap, with a confidence level of $0.99$,
i.e. with $R=4999$ bootstrap replicates (left and
right: confidence interval; center: value of
index computed in \fref{fig-lorenz}).
 \begin{center}
\begin{tabular}{|c|lcr|lcr|}
  % after \\: \hline or \cline{col1-col2} \cline{col3-col4} ...
  \hline
   &
   \multicolumn{3}{c|}{Jain's Fairness Index}
   &
   \multicolumn{3}{c|}{Lorenz Curve Gap}
   \\ \hline \hline
  Old Code & 0.5385  &  0.6223  &  0.7057 &  0.2631  &  0.3209  &  0.3809  \\
  New Code & 0.5673 &   0.6584 &   0.7530 &0.2222  &  0.2754  &  0.3311   \\
%  Byte Count &  &    \\
 \hline
\end{tabular}
\end{center}
For the third example, the bootstrap cannot be applied
directly, as the data set is not iid and the bootstrap requires
i.i.d data. Subsampling does not work as the data set is long
range dependent. A possible method is to fit a long range
dependent model, such as fractional arima, then apply the
bootstrap to the residuals.
\end{ex}

The bootstrap may be used for any metric, not just for fairness
indices. \fref{fig-conf-c3} gives a comparison of confidence
intervals \emph{for the mean} obtained with the bootstrap and
with the classical methods (here
$t(\vec{x})=\frac{1}{n}\sum_{i=1}^n x_i$).

In general, the percentile estimate is an approximation that
tends to be slightly too small. For a theoretical analysis of
the bootstrap method, and other applications, see
\cite{davison1997bootstrap}.


\subsection{Confidence Interval for Success Probability}
This is the frequent case where we do $n$ independent
experiments and are interested in a binary outcome (success or
failure). Assume we observe $z$ successes (with $0\leq z \leq
n$). We would like to find a confidence interval for the
probability $p$ of success, in particular when $p$ is small.

Mathematically, we can describe the situation as follows. We
have a sequence $X_1, ..., X_n$ of independent Bernoulli random
variables such that $\P(X_k= 0)=1-p$ and $\P(X_k= 1)=p$, and we
observe $Z=\sum_{i=1}^n X_i$. The number $n$ of experiments is
known, but not the success probability $p$, which we want to
estimate. A natural estimator of $p$ is
$\frac{1}{n}\sum_{k=1}^n X_i$, i.e. the mean of the outcomes
(this is the maximum likelihood estimator, see
\sref{sec-conf-mle}). Therefore, we can apply the method for
confidence intervals for the mean in \thref{theo-conf-ln};
however, this method is valid only asymptotically, and does not
work when $z$ is very small compared to $n$. A frequent case of
interest is when we observe no success ($z= 0$) out of $n$
experiments; here, \thref{theo-conf-ln} gives $[0;0]$ as
confidence interval for $p$, which is not correct. We can use
instead the following result.

\begin{shadethm} \cite[p. 110]{johnson2005univariate} Assume we observe $z$ successes out of $n$
independent experiments. A confidence interval at level
$\gamma$ for the success probability $p$ is $[L(z) ; U(z)]$
with
  \be
  \bracket{
  L(0)  =  0
  \\
  L(z)  =  \phi_{n,z-1}\lp\frac{1+\gamma}{2}\rp, \; z=1,...,n
  \\
  U(z)  =  1- L(n-z)
  }\label{eq-ci-p-u}
  \ee
  where $\phi_{n,z}(\alpha)$ is defined for $n=2,3,...$, $z
  \in \{0,1,...,n\}$ and $\alpha \in (0;1)$ by
%
  \be
  \bracket{
  \phi_{n,z}(\alpha) = \frac{n_1 f}{n_2 + n_1 f}\\
  n_1 = 2 (z+ 1),\; n_2 = 2 ( n - z), \; 1 - \alpha  = F_{n_1,n_2}( f)
 }\label{eq-ci-p-ksjsda}
  \ee
  ($F_{n_1, n_2}()$ is the CDF of the Fisher distribution
  with $n_1, n_2$ degrees of freedom). In particular, the
  confidence interval for $p$ when we observe $z= 0$ successes
  is $[0; p_0(n)]$ with
  \be
 p_0(n)
 =
 1-\lp \frac{1-\gamma}{2} \rp ^{1 \over n}
 =
 \frac{1}{n}\log \lp\frac{2}{1-\gamma}\rp + o \lp
\frac{1}{n}\rp\mbox{ for large } n
 \label{eq-ci-p-kk}\ee

Whenever $z\geq 6$ and $n-z \geq 6$, the normal approximation
 \be
 \bracket{
 L(z)   \approx   \frac{z}{n} - \frac{\eta}{n}  \sqrt{z \lp 1 -\frac{z}{n} \rp}
 \\
 U(z)   \approx   \frac{z}{n} + \frac{\eta}{n}  \sqrt{z \lp 1 -\frac{z}{n} \rp}
 }
 \label{eq-ci-p-dasksadlk}\ee
 can be used instead, with $N_{0,1}(\eta)=\frac{1
 +\gamma}{2}$.\label{theo-est-proba}
\end{shadethm}

The confidence interval in the theorem is not the best one, but
it is perhaps the simplest. It is based on a symmetric coverage
interval, i.e. the probability of being above (or below) is
$<\frac{1-\gamma}{2}$ and it is the smallest interval with this
property. Other, non symmetric intervals can be derived and are
slightly smaller \cite{blyth1983binomial}.

Note that the function $\phi_{n,z}()$   is the reverse mapping
of $p \mapsto B_{n,p}(z)$ where $B_{n,p}()$ is the CDF of the
binomial distribution (this explains \eref{eq-ci-p-kk}).
\eref{eq-ci-p-ksjsda} is used in numerical implementations
\cite{johnson2005univariate}.

For $\gamma = 0.95$, \eref{eq-ci-p-kk} gives $p_0(n) \approx
  \frac{3.689}{n}$ and this is accurate with less than $10\%$
  relative error for $n \geq 20$ already.

The confidence interval in \eref{eq-ci-p-dasksadlk} is obtained
by application of the asymptotic confidence interval for the
mean; indeed, a direct application of \thref{theo-conf-ln}
gives $\hat{\mu}_n=\frac{z}{n}$ and $s^2_n=\frac{z (n-z)}{n}$.

\begin{exnn}{Sensor Loss Ratio}
We measure environmental data with a sensor network. There is
reliable error detection, i.e. there is a coding system which
declares whether a measurement is correct or not. In a
calibration experiment with $10$ independent replications, the
system declares that all measurements are correct. What can we
say about the probability $p$ of finding an incorrect
measurement ?

Apply \eref{eq-ci-p-kk}: we can say, with $95\%$ confidence,
that $p\leq 30.8\%$.

Later, in field experiments, we find that $32$ out of $145$
readings are declared incorrect. Assuming the measurements are
independent, what can we say about $p$ ?

Apply \eref{eq-ci-p-dasksadlk} with $z=32$, $n=145$: with
$95\%$ confidence we can say that $L \leq p \leq U$ with
 \ben
 \bracket{
 L \approx \frac{z}{n} - \frac{1.96}{n}  \sqrt{z \lp 1 -\frac{z}{n} \rp} = 15.3\%
 \\
 U \approx \frac{z}{n} + \frac{1.96}{n}  \sqrt{z \lp 1 -\frac{z}{n} \rp} = 28.8\%
 }
 \een
Instead of the normal approximation in
\eref{eq-ci-p-dasksadlk}, we could have used the exact formula
in \eref{eq-ci-p-u}, which would give $L=  15.6\%, U= 29.7\%$.
\end{exnn}

\thref{theo-est-proba} is frequently used in conjunction with
Monte Carlo estimation of the $p$-value of a test, see
\exref{sec-montecarlo-pvalue} on \pgref{sec-montecarlo-pvalue}.
%
%
\section{The Independence Assumption}
\mylabel{sec-iid-cose}All results in the previous and the next
section assume the data is a sample of a sequence of
independent and identically distributed (iid) random variables.
We discuss here in detail the meaning of this assumption (in
\sref{sec-normal-assumption} we also discuss the gaussian
assumption, required by Theorems~\ref{theo-conf-ln} and
\ref{theo-conf-normal}).

\subsection{What does iid mean~?}

Iid-ness is a property of a stochastic model, not of the data.
When we say, by an abuse of language, that the collected data
set is iid, we mean that we can do as if the collected data
$x_1, ..., x_n$ is a sample (i.e. a simulation output) for a
sequence of random variables $X_1, ..., X_n$, where $X_1,
...,X_n$ are independent and all have the same (usually
unknown) distribution with CDF $F()$.

To generate such as sample, we draw a random
number from the distribution $F()$, using a
random number generator (see \sref{sec-arbdis}).
Independence means that the random numbers
generated at every step $i$ are discarded and not
re-used in the future steps $i+1, ...$. Another
way to think of independence is with conditional
probabilities: for any set of real numbers $A$
 \be
 \P(X_i \in A \;|\; X_1=x_1, ..., X_{i-1}=x_{i-1})=\P(X_i \in A)
 \ee
i.e. \imp{if we know the distribution $F(x)$}, observing
$X_1,...,X_{i-1}$ does not give more information about $X_i$.

Note the importance of the ``if" statement in the last
sentence: remove it and the sentence is no longer true. To
understand why, consider a sample $x_1,...,x_n$ for which we
assume to know that it is generated from a sequence of iid
random variables $X_1, ...,X_n$ with normal distribution but
with unknown parameter $(\mu, \sigma^2)$. If we observe for
example that the average of $x_1, ...,x_{n-1}$ is 100 and all
values are between 0 and 200, then we can think that it is very
likely that $x_n$ is also in the interval $[0, 200]$ and that
it is unlikely that $x_n$ exceeds $1000$. Though the sequence
is iid, we did gain information about the next element of the
sequence having observed the past. There is no contradiction:
if we know that the parameters of the random generator are
$\mu=100$ and $\sigma^2=10$ then observing $x_1,...,x_{n-1}$
gives us no information about $x_n$.


%
\subsection{How do I know in Practice if the iid Assumption is
Valid~?} If your performance data comes from a \nt{designed
experiment}, i.e. a set of simulation or tests that is entirely
under your control, then it is up to you to design things in
such a way that the collected data are iid. This is done as
follows.

Every experiment has a number of factors, i.e., parameters that
are likely to influence the outcome. Most of the factors are
not really interesting, but you have to account for them in
order to avoid hidden factor errors (see \sref{sec-hf} for
details). The experiment generates iid data if the values of
the factors are chosen in an iid way, i.e., according to a
random procedure that is the same for every measured point, and
is memoriless. Consider \exref{ex-conf-1}, where the run time
for a number of transactions was measured. One factor is the
choice of the transaction. The data is made iid if, for every
measurement, we choose one transactions \imp{randomly with
replacement} in a list of transactions.

A special case of designed experiment is
simulation. Here, the method is to generate
\imp{replications} without resetting the random
number generator, as explained in
\sref{stats-sim}.

Else (i.e. your data does not come from a designed experiment
but from measurements on a running system) there is little
chance that the complete sequence of measured data is iid. A
simple fix is to \imp{randomize the measurements}, in such  a
way that from one measurement point to the other there is
little dependence. For example, assume you are measuring the
response time of an operational web server by data mining the
log file. The response time to consecutive requests is highly
correlated at the time scale of the minute (due to protocols
like TCP); one common solution is to choose requests at random,
for example by selecting one request in average every two
minutes.

If there is some doubt, the following methods can be used to
verify iid-ness:\doitemsep
\begin{enumerate}
\item (Autocorrelation Plot): If the data appears to be
    stationary (no trend, no seasonal component), then we
    can plot the sample autocorrelation coefficients, which
    are an estimate of the true autocorrelation
    coefficients $\rho_k$ (defined on \pgref{def-acf}). If
    the data is iid, then $\rho_k=0$ for $k \geq 1$, and
    the sample autocorrelation coefficients fall within the
    values $\pm 1.96 / \sqrt{n}$ (where $n$ is the sample
    size) with $95\%$ probability. An autocorrelation plot
    displays these bounds as well. A visual inspection can
    determine if this assumption is valid. For example, on
    \fref{fig-cpu-sub} we see that there some
    autocorrelation in the first six diagrams but not in
    the last two. If visual inspection is not possible, a
    formal test can be used (the Ljung-Box test,
    \sref{sec-sacf}). If the data is iid, any point
    transformation of the data (such as the Box Cox
    transformation for any exponent $s$,
    \sref{sec-conf-rescale}) should appear to be non
    correlated as well.

    \item (\nt{Lag-Plot}): We can also plot the value of
        the data at time $t$ versus at time $t+h$, for
        different values of $h$ (lag plots). If the data is
        iid, the lag plots do not show any trend. On
        \fref{fig-joe-sacaisse} we see that there is a
        negative trend at lag 1.

    \item (Turning Point Test):  A test provides an
        automated answer, but is sometimes less sure than a
        visual inspection. A test usually has a null
        hypothesis and returns a so called ``$p$-value"
        (see \cref{ch-tests} for an explanation). If the
        $p$-value is smaller than $\alpha=1-\gamma$, then
        the test rejects the null hypothesis at the
        confidence level $\gamma$. %The turning point test
%    computes the number of times that the data goes from
%    increasing to decreasing. This value should be close to
%    $2/3$ if the data is iid.
See \sref{sec-turning-point} for details.


\end{enumerate}

\begin{figure}
\Ifig{cpu7}{1}{0.5}%
\nfs{cpu.dat cpu.m}%
\mycaption{Execution times for $n=7632$ requests
(top left) and autocorrelation function (bottom
left), and for the data sub-sampled with
probability $p=1/2$ to $1/2^{7}=1/128$. The data
appears stationary and roughly normal so the
auto-correlation function can be used to test
independence. The original data is positively
correlated, but the sub-sampled data looses
correlation when the sampling probability is
 $p=1/64$. The turning point test for the subsampled data with $p=1/64$ has a $p$-value of
$0.52648$, thus at confidence level $0.95$ we
accept the null hypothesis, namely, the data is
iid.
 The sub-sampled data has 116 points, and the confidence interval obtained from this
 for the median of the sub-sampled data is $[66.7,   75.2]$ (using
 \thref{theo-conf-ci-median}). Compare with the confidence interval that
 would be obtained if we would (wrongly) assume the data to be iid :
 $[69.0, 69.8]$. The iid assumption underestimates the
 confidence interval because the data is positively correlated.}\mylabel{fig-cpu-sub}
\end{figure}
%
\subsection{What Happens If The IID Assumption Does Not
Hold~?} \mylabel{sec-non-iid} If we compute a confidence
interval (using a method that assumes iid data) whereas the iid
assumption does not hold, then we introduce some bias. Data
arising from high resolution measurements are frequently
positively correlated. In such cases, the confidence interval
is too small: there is not as much information in the data as
one would have if they would be iid (since the data tends to
repeat itself); see \fref{fig-cpu-sub} for an example.


It may still be possible to obtain confidence
intervals when the data does not appear to be
iid. Two possible methods are:

\begin{description}
    \item[Sub-sampling] This means select a
fraction $p$ of the measured data, and verify
that the iid assumption can be made for the
selected data. The hope is that correlation
disappears between data samples that are far
apart.

A simple way would be to keep every $pn$ data
sample, where $n$ is the total number of points,
but this is not recommended as such a strict
periodic sampling may introduce unwanted
anomalies (called aliasing). A better method is
to decide independently for each data point, with
probability $p$, whether it is sub-sampled or
not.

For example, on \fref{fig-cpu-sub}, sub-sampling
works for $p\leq 1/64$; the confidence interval
for the median is much larger than if we would
(wrongly) assume the original data to be iid.

Sub-sampling is very simple and efficient. It does not
always work, though: it does not work if the data set is
small, nor for some large data sets, which remain
correlated after repeated sub-sampling (such data sets are
called long range dependent).

    \item[Modelling] is more complex but applies
    when sub sampling does not. It consists in
    fitting a parametric model appropriate to the type of data, and
    computing confidence intervals for the model
parameters (for example using
\sref{sec-conf-mle}). We illustrate the method on
the next example.
\end{description}

\begin{figure}[!htbp]
\begin{center}
\subfigure[Data]{\Ifignc{joeplot}{0.5}{0.3}}
\subfigure[Data]{\Ifignc{joeqq}{0.5}{0.3}}
\subfigure[Autocorrelation]{\Ifignc{joeacf}{0.3}{0.3}}
\subfigure[Lag Plots]{\Ifignc{joelag}{0.5}{0.5}}%
\nfs{joe.dat joe.m cais.ssc joepost.m}%
  \end{center}
  \mycaption{Daily balance at Joe's wireless access shop over 93 days. The lag
  plots
  show $y(t)$ versus $y(t+h)$ where $y(t)$ is the time series in (a). The data
  appears to have some correlation at lag 1 and is thus clearly not iid.}
  \mylabel{fig-joe-sacaisse}
\end{figure}


\begin{ex}{Joe's Balance Data}
Joe's shop sells online access to visitors who download
electronic content on their smartphones. At the end of day
$t-1$, Joe's employee counts the amount of cash $c_{t-1}$
present in the cash register and puts it into the safe. In the
morning of day $t$, the cash amount $c_{t-1}$ is returned to
the cash register. The total amount of service sold (according
to bookkeeping data) during day $t$ is $s_t$. During the day,
some amount of money $b_t$ is sent to the bank. At the end of
day $t$, we should have $c_t=c_{t-1}+s_t-b_t$. However, there
are always small errors in counting the coins, in bookkeeping
and in returning change. Joe computes the balance
$Y_t=c_t-c_{t-1}-s_t+b_t$ and would like to know whether there
is a systematic source of errors (i.e. Joe's employee is losing
money, maybe because he is not honest, or because some
customers are not paying for what they take). The data for
$Y_t$ is shown on \fref{fig-joe-sacaisse}. The sample mean is
$\mu=-13.95$, which is negative. However, we need a confidence
interval for $\mu$ before risking any conclusion.

If we would assume that the errors $Y_t$ are iid,
then a confidence interval would be given by
\thref{theo-conf-ln} and we find approximately
$[-43, 15]$. Thus, with the iid model, we cannot
conclude that there is a fraud.

However, the iid assumption is not valid, as
\fref{fig-joe-sacaisse} shows (there is a strong correlation at
lag 1; this is confirmed by the lag plot). We use a modelling
approach. A similar problem is discussed in
\cite[Example~3.2.8]{brockwell2002introduction}, with oil
rather than money leakage; the authors in
\cite{brockwell2002introduction} conclude that a moving average
model can be used. We apply the same approach here. First note
that $Y_t$ appears to be reasonably gaussian (also see
\sref{sec-normal-assumption}), and has correlation only at lag
1. We study such processes in \cref{ch-forecast}; a gaussian
process that has correlation only at lag 1 is the moving
average process, which satisfies
 \ben
 Y_t- \mu = \eps_t + \alpha \eps_{t-1}
\een
 where $\eps_t$ is iid $N_{0,\sigma^2}$. This
is a parametric model, with parameter
$(\mu,\alpha, \sigma)$. We can fit it using a
numerical package or the methods in
\cref{ch-forecast}. A confidence interval for
$\mu$ can be obtained using \thref{theo-mle2} and
\thref{theo-mle-filter}.
 Here, it is plausible that the sample size is large enough.
For any fixed $\mu$, we compute the profile
log-likelihood. It is obtained by fitting an
MA(1) process to $W_t:=Y_t- \mu$. Good
statistical packages give not only the MLE fit,
but also the log-likelihood of the fitted model,
which is exactly the profile log-likelihood
$pl(\mu)$. The MLE $\hat{\mu}$ is the value of
$\mu$ that maximizes $pl(\mu)$, and $-2
(pl(\hat{\mu})-pl(\mu))$ is approximately
$\chi^2_1$. \fref{fig-joe-sacaisse-mle} shows a
plot of $pl(\mu)$.


It follows that $\hat{\mu} = -13.2$ and an
approximate $95\%$-confidence interval is
$[-14.1, -12.2]$. Contrary to the iid model, this
suggests that there \emph{is} a loss of money, in
average $13${\euro} per day.
 \label{ex-joe-sa-caisse}
\end{ex}
\begin{figure}[!htbp]
\Ifig{joell}{0.5}{0.3}%
\nfs{joe.m joe.dat cais.ssc joepost.m}%
  \mycaption{Profile Log Likelihood for the Moving Average model of Joe's balance data. The horizontal
  line is at a value $\eta/2=1.92$ below the maximum, with $\chi^2_1(\eta)=0.95$; it gives an approximate
  confidence interval for the mean of the data on the $x$ axis. }
  \mylabel{fig-joe-sacaisse-mle}
\end{figure}
%
%
\mq{q-conf-kjnliun}{Give an example of identically distributed
but dependent random variables.}{Here is a simple one: assume
$X_1, X_3, X_5, ...$ are iid with CDF $F()$ and let $X_2=X_1$,
$X_4=X_3$ etc. The distribution of $X_i$ is $F()$ but the
distribution of $X_2$ conditional to $X_1=x_1$ is a dirac at
$x_1$, thus depends on $x_1$. The random choices taken for
$X_1$ influence (here deterministically) the value of $X_2$.}

\section{Prediction Interval}
\mylabel{sec-conf-pi}

The confidence intervals studied before quantify the accuracy
of a mean or median; this is useful for diagnostic purposes,
for example we can assert from the confidence intervals on
\fref{fig-conf-c2} that the new option does reduce the run
time, because the confidence intervals for the mean (or the
median) are in the positive numbers.

Sometimes we are interested in a different viewpoint and would
like to characterize the \imp{variability} of the data: for
example we would like to summarize which run time can be
expected for an arbitrary future (non observed) transaction.
Clearly, this run time is random. A \nt{prediction interval} at
level $\gamma$ is an interval that we can compute by observing
a realization of $X_1,...,X_n$ and such that, with probability
$\gamma$, a future transaction will have a run time in this
interval. Intuitively, if the common CDF of all $X_i$s would be
known, then a prediction interval would simply be an
inter-quantile interval, for example $[m_{\alpha/2},
m_{1-{\alpha/2}}]$, with $\alpha=1-\gamma$. For example, if the
distribution is normal with known parameters, a prediction
interval at level $0.95$ would be $\mu \pm 1.96 \sigma$.
However, there is some additional uncertainty, due to the fact
that we do not know the distribution, or its parameters a
priori, and we need to estimate it. The prediction interval
capture both uncertainties. Formally, the definition is as
follows.

\begin{definition}
Let $X_1,...,X_n, X_{n+1}$ be a sequence of random variables. A
prediction interval at level $\gamma$ is an interval of the
form $[u(X_1,...,X_n),v(X_1,...,X_n)]$ such that
 \be
 \P\left(u(X_1,...,X_n) \leq     X_{n+1} \leq
 v(X_1,...,X_n)\right)\geq \gamma
 \ee
\end{definition}

Note that the definition does not assume that $X_i$ is iid,
however we focus in this chapter on the iid case.
% (but see
%\sref{sec-conf-niid} for a discussion of the more general
%case).
The trick is now to find functions $u$ and $v$ that are
pivots, i.e. their distribution is known even if the common
distribution of the $X_i$s is not (or is not entirely known).

There is one general result, which applies in practice to
sample sizes that are not too small ($n \geq 39$), which we
give next.
\subsection{Prediction for an IID Sample based on Order Statistic}

\begin{shadethm}[General IID Case]
\mylabel{theo-conf-pi} Let $X_1,...,X_n,X_{n+1}$ be an iid
sequence and assume that the common distribution has a density.
Let $X^n_{(1)},...,X^n_{(n)}$ be the order statistic of
$X_1,...,X_n$. For $1\leq j \leq k \leq n$:
 \be
 \P\left(X^n_{(j)} \leq X_{n+1}\leq X^n_{(k)}\right)=\frac{k-j}{n+1}
 \ee
thus for $\alpha \geq \frac{2}{n+1}$, $[X^n_{(\lfloor
(n+1)\frac{\alpha}{2}\rfloor)},X^n_{(\lceil
(n+1)\left(1-\frac{\alpha}{2}\right)\rceil)}]$
 is a prediction interval at level at least
 $\gamma=1-\alpha$.
\end{shadethm}

For example, with $n=999$, a prediction interval at level
$0.95$ ($\alpha=0.05$) is $[X_{(25)},X_{(975)}]$. This theorem
is similar to the bootstrap result in
\sref{sec-conf-bootstrap}, but is exact and much simpler.

\mq{q-conf-asasklkasluz}{We have obtained $n$ simulation
results and use the prediction interval $[m,M]$ where $m$ is
the smallest result and $M$ the largest. For which values of
$n$ is this a prediction interval at level at least
$95\%$~?}{The interval is $[X_{(1)},X_{(n)}]$ thus the level is
$\frac{n-1}{n+1}$. It is $\geq 0.95$ for $n\geq 39$. We need at
least 39 samples to provide a 95\% prediction interval.}

For very small $n$, this result gives poor prediction intervals
with values of $\gamma$ that maybe far from 100\%. For
example, with $n=10$, the best prediction we can do is
$[x_{\min}, x_{\max}]$, at level $\gamma=81\%$. If we can
assume that the data is normal, we have a stronger result,
shown next.

\subsection{Prediction for a Normal IID Sample}
\begin{shadethm}[Normal IID Case]
\mylabel{theo-conf-pin} Let $X_1,...,X_n,X_{n+1}$ be an iid
sequence with common distribution $N_{\mu,\sigma^2}$. Let
$\hat{\mu}_n$ and $\hat{\sigma}_n^2$ be as in
\thref{theo-conf-normal}. The distribution of
$\sqrt{\frac{n}{n+1}}\frac{X_{n+1}-\hat{\mu}_n}{\hat{\sigma}_n}$
    is Student's $t_{n-1}$; a prediction interval at
level $1-\alpha$ is
 \be\hat{\mu}_n\pm \eta' \sqrt{1+\frac{1}{n}}\hat{\sigma}_n
 \ee
 where $\eta'$ is the $\left(1-\frac{\alpha}{2}\right)$
quantile of the student distribution $t_{n-1}$.

For large $n$, an approximate prediction interval is
 \be
 \hat{\mu}_n\pm \eta \hat{\sigma}_n \label{eq-pred-n}
 \ee
 where $\eta$ is the $\left(1-\frac{\alpha}{2}\right)$
quantile of the normal distribution $N_{0,1}$.
\end{shadethm}
For example, for $n=100$ and $\alpha=0.05$ we obtain the
prediction interval (we drop the index $n$): $[\hat{\mu}- 1.99
\hat{\sigma}, \hat{\mu}+ 1.99 \hat{\sigma}]$. Compare to the
confidence interval for the mean given by
\thref{theo-conf-normal} where the width of the interval is
$\approx 10= \sqrt{n}$ times smaller. For a large $n$, the
prediction interval is approximately equal to $\hat{\mu}_n\pm
\eta \hat{\sigma}_n$, which is the interval we would have if we
ignore the uncertainty due to the fact that the parameters
$\mu$ and $\sigma$ are estimated from the data. For $n$ as
small as $26$, the difference between the two is $7\%$ and can
be neglected in most cases.

The normal case is also convenient in that it requires the
knowledge of only two statistics, the mean $\hat{\mu}_n$ and
the mean of squares (from which $\hat{\sigma}_n$ is derived).

\imp{Comment} Compare the prediction interval in
\eref{eq-pred-n} to the confidence interval for the mean in
\eref{eq-ci-mn}: there is a difference of $\frac{1}{\sqrt{n}}$;
the confusion between both is frequently done: when comparing
confidence interval, check if the standard deviation is indeed
divided by $\sqrt{n}$~!


\begin{ex}{File Transfer Times}
\mylabel{ex-conf-ft1a} \fref{fig-conf-ft1a} shows the file
transfer times obtained in 100 independent simulation runs,
displayed in natural and log scales. The last panel shows
$95\%$-prediction intervals. The left interval is obtained with
the method of order statistic (\thref{theo-conf-pi}); the
middle one by (wrongly) assuming that the distribution is
normal and applying \thref{theo-conf-pi} -- it differs largely.

The right interval is obtained with a log transformation.
First, a prediction interval $[u(Y_1,...,Y_n),v(Y_1,...Y_n)]$
is computed for the transformed data $Y_i=\ln(X_i)$; the
prediction interval is mapped back to the original scale to
obtain the prediction interval
$[\exp(u(\ln(X_1,...,\ln(X_n))),\exp(v(\ln(X_1,...,\ln(X_n)))]$.
We leave it to the alert reader to verify that this reverse
mapping is indeed valid. The left and right intervals are in
good agreement, but the middle one is obviously wrong.

The prediction intervals also show the central values (with
small circles). For the first one, it is the median. For the
second one, the mean. For the last one,
$\exp\left(\frac{\sum_{i=1}^nY_i}{n}\right)$, i.e. the back
transformed of the mean of the transformed data (here, the
geometric mean).
\begin{figure}
\center
 \subfigure[(Data)]{\Ifignc{pif1}{0.25}{0.3}}
 \subfigure[(Log of data)]{\Ifignc{pif3}{0.25}{0.3}}
 \subfigure[(Prediction Intervals)]{\Ifignc{pif5}{0.4}{0.3}}%%
\nfs{fileTransferLogNormal.dat fileTransferIntervals.m}%%
\nfs{fileTransferLogNormal.dat fileTransferIntervals.m nn.m}%
  \mycaption{File transfer times for 100 independent simulation runs,
  with prediction
  intervals computed with
  the order statistic (1), assuming the data is normal (2) and assuming the
  log of data is normal (3).}
  \mylabel{fig-conf-ft1a}
\end{figure}
\end{ex}
\mq{q-conf-akskljsaklj}{The prediction intervals in
\fref{fig-conf-ft1a} are not all symmetric around the central
values. Explain why.}{First interval: the distribution of the
data is obviously not symmetric, so the median has no reason to
be in the middle of the extreme quantiles. Second interval: by
nature, it is strictly symmetric. Third interval: it is the
exponential of a symmetric interval; exponential is not an
affine transformation, so we should not expect the transformed
interval to be symmetric.}

%\begin{ex}{File Transfer Times} (Continuation of \exref{ex-conf-ft1a}).
%\fref{fig-conf-ft1b} shows the qq-plots of the bootstrap
%replicates used for estimating the mean of the data and of the
%log of the data. For the original data, the bootstrap
%replicates do not appear to be normal, thus the asymptotic
%result in \thref{theo-conf-ln} does not apply. It is the
%opposite for the log of the data.
%
%\mq{q-conf-akskljsaklj}{Compare this finding to the confidence
%intervals found in \fref{fig-conf-ft1a}.}{The figure shows the
%confidence intervals with the normal assumption and the
%bootstrap percentile estimates. With $n=100$, the normal
%assumption (\thref{theo-conf-normal}) and the asymptotic regime
%(\thref{theo-conf-ln}) give practically the same result. Thus
%we expect the confidence intervals obtained with either the
%normal assumption or the asymptotic regime to be wrong for the
%data, and correct for the log of the data, consistent with
%\fref{fig-conf-ft1a}.}
%\begin{figure}
%\Ifig{ftbsqq}{0.6}{0.3}%
%\nfs{fileTransferLogNormal.dat nn.m}%
%  \mycaption{QQplots of bootstrap replicates of the estimator of the mean for the file transfer data
%  in \fref{fig-conf-ft1a}. The bootstrap replicates of the data are not normally distributed,
%  but those of the log of the data are.}
%  \mylabel{fig-conf-ft1b}
%\end{figure}
%\end{ex}
%
%
%
%\Begin{Ex}{File Transfer Times}\Mylabel{Ex-Conf-Ft1} (Continuation Of \Exref{Ex-Conf-Ft1A}).
%\Fref{Fig-Conf-Ft1} Shows The Qq-Plots Of The File Transfer
%Times
% And Their Logs. It
% Shows That The Data Is Not Normal But The Log Of The Data
%Is.
%
%
%\Begin{Figure}
%\Center
% \Subfigure[(Qq-Plot)]{\Ifignc{Pif2}{0.3}{0.3}}
% \Subfigure[(Qq-Plot Of Log Of Data)]{\Ifignc{Pif4}{0.3}{0.3}}
%
%  \Mycaption{File Transfer Times For 100 Independent Simulation Runs, With Prediction
%  Intervals Computed With The Three Methods Discussed In \Exref{Ex-Conf-Ft1}: (1) Based On
%  Order Statistics (2) Based On Mean And Standard Deviation (3) Based On Mean And Standard Deviation After
%  Re-Scaling.}
%  \Mylabel{Fig-Conf-Ft1}
%\End{Figure}
%\End{Ex}
%
%

There is no ``large $n$" result for a prediction interval, like
there is in \thref{theo-conf-ln}: a prediction interval depends
on the original distribution of the $X_i$s, unlike confidence
intervals for the mean, which depend only on first and second
moments thanks to the central limit theorem.
\thref{theo-conf-pin} justifies the common practice of using
the standard deviation as a measure of dispersion; however it
provides useful prediction intervals only if the data appears
to be iid \emph{and} normal. In the next section discuss how to
verify normality.

\subsection{The Normal Assumption}
\label{sec-normal-assumption}


\subsubsection{QQplots} \mylabel{sec-qqplot} This is a simple method for verifying the
normal assumption, based on visual inspection.
% of
%qq-plots (defined in the next section). More formal, automated
%methods use tests, as described in \cref{ch-tests}.
A \nt{probability plot}, also called \nt{qq-plot}, compares two
samples $X_i$, $Y_i$, $i=1,...,n$ in order to determine whether
they come from the same distribution. Call $X_{(i)}$ the
\nt{order statistic}, obtained by sorting $X_i$ in increasing
order. Thus $X_{(1)}\leq X_{(2)}\leq ...$. The qq-plot displays
the points $(X_{(i)}, Y_{(i)})$. If the points are
approximately along a straight line, then the distributions of
$X_i$ and $Y_i$ can be assumed to be the same, modulo a change
of scale and location.

Most often, we use qqplots to check the distribution of $Y_i$
against a probability distribution $F$. To do so, we plot
$(x_i, Y_{(i)})$, where $x_i$ is an estimation of the expected
value of $\E(Y_{(i)})$, assuming the marginal of $Y_i$ is $F$.
The exact value of $\E(Y_{(i)})$ is hard to obtain, but a
simple approximation (assuming that $F$ is strictly increasing)
is \cite{davison2003sm}:
 \ben x_i := F^{-1}\left(\frac{i}{n+1}\right)
 \een
A \nt{normal qqplots}, is a qqplot such that $F=N_{0,1}$, and
is often used to visually test for normality
(\fref{fig-conf-qq}). More formal tests are the Jarque Bera
test (\sref{sec-skku}) and the goodness of fit tests in
\sref{sec-lrs-asymptotic}.
%
%
%
\begin{figure}
\center
 \subfigure[(QQ-plot)]{\Ifignc{pif2}{0.3}{0.3}}
 \subfigure[(QQ-plot of log of data)]{\Ifignc{pif4}{0.3}{0.3}}
 \subfigure[(normal sample)]{\Ifignc{normal}{0.3}{0.3}}%
\nfs{fileTransferLogNormal.dat fileTransferIntervals.m}%
  \mycaption{Normal qqplots of file transfer times in \fref{fig-conf-ft1a}
  and of an artificially generated sample from the normal distribution
  with the same number of points.
 The former plot shows large deviation from normality, the second does not.}
  \mylabel{fig-conf-qq}
\end{figure}



\subsubsection{Rescaling, Harmonic, Geometric and Other Means}
\mylabel{sec-conf-rescale} \fref{fig-conf-ft1a} illustrates
that the use of standard deviation as a basis for a prediction
interval may be better if we re-scale the data, using a point
tranformation. The \nt{Box-Cox transformation} is commonly used
for that. It has one shape parameter $s$ and is given by
\begin{equation}\mylabel{stats-eqbc}
  b_s(x)=
  \left\{
  \begin{array}{ll}
    \frac{x^s-1}{s} & ,\; s \neq 0  \\
    \ln x & ,\; s = 0
  \end{array}
  \right.
\end{equation}
Commonly used parameters are $s=0$ (log tranformation), $s=-1$
(inverse), $s=0.5$ and $s=2$. The reason for this specific form
is to be continuous in $s$.

It is easy to see (as in \exref{ex-conf-ft1a})  that a
\imp{prediction interval} for the original data can be obtained
by reverse-transforming a prediction interval for the
transformed data. In contrast, this does not hold for
\imp{confidence intervals}. Indeed, by reverse-transforming a
confidence interval for the mean of the transformed data, we
obtain a confidence interval for another type of mean
(harmonic, etc.). More precisely, assume we transform a data
set $x_1,...,x_n$ by an invertible (thus strictly monotonic)
mapping $b()$ into $y_1,...y_n$, i.e. $y_i=b(x_i)$ and
$x_i=b^{-1}(y_i)$ for $i=1,...,n$. We called \nt{transformed
sample mean} the quantity $b^{-1}(\frac{1}{n}\sum_{i=1}^n
y_i)$, i.e. the back-transform of the mean of the transformed
data. Similarly, the \nt{transformed distribution mean} of the
distribution of a random variable $X$ is $b^{-1}(\E(b(X))$.
When $b()$ is a Box-Cox transformation with index $s=-1, 0$ or
$2$ we obtain the classical following definitions, valid for a
positive data set $x_i, i=1...,n$ or a random variable $X$:

\vspace{0.5cm}
\begin{tabular}{r|c|c|c}
  % after \\: \hline or \cline{col1-col2} \cline{col3-col4} ...
    & \emph{Transformation} & \emph{Transformed Sample Mean} & \emph{Transformed Distribution
    Mean}
    \\ \hline
  \nt{Harmonic} &
  $b(x)=1/x$ &
  $\frac{1}{\frac{1}{n}\sum_{i=1}^n \frac{1}{x_i}}$ &
    $\frac{1}{\E(\frac{1}{X})}$
    \\
   \nt{Geometric}&
   $b(x)=\ln(x)$&
    $ (\prod_{i=1}^n  x_i)^{\frac{1}{n}}$ &
    $e^{\E(\ln{X})}$
      \\
  \nt{Quadratic}  &
  $b(x)=x^2$&
  $\sqrt{\frac{1}{n}\sum_{i=1}^n x_i^2}$ &
   $\sqrt{\E(X^2)}$\\
\end{tabular}
\vspace{0.5cm}

\begin{shadethm}
A confidence interval for a transformed mean is obtained by the
inverse transformation of a confidence interval for the mean of
the transformed data.\label{theo-harmos}
\end{shadethm}
For example, a confidence interval for the geometric mean is
the exponential of a confidence interval for the mean of the
logarithms of the data.

\section{Which Summarization To Use~?}
\mylabel{sec-conf-mom}

In the previous sections we have seen various summarization
methods. In this section we discuss the use of these different
methods.

The methods differ in their objectives: \imp{confidence
interval} for central value versus \imp{prediction intervals}.
The former quantify the accuracy of the estimated central
value, the latter reflects how variable the data is. Both
aspects are related (the more variable the data is, the less
accurate the estimated central value is) but they are not the
same.

The methods differ in the techniques used, and
overlap to a large extend. They fall in two
categories: methods based on the order statistic
(confidence interval for median or or other
quantiles, \thref{theo-conf-ci-median};
prediction interval computed with order
statistic,  \thref{theo-conf-pi}) or based on
mean and standard deviation
(Theorems~\ref{theo-conf-normal},
\ref{theo-conf-ln}, \ref{theo-conf-pin}). The two
types of methods differ in their \imp{robustness
versus compactness}.

\subsection{Robustness}

\subsubsection{Wrong Distributional Hypotheses}
\label{sec-verify-asymptotic}
The confidence interval for the mean given by
\thref{theo-conf-ln} requires that the central
limit theorem applies  i.e. (1) the common
distribution has a finite variance and (2) the
sample size $n$ is large enough. While these two
assumptions very often hold, it is important to
detect cases where they do not.

Ideally, we would like to test whether the
distribution of $T=\sum_{i=1}^{n} X_i$ is normal
or not, but we cannot do this directly, since we
have only one value of $T$. The bootstrap method
can be used to solve this problem, as explained
in the next example.


\begin{figure}\center
 \subfigure[]{\Ifignc{bsPareto1.25-100}{0.8}{0.25}}
 \subfigure[]{\Ifignc{bsPareto1.25-10000}{0.85}{0.25}}
\mycaption{(a) Left: Artificially generated
sample of 100 values from a Pareto distribution
with exponent $p=1.25$. Center: confidence
intervals for the mean computed from
\thref{theo-conf-ln} (left) and the bootstrap
percentile estimate (center), and confidence
interval for the median (right). Right: qqplot of
999 bootstrap replicates of the mean. The qqplot
shows deviation from normality, thus the
confidence interval given by \thref{theo-conf-ln}
is not correct. Note that in this case the
bootstrap percentile interval is not very good
either, since it fails to capture the true value
of the mean ($=5$). In contrast, the confidence
interval for the median does capture the true
value ($=1.74$). (b) Same with 10000 samples. The
true mean is now within the confidence interval,
but there is still no convergence to normality.}
\mylabel{fig-conf-bs1}
\end{figure}

\begin{ex}{Pareto Distribution} \mylabel{ex-conf-bs-pareto} This is a toy example where we generate
artificial data, iid, from a Pareto
distribution on $[1,+\infty)$. It is defined by
its cdf equal to $F(c):=\P(X >
c)=\frac{1}{c^{p}}$ with $p=1.25$; its mean is
$=5$, its variance is infinite (i.e. it is heavy
tailed) and its median is $1.74$.

Assume we would not know that it comes from a
heavy tailed distribution and would like to use
the asymptotic result in \thref{theo-conf-ln} to
compute a confidence interval for the mean.

We use the bootstrap method to verify convergence
to the normal distribution, as follows. We are
given a data sample $x_1, ...,x_n$ from the
Pareto distribution. We generate $R$ replay
experiments: for each $r$ between $1$ and $R$, we
draw $n$ samples $X^r_i$ $i=1,...,n$ with
replacement from the list $(x_1, ..., x_n)$ and
let $T^r=\frac{i=1}{n}X^r_i$. $T^r$ is the $r$th
bootstrap replicate of $T$; we do a qqplot of the
$T^r, r=1,...,R$. If the distribution of $T$ is
normal, the qqplot should look normal as well.

We see that the qqplots do not appear normal,
which is an indication that the central limit
theorem might not hold. Indeed, the confidence
interval for the mean is not very good.
\end{ex}



The previous example shows a case where the
confidence interval for the mean is not good,
because a distributional assumption was made,
which is not correct. In contrast, the confidence
interval for the median \imp{is} correct
(\fref{fig-conf-bs1}), as it does not require any
distributional assumption (other than the iid
hypothesis).
%
%
\subsubsection{Outliers}
%
\begin{figure}
\center
 \subfigure[(Data without outlier)]{\Ifignc{pif7}{0.3}{0.2}}
 \subfigure[(QQ-plot of (a))]{\Ifignc{pif8}{0.3}{0.2}}\\
 \subfigure[(Log of data without outlier)]{\Ifignc{pif9}{0.3}{0.2}}
 \subfigure[(QQ-plot of (c))]{\Ifignc{pif10}{0.3}{0.2}}\\
 \subfigure[(Confidence Intervals)]{\Ifignc{pif14}{0.4}{0.3}}
 \subfigure[(Prediction Intervals)]{\Ifignc{pif13}{0.4}{0.3}}
  \mycaption{File transfer times for 100 independent simulation runs with outlier removed. Confidence
  intervals are without (left) and with (right) outlier, and with method (1) median (2) mean and (3) geometric mean.
  Prediction intervals are without (left) and with (right)
  outlier,
   computed with the three alternative methods discussed in \exref{ex-conf-ft1}: (1) order statistics
   (2) based on mean and standard deviation (3) based on mean and standard deviation after
  re-scaling.}
  \mylabel{fig-conf-ft2}
\end{figure}
%
Methods based on the order statistic are more
robust to outliers. An \nt{outlier} is a value
that significantly differs from the average. The
median and the prediction interval based on order
statistic are not affected by a few outliers,
contrary to the mean and the prediction interval
based on mean and standard deviation, as
illustrated by the following example.
%
\begin{ex}{File Transfer with One Outlier}\mylabel{ex-conf-ft1}
In fact in the data of \exref{ex-conf-ft1} there is one very
large value, 5 times larger than the next largest value. One
might be tempted to remove it, on the basis that such a large
value might be due to measurement error. A qqplot of the data
without this ``outlier" is shown on \fref{fig-conf-ft2},
compare to the corresponding qq-plot with the outlier in
\fref{fig-conf-qq}~(a,b). The prediction intervals based on
order statistics are not affected, but the one based on mean
and standard deviation is completely different.

\tref{tab-ci-fi} shows the values of Jain's fairness index and
the Lorenz curve gap is very sensitive to the presence of one
outlier, which is consistent with the previous observation
since Jain's fairness index is defined by the ratio of standard
deviation to mean (coefficient of variation). The Lorenz curve
gap is less sensitive.

The outlier is less of an outlier on the re-scaled data (with
the log transformation). The qqplot of the rescaled data is not
affected very much, neither is the prediction interval based on
mean and standard deviation of the rescaled data. Similarly,
the confidence intervals for median and geometric mean are not
affected, whereas that for the mean is. We do not show fairness
indices for the re-scaled data since re-scaling changes the
meaning of these indices.
\end{ex}

Care should be taken to screen the data
collection procedure for \imp{true outliers},
namely values that are wrong because of
measurement errors or problems. In the previous
example, we should not remove the outlier. In
practice it may be difficult to differentiate
between true and spurious outliers. The example
illustrates the following facts:
\begin{itemize}
    \item Outliers may affect the prediction and confidence
        intervals based on mean and standard deviation, as
        well as the values of fairness indices. Jain's
        fairness index is more sensitive than the Lorenz
        curve gap.
    \item This may go away if the data is properly
        rescaled. An outlier in some scale may not be an
        outlier in some other scale.
    \item In contrast, confidence intervals for the median
        and prediction intervals based on order statistics
        are more robust to outliers. They are not affected
        by re-scaling.
\end{itemize}
%
\begin{table}
  \centering
  \begin{tabular}{|c|c|r|c|l|}
    % after \\: \hline or \cline{col1-col2} \cline{col3-col4} ...
 \hline
      & Index & Lower Bound, CI & Index & Upper Bound, CI \\ \hline
      \hline
    Without Outlier & JFI & 0.1012  &  0.1477  &  0.3079 \\ \cline{2-5}
      & gap & 0.4681  &  0.5930   & 0.6903 \\ \hline
    With  Outlier & JFI & 0.0293  &  0.0462  &  0.3419 \\ \cline{2-5}
      & gap &  0.4691  &  0.6858  &  0.8116 \\ \hline
  \end{tabular}
  \mycaption{Fairness indices with and without outlier.}\label{tab-ci-fi}
\end{table}
%
%
\subsection{Compactness}
Assume we wish to obtain both a central value with confidence
interval and a prediction interval for a given data set. If we
use methods based on order statistics, we will obtain a
confidence interval for the median, and, say, a prediction
interval at level $95\%$. Variability and accuracy are given by
different sample quantiles, and cannot be deduced from one
another.  Furthermore, if we later are interested in $99\%$
prediction intervals rather than 95\%, we need to recompute new
estimates of the quantiles. The same argument speaks in favour
of quantifying the variability by means of the Lorenz curve
gap.

In contrast, if we use methods based on mean and standard
deviation, we obtain both confidence intervals and prediction
intervals at any level with just 2 parameters (the sample mean
and the sample standard deviation). In particular, the sample
standard deviation gives indication on both accuracy of the
estimator and variability of the data. However, as we saw
earlier, these estimators are meaningful only in a scale where
the data is roughly normal, if there is any.

Also, mean and standard deviation are less
complex to compute than estimators based on order
statistics, which require sorting the data. In
particular, mean and standard deviation can be
computed incrementally online, by keeping only 2
counters (sum of values and sum of squares). This
reason is less valid today than some years ago,
since there are sorting algorithms with
complexity $ n \ln(n)$.

\section{Other Aspects of Confidence/Prediction Intervals}
\subsection{Intersection of Confidence/Prediction Intervals} In
some cases we have several confidence or prediction intervals
for the same quantity of interest. For example, we can have a
prediction interval $I$ based on mean and standard deviation or
$I'$ based on order statistics. A natural deduction is to
consider that the intersection $I \cap I'$ is a better
confidence interval. This is almost true:

\begin{shadethm}
If the random intervals $I$, $I'$ are some
confidence intervals at level $\gamma=1-\alpha$,
$\gamma'=1-\alpha'$ then the intersection $I \cap
I'$ is a confidence interval at level at least
$1-\alpha-\alpha'$. The same holds for prediction
intervals.\label{theo-inter-ci}
\end{shadethm}

\begin{ex}{File Transfer Times}(Continuation of
\exref{ex-conf-ft1}). We can compute two prediction intervals
at level 0.975, using the order statistic method and the mean
and standard deviation after rescaling (the prediction obtained
without rescaling is not valid since the data is not normal).
We obtain $[0.0394,336.9]$ and $[0.0464,392.7]$. We can
conclude that a prediction interval at level $0.95$ is
$[0.0464,  336.9]$, which is better than the two.

Compare this interval to the prediction intervals at level
$95\%$ for each of the two methods; they are $[0.0624, 205.6]$
and $[0.0828, 219.9]$. Both are better.
\end{ex}

Thus, for example if we combine two confidence intervals at
level $97.5\%$ we obtain a confidence interval at level $95\%$.
As the example shows, this may be less good than an original
confidence interval at level $95\%$.

\subsection{The Meaning of Confidence}
When we say that an interval $I$ is a confidence interval at
level 0.95 for some parameter $\theta$, we mean the following.
If we could repeat the experiment many times, in about 95\% of
the cases, the interval $I$ would indeed contain the true value
$\theta$.

\mq{stats-q711}
 {Assume $1000$ students independently perform a simulation of an M/M/1 queue with
  load factor $\rho=0.9$ and find a $95\%$ confidence interval for the result. The true result,
  unknown to these (unsophisticated) students is $9$.
  The students are unsophisticated but conscientious, and all did correct simulations.
  How many of the 1000 students do you
  expect to find a wrong confidence interval, namely one that does \emph{not} contain the true value ?
 }
 {Approximately 50 students should find a wrong interval.
 }


%\section{Parametric Estimation Theory}
\mylabel{sec-conf-mle}  In this appendix we give a large sample
theory which is used for some asymptotic confidence interval
computations in \cref{ch-conf} and for the general framework of
likelihood ratio tests of \cref{ch-tests}.

\subsection{The Parametric Estimation Framework. }

Consider a data set $x_i$, $i=1...,n$, which we view as the
realization of a stochastic system (in other words, the output
of a simulator). The framework of parametric estimation theory
consists in assuming that the parameters of the stochastic
system are well defined, but unknown to the observer, who tries
to estimate it as well as she can, using the data set.

We assume here that the model has a density of probability,
denoted with $f(x_1,...,x_n|\theta)$, where $\theta$ is the
parameter. It is also called the \nt{likelihood} of the
observed data. An \nt{estimator} of $\theta$ is any function
$T()$ of the observed data. A good estimator is one such that,
in average, $T(x_1,...,x_n)$ is ``close" to the true value
$\theta$.

\begin{ex}{i.i.d. Normal Data} Assume we can believe that our data is
iid and normal with mean $\mu$ and variance $\sigma^2$. The
likelihood is \be
 \frac{1}{\left(\sqrt{2
\pi}\sigma\right)^n}\exp\left(
-\frac{1}{2}\sum_{i=1}^n\frac{(x_i-\mu)^2}{\sigma^2} \right)
 \mylabel{eq-mle-niid}
 \ee
 and $\theta=(\mu,\sigma)$. An estimator of $\theta$ is
$\hat{\theta}=(\hat{\mu}_n, \hat{\sigma}_n)$ given by
\thref{theo-conf-normal}. Another, slightly different estimator
is $\hat{\theta}_1=(\hat{\mu}_n, s_n)$ given by
\thref{theo-conf-ln}.
\end{ex}

An estimator provides a random result: for every realization of
the data set, a different estimation is produced. The
``goodness" of an estimator is captured by the following
definitions. Here $\vX$ is the random data set,  $T(\vX)$ is
the estimator and $\E_{\theta}$ means the expectation when the
unknown but fixed parameter value is $\theta$.
\begin{itemize}
    \item \nt{Unbiased estimator}:
        $\E_{\theta}\left(T(\vX)\right)=\theta$. For
        example, the estimator $\hat{\sigma}^2_n$ of
        variance of a normal i.i.d. sample given by
        \thref{theo-conf-normal} is unbiased.
    \item \nt{Consistent family of estimators}:
        $\P_{\theta}(|T(\vX)-\theta|)> \epsilon)
        \rightarrow 0$ when the sample size $n$ goes to
        $\infty$. For example, the estimator $(\hat{\mu}_n,
        \hat{\sigma}_n)$ of \thref{theo-conf-normal} is
        consistent. This follows from the weak law of large
        numbers.
\end{itemize}

\subsection{Maximum Likelihood Estimator (MLE)\index{MLE}}
\label{sec-mle-defll} A commonly used method for
deriving estimators is that of \nt{Maximum
Likelihood}. The maximum likelihood estimator is
the value of $\theta$ that maximizes the
likelihood $f(x_1,...,x_n|\theta)$. This
definition makes sense if the maximum exists and
is unique, which is often true in practice. A
formal set of conditions is the regularity
condition in Definition~\ref{def-mle}.

In \sref{sec-mle-as}, we give a result that shows that the MLE
for an i.i.d. sample with finite variance is asymptotically
unbiased, i.e. the bias tends to $0$ as the sample size
increases. It is also consistent.

\begin{ex}{MLE for i.i.d. normal data}
Consider a sample $(x_1, ..., x_n)$ obtained from a normal
i.i.d. random vector $(X_1, ...,X_n)$. The likelihood is given
by \eref{eq-mle-niid}. We want to maximize it, where
 $x_1,...,x_n$ are given and $\mu, v=\sigma^2$ are the variables. For
 a given $v$, the maximum is reached when $\mu=\hat{\mu}_n=\frac{1}{n}\sum_{i=1}^n
 x_i$. Let $\mu$ have this value and find the value of $v$
 that maximizes the resulting expression, or to simplify, the log of it. We thus have to  maximize
 \be
-\frac{n}{2} \ln v -\frac{1}{2v} S_{x,x} + C
 \ee where
 $S_{x,x}\eqdef\sum_{i=1}^n(x-\hat{\mu}_n)^2$\index{1sxx@$S_{x,x}$} and $C$ is a constant with respect to $v$. This is a simple maximization problem in one variable $v$,
 which can be solved by computing the derivative. We find that
 there is a maximum for
 $v=\frac{S_{x,x}}{n}$. The maximum likelihood
 estimator of $(\mu,v)$ is thus precisely the estimator in
 \thref{theo-conf-ln}.
\end{ex}

We say that an estimation method \nt{invariant by
re-parametrization} if a different parametrization gives
essentially the same estimator. More precisely assume we have a
method which produces some estimator $T(\vX)$ for $\theta$.
 Assume we re-parametrize the problem by considering that the parameter is
 $\phi(\theta)$, where $\phi$ is some
 invertible mapping. For example, a normal i.i.d. sample can be parametrized by
 $\theta=(\mu, v)$ or by
 $\phi(\theta)=(\mu, \sigma)$, with $v=\sigma^2$.
%\mq{q-conf-sksak}{What is the mapping $\phi$ in this
%case~?}{$\phi(x,y)=(x,\sqrt{y})$ defined for $x \in \Reals$ and
%$y \geq 0$.}
The method is called invariant by re-parametrization if the
estimator of $\phi(\theta)$ is precisely $\phi(T(\vX))$.

The maximum likelihood method \emph{is} invariant by
re-parametrization. This is because the property of being a
maximum is invariant by re-parametrization. It is an important
property in our context, since the model is usually not given a
priori, but has to be invented by the performance analyst.

A method that provides an unbiased estimator cannot be invariant
by re-parametrization, in general. For example, $(\hat{\mu}_n,
\hat{\sigma}_n^2)$ of \thref{theo-conf-normal} is an unbiased
estimator of $(\mu, \sigma^2)$, but $(\hat{\mu}_n,
\hat{\sigma}_n)$ is a \imp{biased} estimator of $(\mu, \sigma)$
(because usually $\E(S)^2 \ne \E(S^2)$ except if $S$ is
non-random). Thus, the property of being unbiased is incompatible
with invariance by re-parametrization, and may thus be seen as an
inadequate requirement for an estimator.

Furthermore, maximum likelihood is also \nt{invariant by
reversible data transformation}, i.e. the MLE of $\theta$ is
the same, whether we look at the data or at a one to one
transform, independent of $\theta$. More precisely, assume
$\vX=(X_i)_{i=1...n}$ has a joint PDF $f_{\vX}(\vx|\theta)$,
and let $\vY=\varphi(\vX)$, with $\varphi$ a one-to-one,
differentiable mapping independent of $\theta$.

Take $\vX$ as data and estimate $\theta$; we have to maximize
$f_{\vX}(\vx|\theta)$ with respect to $\theta$, where
$\vx=(x_i)_{i=1...n}$ is the available data. If, instead, we
observe $y_i= \varphi(x_i)$ for all $i$, we have to maximize
 \ben
 f_Y(\vy) = \frac{1}{\abs{\varphi'(\vx)}}f_{\vX}(\vx)
 \een where $\abs{\varphi'(\vx)}$ is the absolute
 value of the determinant of the differential of
 $\varphi$ (i.e. the Jacobian matrix).

In particular, the MLE is invariant by \emph{re-scaling} of the
data. For example, if $Y_i$ is a log-normal sample (i.e. if
$Y_i=e^{X_i}$ and $X_i\sim$ iid $N_{\mu,\sigma^2}$), then the
MLE of the parameters $\mu,\theta$ can be obtained by
estimating the mean and standard deviation of $\ln(Y_i)$.

\subsection{Efficiency and Fisher Information}
The \nt{efficiency} of an estimator $T(\vX)$ of the parameter
$\theta$ is defined as the expected square error
$\E_{\theta}(\norm{T(\vX)-\theta}^2)$ (here we assume that
$\theta$ takes values in some space $\Theta$ where the norm is
defined). The efficiency that can be reached by an estimator is
captured by the concept of Fisher information, which we now
define.
%
Assume first to simplify that $\theta \in \Reals$. The
\nt{observed information} is defined by
$$
J(\theta)=-\frac{\partial^2 l(\theta)}{\partial \theta^2}
$$
where $l(\theta)$ is the \nt{log-likelihood}, defined by
$$
l(\theta) = \ln \mbox{lik}(\theta) = \ln f(x_1,...,x_n|\theta)
$$
The \nt{Fisher information}, or \nt{expected information} is
defined by
$$
I(\theta)
=\E_{\theta}(J(\theta))=\E_{\theta}\left(-\frac{\partial^2
l(\theta)}{\partial \theta^2}\right)
$$
For an i.i.d. model $X_1, ...,X_n$, $l(\theta)=\sum_i \ln
f_1(x_i|\theta)$ and thus $I(\theta)=n I_1(\theta)$, where
$I_1(\theta)$ is the Fisher information for a one point sample
$X_1$. The Cramer-Rao theorem says that the efficiency of any
\imp{unbiased} estimator is lower bounded by
$\frac{1}{I(\theta)}$. Further, under the conditions in
Definition~\ref{def-mle}, the MLE for an i.i.d. sample is
asymptotically maximally efficient, i.e.
$\E\left(\norm{T(\vX)-\theta}\right)/I(\theta)$ tends to $1$ as
the sample size goes to infinity.

In general, the parameter $\theta$ is multi-dimensional, i.e.,
varies in an open subset $\Theta$ of $\Reals^k$. Then $J$ and $I$
are symmetric matrices defined by
$$
[J(\theta)]_{i,j}=-\frac{\partial^2 l(\theta)}{\partial \theta_i
\partial \theta_j}
$$
and
$$
[I(\theta)]_{i,j}=-\E_{\theta}\left(\frac{\partial^2
l(\theta)}{\partial \theta_i \partial \theta_j}\right)
$$

The Cramer-Rao lower bound justifies the name of
``information". The variance of the MLE is of the order of the
Fisher information: the higher the information, the more the
sample tells us about the unknown parameter $\theta$. The
Fisher information is not the same as entropy, used in
information theory. There are some (complicated) relations --
see \cite[Chapter 16]{cover1991elements}.

In the next section we give a more accurate result, which can
be used to give approximate confidence intervals for large
sample sizes.

\section{Asymptotic Confidence Intervals} \mylabel{sec-mle-as}
Here we need to assume some regularity
conditions. Assume the sample comes from an
i.i.d. sequence of length $n$ and further, that
the following regularity conditions are met.
\begin{definition}\mylabel{def-mle}
Regularity Conditions for Maximum Likelihood Asymptotics
\cite{davison2003sm}.
\begin{enumerate}
  \item The set $\Theta$ of values of $\theta$ is compact
  (closed and bounded) and the true value $\theta_0$ is not on
  the boundary.
  \item (identifiability) for different values of $\theta$,
      the densities $f(\vx|\theta)$ are different.
  \item (regularity of derivatives) There exist a
      neighborhood $B$ of $\theta_0$ and a constant $K$
      such that for $\theta \in B$  and for all $i,j,k,n$~:
      $\frac{1}{n}\E_{\theta}(
      \left|\partial^3l_{\vX}(\theta)/\partial\theta_i\partial\theta_j\partial\theta_k\right|)\leq
      K$
  \item For
  $\theta \in B$ the Fisher information has
  full rank
  \item For $\theta \in B$ the interchanges of integration
      and derivation in $\int \frac{\partial
      f(\vx|\theta)}{\partial \theta_i}  dx= \frac{\partial
      }{\partial \theta_i} \int f(\vx|\theta) dx$ and $
      \int \frac{\partial^2 f(x|\theta)}{\partial \theta_i
      \partial \theta_j}  dx= \frac{\partial }{\partial
      \theta_i} \int \frac{\partial
      f({\vx}|\theta)}{\partial \theta_j} dx $ are valid
  \end{enumerate}
\end{definition}
The following theorem is proven in \cite{davison2003sm}.

\begin{shadethm}\mylabel{theo-mle}
Under the conditions in \dref{def-mle}, the MLE exists, converges
almost surely to the true value. Further $I(\theta)^{\frac{1}{2}}
(\hat{\theta} - \theta)$ converges in distribution towards a
standard normal distribution, as $n$ goes to infinity. It follows
that, asymptotically:
\begin{enumerate}
  \item the distribution of $\hat{\theta}-\theta$ can be approximated by
$N\left(0, I(\hat{\theta})^{-1}\right)$ or $N\left(0,
J(\hat{\theta})^{-1}\right)$
  \item the distribution of $2\left(l(\hat{\theta})-l(\theta)\right)$ can be approximated by  $\chi^2_k$  (where $k$ is the dimension
  of $\Theta$).
\end{enumerate}
\end{shadethm}
The quantity $2\left(l(\hat{\theta})-l(\theta)\right)$ is called
the \nt{likelihood ratio statistic}.

%\begin{petiteNote}
In the examples seen in this book, the regularity conditions
are always satisfied, as long as : the true value $\theta$ lies
within the interior of its domain, the derivatives of
$l(\theta)$ are smooth (for example, if the density
$f({\vx}|\theta)$ has derivatives at all orders) and the
matrices $J(\theta)$ and $I(\theta)$ have full rank. If the
regularity conditions hold, then we have an equivalent
definition of Fisher information:
$$
[I(\theta)]_{i,j}\eqdef-\E_{\theta}\left(\frac{\partial^2
l(\theta)}{\partial \theta_i \partial \theta_j}\right)=
\E_{\theta}\left(\frac{\partial l(\theta)}{\partial \theta_i}
\frac{\partial l(\theta)}{\partial \theta_j}\right)
$$
this follows from differentiating with respect to $\theta$ the
identity $\int f(x|\theta) d\!x=1$.

Item 2 is more approximate than item 1, but does not require to
compute the second derivative of the likelihood.

\thref{theo-mle} also holds for non-i.i.d. cases, as long as
the Fisher information goes to infinity with the sample size.
%\end{petiteNote}
%
%\mq{q-stat-asdasklksdk}{\thref{theo-mle} provides two asymptotic
%pivots. What are they~?}{$I(\theta)^{\frac{1}{2}} (\hat{\theta} -
%\theta)$ and $2\left(l(\hat{\theta})-l(\theta)\right)$.}

\begin{ex}{Fisher Information of Normal i.i.d. Model}
Assume $(X_i)_{i=1\ldots n}$ is i.i.d. normal with mean $\mu$
and variance $\sigma^2$. The observed information matrix is
computed from the likelihood function; we obtain:
$$J=\left(\begin{array}{cc}
 \frac{n}{\sigma^2} & \frac{2n}{\sigma^3}(\hat{\mu}_n-\mu)\\
  \frac{2n}{\sigma^3}(\hat{\mu}_n-\mu) & \frac{-n}{\sigma^2}+ \frac{3}{\sigma^4}\left(S_{xx}+n(\hat{\mu}_n-\mu)^2\right)
\end{array}\right)
$$
 and the expected information matrix (Fisher's information) is
$$I=\left(\begin{array}{cc}
 \frac{n}{\sigma^2} & 0\\
 0 & \frac{2n}{\sigma^2}
\end{array}\right)
$$
\end{ex}

The following corollary is used in practice. It follows
immediately from the theorem.
\begin{corollary}[Asymptotic Confidence Intervals] When $n$
is large, approximate confidence intervals can be obtained as
follows:\begin{enumerate}
    \item For the $i$th coordinate of $\theta$, the interval is: $\hat{\theta}_i \pm \eta \sqrt{\left[I(\hat{\theta})^{-1}\right]_{i,i}}$ or
     $\hat{\theta} \pm \eta \sqrt{\left[J(\hat{\theta})^{-1}\right]_{i,i}}$, where
    $N_{0,1}(\eta)=\frac{1+\gamma}{2}$ (for example, with $\gamma=0.95$,
$\eta=1.96$).
    \item If $\theta$ is in $\Reals$: the interval can be defined implicitly as $\{\theta\; : \; l(\hat{\theta})- \frac{\xi}{2} \leq l(\theta) \leq   l(\hat{\theta}) \}$,
    where $\chi^2_1( \xi)= \gamma$. For example, with $\gamma=0.95$, $\xi=3.84$.
\end{enumerate}
\mylabel{cor-conf-asfis}
\end{corollary}

\begin{ex}{Lazy Normal i.i.d.} Assume our data comes from an i.i.d.
normal model $X_i$, $i=1,...n$. We compare the exact confidence
interval for the mean (from \thref{theo-conf-normal}) to the
approximate ones given by the corollary.

The MLE of $(\mu,\sigma)$ is $(\hat{\mu}_n, s_n)$. The exact
confidence interval is \ben \hat{\mu}_n \pm \eta'
\frac{\hat{\sigma}_n}{\sqrt{n}}\een with $\hat{\sigma}^2_n=S_{x,x}
/ (n-1)$ and $t_{n-1}(\eta')=\frac{1+\gamma}{2}$.

Now we compute the approximate confidence interval obtained from
the Fisher information. We have
 \ben
 I(\mu,\sigma)^{-1}=\left(\begin{array}{cc}
 \frac{\sigma^2}{n} & 0\\
 0 & \frac{\sigma^2}{2n}
\end{array}\right)
 \een
thus the distribution of $(\mu-\hat{\mu}_n,\sigma-s_n)$ is
approximately normal with $0$ mean and covariance matrix
$\left(\begin{array}{cc}
 \frac{\sigma^2}{n} & 0\\
 0 & \frac{\sigma^2}{2n}
\end{array}\right)$. It follows that
$\mu-\hat{\mu}_n$ is approximately $N(0,\frac{s_n^2}{n})$, and an
approximate confidence interval is
 \ben
\hat{\mu}_n \pm \eta \frac{s_n}{\sqrt{n}}
 \een
with $s_n=s_{x,x}/n$ and $N_{0,1}(\eta)=\frac{1+\gamma}{2}$.

Thus the use of Fisher information gives the same asymptotic
interval for the mean as \thref{theo-conf-ln}. This is quite
general: the use of Fisher information is the generalization of
the large sample asymptotic of \thref{theo-conf-ln}.

We can also compare the approximate confidence interval for
$\sigma$. The exact interval is given by \thref{theo-conf-normal}:
with probability $\gamma$ we have
 \ben
  \frac{\xi_2}{n-1} \leq \frac{\hat{\sigma}^n_2}{\sigma^2} \leq \frac{\xi_1}{n-1}
 \een
 with $\chi^2_{n-1}(\xi_2)=\frac{1-\gamma}{2}$ and
 $\chi^2_{n-1}(\xi_1)=\frac{1+\gamma}{2}$. Thus an exact confidence
 interval for $\sigma$ is
  \be
\hat{\sigma}_n \; \left[\sqrt{\frac{n-1}{\xi_1}},\;
\sqrt{\frac{n-1}{\xi_2}}\right]
  \ee

With Fisher information, we have that $\sigma -s_n$ is
approximately $N_{0,\frac{\sigma^2}{2n}}$ Thus with probability
$\gamma$
 \ben |\sigma - s_n | \leq \eta \frac{\sigma}{\sqrt{2n}}
 \een
 with $N_{0,1}(\eta)=\frac{1+\gamma}{2}$ .

Divide by $\sigma$ and obtain, after some algebra, that with
probability $\gamma$:
 \ben
 \frac{1}{1 + \frac{\eta}{\sqrt{2n}}} \leq
 \frac{\sigma}{s_n} \leq
 \frac{1}{1 - \frac{\eta}{\sqrt{2n}}}
 \een
Taking into account that $s_n=\sqrt{\frac{n-1}{n}}\hat{\sigma}_n$,
we obtain the approximate confidence interval for $\sigma$
 \be
 \hat{\sigma}_n \;
 \left[
 \sqrt{\frac{n-1}{n}}\frac{1}{1 + \frac{\eta}{\sqrt{2n}}},
 \;
\sqrt{\frac{n-1}{n}}\frac{1}{1 - \frac{\eta}{\sqrt{2n}}},
 \right]
 \ee

 For $n=30,60,120$ and $\gamma=0.95$, the confidence intervals are
 as shown in \tref{tab-conf-nlz}, where we compare to exact values;
 the difference is negligible
 already for $n=30$.
\begin{table}
  \centering
  \begin{tabular}{|l||c|c|c|}\hline
    % after \\: \hline or \cline{col1-col2} \cline{col3-col4} ...
    $n$  & 30 & 60 & 120 \\ \hline
    Exact  & $0.7964 -  1.3443$& $0.8476 - 1.2197$& $0.8875 - 1.1454$\\
    Fisher & $0.7847 -  1.3162$& $0.8411 - 1.2077$& $0.8840 - 1.1401$\\
    \hline
  \end{tabular}
  \mycaption{Confidence Interval for $\sigma$ for an i.i.d., normal sample
of $n$ data points by exact method and asymptotic result with
Fisher information (\coref{cor-conf-asfis}). The values are the
confidence bounds for the ratio $\frac{\sigma}{\hat{\sigma}_n }$
where $\sigma$ is the true value and $\hat{\sigma}_n $ the
estimated standard deviation as in \thref{theo-conf-normal}.}
  \mylabel{tab-conf-nlz}
\end{table}
\end{ex}
\mq{stats-qmle200}
 {Which of the following are random variables: $\hat{\theta}$, $\theta$,
  $l(\theta)$, $l(\hat{\theta})$,  $J(\theta)$, $I(\theta)$, $J(\hat{\theta})$, $I(\hat{\theta})$~?
 }
 {In the classical,
  non Bayesian framework: $\hat{\theta}$, $l(\theta)$, $l(\hat{\theta})$, $J(\theta)$, $J(\hat{\theta})$,
  $I(\hat{\theta})$ are random variables; $\theta$ and $I(\theta)$ are non-random but unknown.
 }



 \section{Confidence Interval in Presence of Nuisance Parameters}
 \mylabel{sec-conf-nuisance}
 In many cases, the parameter has the form $\theta=(\mu, \nu)$, and we are interested only in $\mu$ (for example, for a
 normal model: the mean) while the remaining element $\nu$, which still needs to be estimated, is
 considered a nuisance (for example: the variance). In such cases, we can use the following
 theorem to find confidence intervals.
 \begin{shadethm}[\cite{davison2003sm}]\mylabel{theo-mle2}
 Under the conditions in \dref{def-mle}, assume that $\Theta = M \times
 N$, where $M, N$ are open subsets of $\Reals^p, \Reals^q$. Thus the parameter is $\theta=(\mu, \nu)$
 with $\mu \in M$ and $\nu \in N$ ($p$ is the ``dimension", or number of degrees of freedom, of
 $\mu$).

 For any $\mu$, let $\hat{\nu}_{\mu}$ be the
 solution to
 $$ l(\mu,\hat{\nu}_{\mu})=\max_{\nu}l(\mu, \nu)$$
  and define the \nt{profile log likelihood} $pl$ by
 $$
 pl(\mu)\eqdef\max_{\nu} l(\mu,\nu)= l(\mu, \hat{\nu}_{\mu})$$
Let $(\hat{\mu},\hat{\nu})$ be the MLE. If $(\mu,\nu)$ is the true
value of the parameter, the distribution of
$2\left(pl(\hat{\mu})-pl(\mu)\right)$ tends to $\chi^2_p$.

An approximate confidence region for $\mu$ at level $\gamma$ is
$$
\{\mu \in M: pl(\mu) \geq pl(\hat{\mu})-\frac{1}{2}\xi\}
$$
where $\chi^2_p(\xi)=\gamma$.
\end{shadethm} The theorem essentially says that we can find an approximate confidence
interval for the parameter of interest $\mu$ by computing the
profile log-likelihood for all values of $\mu$ around the
estimated value. The estimated value is the one that maximizes the
profile log-likelihood. The profile log likelihood is obtained by
fixing the parameter of interest $\mu$ to some arbitrary value and
compute the MLE for the other parameters. A confidence interval is
obtained implicitly as the set of values of $\mu$ for which the
profile log likelihood is close to the maximum. In practice, all
of this is done numerically.

\begin{ex}{Lazy Normal i.i.d. Revisited}
Consider the log of the data in \fref{fig-conf-ft1a}, which
appears to be normal. The model is $Y_i \sim i.i.d.
N_{\mu,\sigma^2}$ where $Y_i$ is the log of the data. Assume we
would like to compute a confidence interval for $\mu$ but are
too lazy to apply the exact student statistic in
\thref{theo-conf-normal}.

For any $\mu$, we estimate the nuisance parameter $\sigma$, by
maximizing the log-likelihood: \ben
l(\mu,\sigma)=-\frac{1}{2}\left(n \ln \sigma^2 +
\frac{1}{\sigma^2}\sum_i (Y_i-\mu)^2 \right)
 \een
It comes
$$
\hat{\sigma}^2_{\mu}=\frac{1}{n}\sum_i
(Y_i-\mu)^2=\frac{1}{n}S_{YY}+(\bar{Y}-\mu)^2
$$
and thus
 $$
 pl(\mu)\eqdef l(\mu,\hat{\sigma}_{\mu})=-\frac{n}{2}(\ln\hat{\sigma}_{\mu}^2 + 1)
 $$
On \fref{fig-conf-lazynorm} we plot $pl(\mu)$. We find
$\hat{\mu}=\input{lazyNormhatmu}$ as the point that maximizes
$pl(\mu)$. A $95\%$-confidence interval is obtained as the set
$\{pl(\mu) \geq pl(\hat{\mu})-\frac{1}{2}3.84\}$. We obtain the
interval $[\input{lazyNormmuci}]$. Compare to the exact confidence
interval obtained with \thref{theo-conf-normal}, which is equal to
$[ 1.103 ,  1.918]$: the difference is negligible.

 \mq{q-mle2klkls9}
 {Find an analytical expression of the confidence interval obtained with the profile
 log likelihood for this example and compare with the exact interval.}
{The profile log likelihood method gives a confidence interval
defined by
$$
\frac{(\hat{\mu}-\mu)^2}{\frac{S_{YY}}{n}} \leq
e^{\frac{\eta}{n}}-1\approx \frac{\eta}{n}
$$
Let $t\eqdef\frac{\hat{\mu}-\mu}{\sqrt{\frac{S_{YY}}{n(n-1)}}}$
be the student statistic. The asymptotic confidence interval
can be rewritten as
 $$
t^2 \leq (n-1)(e^{\frac{\eta}{n}}-1)\approx \frac{\eta (n-1)}{n}
 $$
 An exact confidence interval is
 $$
t^2 \leq \xi^2
 $$
 where $\xi=t_{n-1}(1-\alpha/2)$. For large $n$, $\xi^2 \approx \eta$ and $\frac{n-1}{n} \approx
 1$ so the two intervals are equivalent.
 }
 \end{ex}
\begin{figure}[!htbp]
  \Ifig{lazyNorm}{0.6}{0.4}%
\nfs{fileTransferLogNormal.dat lazyn.m}%
  \mycaption{Profile log-likelihood for parameter $\mu$ of the log of the data in \fref{fig-conf-ft1a}.
  The confidence interval for $\mu$ is obtained by application of \thref{theo-mle2}.}
  \mylabel{fig-conf-lazynorm}
\end{figure}


\begin{ex}{Re-Scaling}
\mylabel{ex-conf-pl} Consider the data in \fref{fig-conf-ft1a},
which does not appear to be normal in natural scale, and for which
we would like to do a Box-Cox transformation. We would like a
confidence interval for the exponent of the transformation.

The transformed data is $Y_i= b_s (X_i)$, and the model now
assumes that $Y_i$ is i.i.d. $\sim N_{\mu,\sigma^2}$. We take
the unknown parameter to be  $\theta=(\mu, \sigma, s)$. The
distribution of $X_i$, under $\theta$ is:
  \ben f_{X_i}(x|\theta)=
  b_s'(x) f_{Y_i}\left(b_s(x)|\mu,\sigma\right)=x^{s-1}h(b_s(x)|\mu,\sigma^2)
  \een
where $h(x|\mu,\sigma^2)$ is the density of the normal
distribution with mean $\mu$ and variance $\sigma^2$.

The log-likelihood is
 \ben
  l(\mu,\sigma,s)= C - n \ln\sigma + \sum_i \left((s-1) \ln x_i -\frac{\left(b_s(x_i)-\mu\right)^2}{2 \sigma^2}\right)
 \een
 where $C$ is some constant (independent of the parameter). For a fixed $s$ it is maximized by the MLE for a Gaussian sample
 \ben\hat{\mu}_s=\frac{1}{n}\sum_i b_s(x_i)\een

  \ben \hat{\sigma}^2_s=\frac{1}{n}\sum_i
  \left(b_s(x_i)-\hat{\mu}\right)^2\een
We can use a numerical estimation to find the value of $s$ that
maximizes $l(\hat{\mu}_s,\hat{\sigma}_s,s)$;  see
\fref{fig-conf-pl} for a plot. The estimated value is
$\hat{s}=0.0041$, which gives $\hat{\mu}=1.5236$ and
$\hat{\sigma}=2.0563$.

We now give a confidence interval for $s$, using the asymptotic
result in \thref{theo-mle2}. A $95\%$ confidence interval is
readily obtained from \fref{fig-conf-pl}, which gives the interval
$[ -0.0782, 0.0841]$.

\mq{mle-qkla9}{Does the confidence interval justify the log
transformation~?} {Yes, since $0$ is in the interval.}

Alternatively, by \thref{theo-mle}, we can approximate the
distribution of $\hat{\theta}-\theta$ by a centered normal
distribution with covariance matrix $J(\hat{\theta})^{-1}$. After
some algebra, we compute the Fisher information matrix. We compute
the second derivative of the log-likelihood, and estimate the
Fisher information by the observed information (i.e. the value of
the second derivative at $\theta=\hat{\theta}$). We find:
$$
J=\left(\begin{array}{ccc}
  23.7 &        0  &  -77.1 \\
         0  &  47.3 &  -146.9 \\
   77.1 &  -146.9  &  1291.1
\end{array}\right)
$$
and
$$
J^{-1}=\left(\begin{array}{ccc}
 0.0605  &  0.0173  &  0.0056\\
    0.0173   & 0.0377  &  0.0053 \\
    0.0056  &  0.0053 &   0.0017
\end{array}\right)
$$
The last term of the matrix is an estimate of the variance of
$\hat{s}-s$. The 0.95 confidence interval obtained from a normal
approximation is $\hat{s} \pm 1.96 \sqrt{0.0017}=[-0.0770,
0.0852]$.
\end{ex}


\begin{figure}\Ifig{pl}{0.7}{0.3}%
\nfs{pl.m}%
 \mycaption{Profile log-likelihood for \exref{ex-conf-pl}, as a function
of the Box-Cox exponent $s$. The maximum likelihood estimator of
$s$ is the value that maximizes the profile log likelihood: a
confidence interval for $s$ is the set of $s$ for which the
profile log likelihood is below the horizontal dashed line.}
\mylabel{fig-conf-pl}
\end{figure}



\section{Proofs}
\begin{petit}
\paragraph{\thref{theo-conf-ci-median}}
Let $Z=\sum_{k=1}^n\ind{X_k \leq m_p}$ be the
number of samples that lie below or at $m_p$. The
CDF of $Z$ is $B_{n,p}$ since the events $\{X_k
\leq m_p\}$ are independent and $\P\lp X_k \leq
m_p\rp = p$ by definition of the quantile $m_p$.
Further:
 \bearn
 j \leq Z & \Leftrightarrow & X_{(j)} \leq m_p \\
 k \geq Z+1 & \Leftrightarrow & X_{(k)} > m_p
 \eearn
 thus we have the event equalities
 \ben
 \left\{X_{(j)} \leq m_p < X_{(k)}\right\} =
 \left\{ j \leq Z \leq k-1\right\}
 = \left\{ j-1 < Z \leq k-1\right\}
 \een and
 \ben
 \P \lp X_{(j)} \leq m_p < X_{(k)}\rp =
 B_{n,p}(k-1)-B_{n,p}(j-1)
 \een
 It follows that $[X_{(j)}, X_{(k)})$ is a
 confidence interval for $m_p$ at level $\gamma$
 as soon as $B_{n,p}(k-1)-B_{n,p}(j-1) \geq
 \gamma$.

The distribution of the $X_i$s has a density,
thus
 $(X_{(j)},X_{(k)})$ as well and
 $\P\left(X_{(j)}< m_p \leq X_{(k)}\right)=\P\left(X_{(j)}< m_p
 <X_{(k)}\right)$, thus $[X_{(j)}, X_{(k)}]$ is also a confidence interval at the same level.

For large $n$, we approximate the binomial CDF by
$N_{\mu,\sigma^2}$ with $\mu=n p$ and $\sigma^2=n
p (1-p)$, as follows:
 \ben
 \P\lp j-1 < Z \leq k-1  \rp = \P\lp j \leq  Z \leq k-1  \rp
 \approx N_{\mu, \sigma^2}(k-1)-N_{\mu, \sigma^2}(j)
 \een
 and we pick $j$ and $k$ such that
 \bearn
 N_{\mu, \sigma^2}(k-1) &\geq& 0.5 +
 \frac{\gamma}{2}\\
 N_{\mu, \sigma^2}(j) &\leq& 0.5 -
 \frac{\gamma}{2}
 \eearn
 which guarantees that $N_{\mu, \sigma^2}(k-1)-N_{\mu,
 \sigma^2}(j) \geq \gamma$. It follows that we
 need to have
 \bearn
 k-1 &\geq& \eta \sigma + \mu
 \\
 j &\leq& - \eta \sigma + \mu
 \eearn We take the smallest $k$ and the largest $j$ which satisfy these constraints,
 which gives the formulas in the theorem.

\paragraph{\thref{theo-conf-pi}}
Transform $X_i$ into $U_i=F(X_i)$ which is iid uniform. For
uniform RVs, use the fact that $\E(U_{(j)})=\frac{j}{n+1}$.
Then
 \bearn
\lefteqn{\P\left(U^n_{(j)} \leq U_{n+1}\leq U^n_{(k)}|
 U^n_{(1)}=u_{(1)},...,U^n_{(n)}=u_{(n)}\right)}\\
 &=&\P\left(u_{(j)} \leq U_{n+1}\leq u_{(k)}\right)\\
 &=&u_{(k)}-u_{(j)}
 \eearn
The former is since $U_{n+1}$ is independent of $(U_1,
...,U_n)$ and the latter since $U_{n+1}$ has a uniform
distribution on $[0,1]$. Thus
 \ben
\P\left(U^n_{(j)} \leq U_{n+1}\leq
U^n_{(k)}\right)=\E\left(U^n_{(k)}-U^n_{(j)}\right)=\frac{k-j}{n+1}
 \een
%
\paragraph{\thref{theo-conf-pin}}
%
First note that $X_{n+1}$ is independent of $\hat{\mu}_n,
\hat{\sigma}_n$. Thus $X_{n+1}-\hat{\mu}_n$ is normal with mean
$0$ and variance
 \ben \var(X_{n+1})+ \var(\hat{\mu}_n)=\sigma^2 + \frac{1}{n}\sigma^2
 \een
Further, $(n-1)\hat{\sigma}^2_n/\sigma^2$ has a $\chi^2_{n-1}$
distribution and is independent of $X_{n+1}-\hat{\mu}_n$. By
definition of Student's $t$, the theorem follows.
%
\paragraph{\thref{theo-harmos}}
Let $m'$ be the distribution mean of $b(X)$. By definition of a
confidence interval, we have $\P(u(Y_1,...,Y_n)
<m'<v(Y_1,...,Y_n)) \geq \gamma$ where the confidence interval
is $[u,v]$. If $b()$ is increasing (like the Box-Cox
transformation with $s \geq 0$)  then so is $b^{-1}()$ and this
is equivalent to $\P\left(b^{-1}(u(Y_1,...,Y_n))
<b^{-1}(m')<b^{-1}(v(Y_1,...,Y_n))\right) \geq \gamma$. Now
$b^{-1}(m')$ is the transformed mean, which shows the statement
in this case. If $b()$ is decreasing (like the Box-Cox
transformation with $s<0$) then the result is similar with
inversion of $u$ and $v$.
%



\paragraph{\thref{theo-inter-ci}}
 We do the proof for a confidence interval for some quantity
 $\theta$, the proof is the same for a prediction interval. By
 definition $\P( \theta \nin I )\leq \alpha$ and $\P( \theta \nin I' )\leq
 \alpha'$. Thus
  \ben
  \P( \theta \nin I \cap I' )= \P\left(
  (\theta \nin I ) \mor (\theta \nin I' )\right) \leq \P\left(
  \theta \nin I \right) + \P\left(\theta \nin I' )\right)\leq
  \alpha+\alpha'
  \een

\end{petit}
%

\section{Review}

\subsection{Summary}
\mylabel{sec-conf-summary}

\begin{enumerate}

\item A \imp{confidence} interval is used to quantify the
    \imp{accuracy} of a parameter estimated from the data.

\item For computing the central value of a data set, you
    can use either mean or median. Unless you have special
    reasons (see below) for not doing so, the median is a
    preferred choice as it is more robust. You should
    compute not only the median but also a confidence
    interval for it, using \tref{tab-q-05} on
    \pgref{tab-q-05}.

%\item For a \imp{very small data set} (10 to 20 points) it is not
%meaningful to compute confidence intervals for the mean unless you
%have a priori knowledge that the data comes from a normal
%distribution.

\item A \imp{prediction} interval reflects the
    \imp{variability} of the data. For small data sets ($n
    <  38$) it is not meaningful. For larger data sets, it
    can be obtained by \thref{theo-conf-pi}. The Lorenz
    curve gap also gives a scale free representation of the
    variability of the data.

\item Fairness indices are essentially the same as indices
    of variability. Jain' Fairness index is based on
    standard deviation, and is less robust than the Lorenz
    Curve gap, which should be preferred.

\item A confidence interval for the mean characterizes both
    the \imp{variability} of the data and the
    \imp{accuracy} of the measured average. In contrast, a
    confidence interval for the median does not reflect
    well the variability of the data, therefore if we use
    the median we need both a confidence interval for the
    median and some measure of variability (the quantiles,
    as on a Box Plot). Mean and standard deviation give an
    accurate idea of the \imp{variability} of the data, but
    only if the data is roughly normal. If it is not, it
    should be re-scaled using for example a Box-Cox
    transformation. Normality can be verified with a
    qq-plot.

\item The standard deviation gives an accurate idea of the
    \imp{accuracy} of the mean if the data is normal, but
    also if the data set is large. The latter can be
    verified with a bootstrap method.

\item The geometric [resp. harmonic] mean is meaningful if
    the data is roughly normal in log [resp. $1/x$] scale.
    A confidence interval for the geometric [resp.
    harmonic] mean is obtained as the exponential [resp.
    inverse] of the mean in log [resp. $1/x$] scale.

\item All estimators in this chapter are valid only if the
    data points are independent (non correlated). This
    assumption must be verified, either by designing the
    experiments in a randomized way, (as is the case with
    independent simulation runs), or by formal correlation
    analysis.
\item If you have a choice, use median and
quantiles rather than mean and standard
deviation, as they are robust to distributional
hypotheses and to outliers. Use prediction
intervals based on order statistic rather than
the classical mean and standard deviation.
\end{enumerate}
%
%Assume that we have obtained the outputs $x_1,...,x_n$ from $n$
%independent replications. We want a confidence interval for the
%median and for the mean.
%\paragraph{Confidence Interval for the Median} A confidence
%interval for the median is $[x_{(j)},x_{(k)}]$, where $x_{(j)}$
%is the $j$th value in ascending order. The values of $j$ and
%$k$ are taken from \tref{tab-q-05} on \pgref{tab-q-05}.
%
%\paragraph{Computing Confidence Interval for the Mean}
%\begin{enumerate}
%  \item Test whether $x_1, ...,x_n$ roughly fits a normal
%      distribution (visual test on qqplot).
%    \item If yes, apply the student $t$-statistic to obtain
%        a confidence interval for the mean. The confidence
%        interval is \be \bar{x}\pm \eta
%        \frac{s}{\sqrt{n}}\mylabel{eq-confint-1}\ee with $s
%        =
%        \sqrt{\frac{1}{n-1}\sum_{i=1}^n\left(x_i-\bar{x}\right)^2}$
%        and $t_{n-1}(\eta) = {1+\alpha \over 2}$.
% Here, $t_{n-1}$ is the student CDF with $n-1$ degrees of
% freedom and $\alpha$ is the confidence level (a typical
% value is $\alpha=0.95$).
%
%We are frequently in this case because each output $x_i$ is
%often itself an average of many entities, and tends to be
%normally distributed.
%
% \item Else (i.e. the sample $(x_1,...x_n)$ does not appear
%     to be normal),
%by the law of large numbers, $\bar{x}$ might still be
%normal, if $n$ is large.  The confidence interval is \ben
%\bar{x}\pm \eta \frac{s}{\sqrt{n}}\een with $N_{0,1}(\eta)
%= {1+\alpha \over 2}$. If $n\geq 24$, the value of $\eta$
%is within 5\% of that obtained by \eref{eq-confint-1}.
%
%
%Test whether $n$ is large enough by the bootstrap method
%(\sref{sec-conf-bootstrap}). Do a qq-plot of the $R$
%bootstrap estimates $T^r$; if they appear to be normal, $n$
%is large enough.
%
%\item Else (i.e. the sample $(x_1,...x_n)$ does not appear
%    to be normal and $n$ is not large enough), use the
%    bootstrap percentile estimate
%    (\sref{sec-conf-bootstrap}).
%\end{enumerate}

\subsection{Review Questions}

\mq{q-conf-aslasliq}
 {Compare (1) the confidence interval for the median of a sample of $n$ data values, at level $95\%$
 and (2) a prediction interval at level at least $95\%$, for
 $n=9,39,99$.}
 {From the tables in \cref{ch-tables} and \thref{theo-conf-pi} we obtain:
 (confidence interval for median, prediction interval):
 $n=9$: $[x_{(2)},x_{(9)}]$, impossible;
 $n=39$: $[x_{(13)},x_{(27)}]$, $[x_{(1)},x_{(39)}]$;
 $n=99$: $[x_{(39)},x_{(61)}]$, $[x_{(2)},x_{(97)}]$.
 The confidence interval is always smaller than the prediction interval.
 }


 \mq{stats-q701}
 {Call $L=\min\{X_1, X_2\}$ and $U=\max\{X_1, X_2\}$. We do an experiment and find $L=7.4$, $U=8.0$.
 Say which of the following statements is correct: ($\theta$ is
 the median of the distribution).
 (1) the probability of the event  $\{L\leq \theta \leq U\}$ is $0.5$
 (2) the probability of the event  $\{7.4\leq \theta \leq 8.0\}$ is $0.5$
 }
 {In the classical (non-Bayesian) framework, (1) is correct and (2) is wrong.
 There is nothing random in the event $\{7.4\leq \theta \leq
 8.0\}$, since $\theta$ is a fixed (though unknown) parameter. The probability of this event is
 either $0$ or $1$, here it happens to be $1$. Be careful with the ambiguity of a statement such
 as ``the probability that $\theta$ lies between $L$ and $U$ is $0.5$". In case of doubt,
 come back to the roots: the probability of an event can be interpreted as the ideal proportion
 of simulations that would produce the event.}



 \mq{stats-q729}{How do we expect a $90\%$ confidence interval to compare to a $95\%$ one~? Check this on the tables
 in  \sref{sec-ci}.}
 {It should be smaller. If we take more risk we can accept a smaller interval.
We can check that the values of $j$ [resp. $k$] in the tables
confidence intervals at level $\gamma=0.95$ are larger [resp.
smaller] than at confidence level $\gamma=0.99$.}


\mq{q-conf-asksdkldklskldsfuwe}
 {A data set has 70 points. Give the formulae
 for confidence intervals at level 0.95 for the median and the mean}
 {Median: from the table in \sref{sec-ci}
 $[x_{(27)},x_{(44)}]$. Mean: from \thref{theo-conf-ln}:
 $\hat{\mu} \pm 0.2343 S$ where $\hat{\mu}$ is the sample mean and
 $S$ the sample standard deviation. The latter is assuming the normal approximation holds, and should be verified by either
 a qqplot or the bootstrap.}

\mq{q-conf-asksdkldklskldsfuwe2}
 {A data set has 70 points. Give formulae
 for a prediction intervals at level 95\%}
 {From \thref{theo-conf-pi}: $[\min_i x_i, \max_i x_i]$.}

\mq{q-conf-sdkslkkl}
 {A data set $x_1, ...x_n$ is such that $y_i=\ln x_i$ looks normal. We obtain a confidence interval $[\ell, u]$ for the mean of $y_i$.
 Can we obtain a confidence interval for the mean of $x_i$ by a transformation of  $[\ell, u]$~?}
 {No, we know that $[e^{\ell}, e^{u}]$ is a confidence interval for the geometric mean, not the mean of $x_i$. In fact
 $x_i$ comes from a log-normal distribution, whose mean is $e^{\mu +
 \frac{\sigma^2}{2}}$ where $\mu$ is the mean of the distribution
 of $y_i$, and $\sigma^2$ its variance.}

\mq{q-conf-lkds}
 {Assume a set of measurements is corrupted by an error term that is normal, but positively correlated. If we would compute
 a confidence interval for the mean using the iid hypothesis, would the confidence interval be too small or too large~?}
 {Too small: we underestimate the error. This phenomenon is known in physics under
 the term \nt{personal equation}: if the errors are linked to the experimenter, they are positively
 correlated.}


\mq{q-conf-jksjk}
 {We estimate the mean of an iid data set by two different methods and obtain 2 confidence
 intervals at level $95\%$: $I_1=[2.01, 3.87]$, $I_2=[2.45, 2.47]$. Since the second interval is smaller, we
 discard the first and keep only the second. Is this a correct $95\%$ confidence interval~?}
 {No, by doing so we keep the interval $I=I_1 \cap I_2$, which is a $90\%$ confidence interval, not a $95\%$ confidence interval.}
