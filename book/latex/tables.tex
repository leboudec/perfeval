 \mylabel{sec-ci}
% \begin{minipage}[b]{0.5\textwidth}
 The following tables can be used to determine confidence
 intervals for quantiles (including median),
 according to
 \thref{theo-conf-ci-median}.
%\end{minipage}
%
%\hfill HERE COMES A DRAWING\\
For a sample of $n$ iid data points
$x_1,...,x_n$, the tables give a confidence
interval at the confidence level $\gamma=0.95$ or
$0.99$ for the $q$-quantile with $q=0.5$
(median), $q=0.75$ (quartile) and $q=0.95$. The
confidence interval is $[x_{(j)},x_{(k)}]$, where
$x_{(j)}$ is the $j$th data point sorted in
increasing order.

The confidence intervals for $q=0.05$ and $q=0.25$ are not given
in the tables. They can be deduced by the following rule. Let
$[x_{(j)},x_{(k)}]$ be the confidence interval for the
$q$-quantile given by the table.  A confidence interval for the
$1-q$-quantile is $[x_{(j')},x_{(k')}]$ with
 \bearn
 j' = n+1 -k \\
 k' = n+1 -j
 \eearn
 For example, with $n=50$, a confidence interval for the third quartile
 ($q=0.75$) at confidence level $0.99$ is $[x_{(29)},x_{(45)}]$,
 thus a confidence interval for the first quartile ($q=0.25$) at
 confidence level $0.99$ is $[x_{(6)},x_{(22)}]$.

For small values of $n$ no confidence interval is
possible. For large $n$, an approximate value is
given, based on a normal approximation of the
binomial distribution.

\begin{petiteNote} The tables give $p$, the actual confidence
level obtained, as it is not possible to obtain a
confidence interval at exactly the required
confidence levels. For example, for $n=10$ and
$\gamma=0.95$ the confidence interval given by
the table is $ \left[X_{(2)},X_{(9)}\right]$; the
table says that it is in fact a confidence
interval at level $0.979$.

The values of $j$ and $k$ are chosen such that
$j$ and $k$ are as symmetric as possible around
$\frac{n+1}{2}$. For example, for $n=31$ the
table gives the interval
$\left[X_{(10)},X_{(22)}\right]$. Note that this
is not the only interval that can be obtained
from the theorem. Indeed, we have:\begin{quote}
\begin{tabular}{c c c}
  % after \\: \hline or \cline{col1-col2} \cline{col3-col4} ...
  $j$ & $k$ &  $\P\left(X_{(j)}< m_{0.5} < X_{(k)}\right)$\\
  \hline
  \input{bin31ci.dat}%
\end{tabular}%
  \nfs{binomial.m}%
\end{quote}
Thus we have \imp{several} possible confidence
intervals. The table simply picked one for which
the indices are closest to
 being symmetrical around the estimated median, i.e. the indices $j$ and $k$ are equally
 spaced around $\frac{n+1}{2}$, which is used for estimating the median. In some cases,
  like $n=32$, we do not find such an interval exactly; we have for instance:
  \begin{quote}
\begin{tabular}{c c c}
  % after \\: \hline or \cline{col1-col2} \cline{col3-col4} ...
  $j$ & $k$ &  $\P\left(X_{(j)}< m_{0.5} < X_{(k)}\right)$\\
  \hline
  \input{bin32ci.dat}
\end{tabular}%
  \nfs{binomial.m}%
\end{quote}
Here, the table arbitrarily picked the former.
\end{petiteNote}



\begin{table}\center \scriptsize
\hspace{2mm}
 \begin{tabular}{|c|c|c|c|}
 \hline $n$ & $j$ & $k$ & $p$ \\ \hline \hline
\multicolumn{4}{|c|}{
$n \leq 5 $: no confidence interval possible.}\\ \hline
 6  & 1  & 6 & 0.969  \\ \hline
 7  & 1  & 7 & 0.984  \\ \hline
 8  & 1  & 7 & 0.961  \\ \hline
 9  & 2  & 8 & 0.961  \\ \hline
 10  & 2  & 9 & 0.979  \\ \hline
 11  & 2  & 10 & 0.988  \\ \hline
 12  & 3  & 10 & 0.961  \\ \hline
 13  & 3  & 11 & 0.978  \\ \hline
 14  & 3  & 11 & 0.965  \\ \hline
 15  & 4  & 12 & 0.965  \\ \hline
 16  & 4  & 12 & 0.951  \\ \hline
 17  & 5  & 13 & 0.951  \\ \hline
 18  & 5  & 14 & 0.969  \\ \hline
 19  & 5  & 15 & 0.981  \\ \hline
 20  & 6  & 15 & 0.959  \\ \hline
 21  & 6  & 16 & 0.973  \\ \hline
 22  & 6  & 16 & 0.965  \\ \hline
 23  & 7  & 17 & 0.965  \\ \hline
 24  & 7  & 17 & 0.957  \\ \hline
 25  & 8  & 18 & 0.957  \\ \hline
 26  & 8  & 19 & 0.971  \\ \hline
 27  & 8  & 20 & 0.981  \\ \hline
 28  & 9  & 20 & 0.964  \\ \hline
 29  & 9  & 21 & 0.976  \\ \hline
 30  & 10  & 21 & 0.957  \\ \hline
 31  & 10  & 22 & 0.971  \\ \hline
 32  & 10  & 22 & 0.965  \\ \hline
 33  & 11  & 23 & 0.965  \\ \hline
 34  & 11  & 23 & 0.959  \\ \hline
 35  & 12  & 24 & 0.959  \\ \hline
 36  & 12  & 24 & 0.953  \\ \hline
 37  & 13  & 25 & 0.953  \\ \hline
 38  & 13  & 26 & 0.966  \\ \hline
 39  & 13  & 27 & 0.976  \\ \hline
 40  & 14  & 27 & 0.962  \\ \hline
 41  & 14  & 28 & 0.972  \\ \hline
 42  & 15  & 28 & 0.956  \\ \hline
 43  & 15  & 29 & 0.968  \\ \hline
 44  & 16  & 29 & 0.951  \\ \hline
 45  & 16  & 30 & 0.964  \\ \hline
 46  & 16  & 30 & 0.960  \\ \hline
 47  & 17  & 31 & 0.960  \\ \hline
 48  & 17  & 31 & 0.956  \\ \hline
 49  & 18  & 32 & 0.956  \\ \hline
 50  & 18  & 32 & 0.951  \\ \hline
 51  & 19  & 33 & 0.951  \\ \hline
 52  & 19  & 34 & 0.964  \\ \hline
 53  & 19  & 35 & 0.973  \\ \hline
 54  & 20  & 35 & 0.960  \\ \hline
 55  & 20  & 36 & 0.970  \\ \hline
 56  & 21  & 36 & 0.956  \\ \hline
 57  & 21  & 37 & 0.967  \\ \hline
 58  & 22  & 37 & 0.952  \\ \hline
 59  & 22  & 38 & 0.964  \\ \hline
 60  & 23  & 39 & 0.960  \\ \hline
 61  & 23  & 39 & 0.960  \\ \hline
 62  & 24  & 40 & 0.957  \\ \hline
 63  & 24  & 40 & 0.957  \\ \hline
 64  & 24  & 40 & 0.954  \\ \hline
 65  & 25  & 41 & 0.954  \\ \hline
 66  & 25  & 41 & 0.950  \\ \hline
 67  & 26  & 42 & 0.950  \\ \hline
 68  & 26  & 43 & 0.962  \\ \hline
 69  & 26  & 44 & 0.971  \\ \hline
 70  & 27  & 44 & 0.959  \\ \hline
\hline $n \geq 71$ &
\multicolumn{1}{p{15mm}|}{$\approx \lfloor 0.50 n - 0.980 \sqrt{n}\rfloor$} &
\multicolumn{1}{p{15mm}|}{$\approx \lceil 0.50 n + 1 + 0.980 \sqrt{n}\rceil$} &
0.950 \\ \hline
\end{tabular}
 \hspace{2mm}
\hspace{2mm}
 \begin{tabular}{|c|c|c|c|}
 \hline $n$ & $j$ & $k$ & $p$ \\ \hline \hline
\multicolumn{4}{|c|}{
$n \leq 7 $: no confidence interval possible.}\\ \hline
 8  & 1  & 8 & 0.992  \\ \hline
 9  & 1  & 9 & 0.996  \\ \hline
 10  & 1  & 10 & 0.998  \\ \hline
 11  & 1  & 11 & 0.999  \\ \hline
 12  & 2  & 11 & 0.994  \\ \hline
 13  & 2  & 12 & 0.997  \\ \hline
 14  & 2  & 12 & 0.993  \\ \hline
 15  & 3  & 13 & 0.993  \\ \hline
 16  & 3  & 14 & 0.996  \\ \hline
 17  & 3  & 15 & 0.998  \\ \hline
 18  & 4  & 15 & 0.992  \\ \hline
 19  & 4  & 16 & 0.996  \\ \hline
 20  & 4  & 16 & 0.993  \\ \hline
 21  & 5  & 17 & 0.993  \\ \hline
 22  & 5  & 18 & 0.996  \\ \hline
 23  & 5  & 19 & 0.997  \\ \hline
 24  & 6  & 19 & 0.993  \\ \hline
 25  & 6  & 20 & 0.996  \\ \hline
 26  & 7  & 20 & 0.991  \\ \hline
 27  & 7  & 21 & 0.994  \\ \hline
 28  & 7  & 21 & 0.992  \\ \hline
 29  & 8  & 22 & 0.992  \\ \hline
 30  & 8  & 23 & 0.995  \\ \hline
 31  & 8  & 24 & 0.997  \\ \hline
 32  & 9  & 24 & 0.993  \\ \hline
 33  & 9  & 25 & 0.995  \\ \hline
 34  & 10  & 25 & 0.991  \\ \hline
 35  & 10  & 26 & 0.994  \\ \hline
 36  & 10  & 26 & 0.992  \\ \hline
 37  & 11  & 27 & 0.992  \\ \hline
 38  & 11  & 27 & 0.991  \\ \hline
 39  & 12  & 28 & 0.991  \\ \hline
 40  & 12  & 29 & 0.994  \\ \hline
 41  & 12  & 30 & 0.996  \\ \hline
 42  & 13  & 30 & 0.992  \\ \hline
 43  & 13  & 31 & 0.995  \\ \hline
 44  & 14  & 31 & 0.990  \\ \hline
 45  & 14  & 32 & 0.993  \\ \hline
 46  & 15  & 33 & 0.992  \\ \hline
 47  & 15  & 33 & 0.992  \\ \hline
 48  & 15  & 33 & 0.991  \\ \hline
 49  & 16  & 34 & 0.991  \\ \hline
 50  & 16  & 35 & 0.993  \\ \hline
 51  & 16  & 36 & 0.995  \\ \hline
 52  & 17  & 36 & 0.992  \\ \hline
 53  & 17  & 37 & 0.995  \\ \hline
 54  & 18  & 37 & 0.991  \\ \hline
 55  & 18  & 38 & 0.994  \\ \hline
 56  & 18  & 38 & 0.992  \\ \hline
 57  & 19  & 39 & 0.992  \\ \hline
 58  & 20  & 40 & 0.991  \\ \hline
 59  & 20  & 40 & 0.991  \\ \hline
 60  & 20  & 40 & 0.990  \\ \hline
 61  & 21  & 41 & 0.990  \\ \hline
 62  & 21  & 42 & 0.993  \\ \hline
 63  & 21  & 43 & 0.995  \\ \hline
 64  & 22  & 43 & 0.992  \\ \hline
 65  & 22  & 44 & 0.994  \\ \hline
 66  & 23  & 44 & 0.991  \\ \hline
 67  & 23  & 45 & 0.993  \\ \hline
 68  & 23  & 45 & 0.992  \\ \hline
 69  & 24  & 46 & 0.992  \\ \hline
 70  & 24  & 46 & 0.991  \\ \hline
 71  & 25  & 47 & 0.991  \\ \hline
 72  & 25  & 47 & 0.990  \\ \hline
\hline $n \geq 73$ &
\multicolumn{1}{p{15mm}|}{$\approx \lfloor 0.50 n - 1.288 \sqrt{n}\rfloor$} &
\multicolumn{1}{p{15mm}|}{$\approx \lceil 0.50 n + 1 + 1.288 \sqrt{n}\rceil$} &
0.990 \\ \hline
\end{tabular}
 \hspace{2mm}
\mycaption{Quantile $q=50$\%, Confidence Levels $\gamma=95\%$ (left) and $0.99\%$ (right) }
 \mylabel{tab-q-05}
\end{table}


\begin{table}\center \scriptsize
\hspace{2mm}
 \begin{tabular}{|c|c|c|c|}
 \hline $n$ & $j$ & $k$ & $p$ \\ \hline \hline
\multicolumn{4}{|c|}{
$n \leq 10 $: no confidence interval possible.}\\ \hline
 11  & 5  & 11 & 0.950  \\ \hline
 12  & 6  & 12 & 0.954  \\ \hline
 13  & 7  & 13 & 0.952  \\ \hline
 14  & 7  & 14 & 0.972  \\ \hline
 15  & 8  & 15 & 0.969  \\ \hline
 16  & 9  & 16 & 0.963  \\ \hline
 17  & 9  & 17 & 0.980  \\ \hline
 18  & 9  & 17 & 0.955  \\ \hline
 19  & 10  & 18 & 0.960  \\ \hline
 20  & 12  & 20 & 0.956  \\ \hline
 21  & 12  & 20 & 0.960  \\ \hline
 22  & 13  & 21 & 0.956  \\ \hline
 23  & 13  & 22 & 0.974  \\ \hline
 24  & 14  & 23 & 0.970  \\ \hline
 25  & 14  & 24 & 0.982  \\ \hline
 26  & 15  & 24 & 0.959  \\ \hline
 27  & 16  & 25 & 0.958  \\ \hline
 28  & 17  & 26 & 0.954  \\ \hline
 29  & 17  & 27 & 0.971  \\ \hline
 30  & 17  & 27 & 0.954  \\ \hline
 31  & 18  & 28 & 0.958  \\ \hline
 32  & 20  & 30 & 0.956  \\ \hline
 33  & 20  & 30 & 0.958  \\ \hline
 34  & 21  & 31 & 0.955  \\ \hline
 35  & 22  & 32 & 0.950  \\ \hline
 36  & 22  & 33 & 0.968  \\ \hline
 37  & 22  & 34 & 0.979  \\ \hline
 38  & 23  & 34 & 0.961  \\ \hline
 39  & 24  & 35 & 0.960  \\ \hline
 40  & 25  & 36 & 0.958  \\ \hline
 41  & 25  & 37 & 0.972  \\ \hline
 42  & 25  & 37 & 0.961  \\ \hline
 43  & 26  & 38 & 0.963  \\ \hline
 44  & 28  & 40 & 0.961  \\ \hline
 45  & 28  & 40 & 0.963  \\ \hline
 46  & 28  & 40 & 0.951  \\ \hline
 47  & 29  & 41 & 0.953  \\ \hline
 48  & 31  & 43 & 0.952  \\ \hline
 49  & 31  & 43 & 0.954  \\ \hline
 50  & 32  & 44 & 0.952  \\ \hline
 51  & 32  & 45 & 0.966  \\ \hline
 52  & 33  & 46 & 0.964  \\ \hline
 53  & 33  & 47 & 0.975  \\ \hline
 54  & 34  & 47 & 0.959  \\ \hline
 55  & 35  & 48 & 0.959  \\ \hline
 56  & 36  & 49 & 0.957  \\ \hline
 57  & 36  & 50 & 0.969  \\ \hline
 58  & 37  & 50 & 0.951  \\ \hline
 59  & 38  & 51 & 0.951  \\ \hline
 60  & 39  & 53 & 0.961  \\ \hline
 61  & 39  & 53 & 0.963  \\ \hline
 62  & 39  & 53 & 0.954  \\ \hline
 63  & 40  & 54 & 0.956  \\ \hline
 64  & 42  & 56 & 0.955  \\ \hline
 65  & 42  & 56 & 0.956  \\ \hline
 66  & 43  & 57 & 0.955  \\ \hline
 67  & 44  & 58 & 0.952  \\ \hline
 68  & 44  & 59 & 0.966  \\ \hline
 69  & 44  & 60 & 0.975  \\ \hline
 70  & 45  & 60 & 0.962  \\ \hline
 71  & 46  & 61 & 0.961  \\ \hline
 72  & 47  & 62 & 0.960  \\ \hline
 73  & 47  & 63 & 0.971  \\ \hline
 74  & 48  & 63 & 0.956  \\ \hline
 75  & 49  & 64 & 0.956  \\ \hline
\hline $n \geq 76$ &
\multicolumn{1}{p{15mm}|}{$\approx \lfloor 0.75 n - 0.849 \sqrt{n}\rfloor$} &
\multicolumn{1}{p{15mm}|}{$\approx \lceil 0.75 n + 1 + 0.849 \sqrt{n}\rceil$} &
0.950 \\ \hline
\end{tabular}
 \hspace{2mm}
\hspace{2mm}
 \begin{tabular}{|c|c|c|c|}
 \hline $n$ & $j$ & $k$ & $p$ \\ \hline \hline
\multicolumn{4}{|c|}{
$n \leq 16 $: no confidence interval possible.}\\ \hline
 17  & 7  & 17 & 0.992  \\ \hline
 18  & 8  & 18 & 0.993  \\ \hline
 19  & 9  & 19 & 0.993  \\ \hline
 20  & 10  & 20 & 0.993  \\ \hline
 21  & 11  & 21 & 0.991  \\ \hline
 22  & 11  & 22 & 0.995  \\ \hline
 23  & 12  & 23 & 0.994  \\ \hline
 24  & 13  & 24 & 0.992  \\ \hline
 25  & 13  & 25 & 0.996  \\ \hline
 26  & 13  & 25 & 0.993  \\ \hline
 27  & 15  & 27 & 0.992  \\ \hline
 28  & 15  & 27 & 0.993  \\ \hline
 29  & 16  & 28 & 0.992  \\ \hline
 30  & 16  & 29 & 0.995  \\ \hline
 31  & 17  & 30 & 0.994  \\ \hline
 32  & 18  & 31 & 0.993  \\ \hline
 33  & 18  & 32 & 0.996  \\ \hline
 34  & 19  & 32 & 0.991  \\ \hline
 35  & 20  & 33 & 0.990  \\ \hline
 36  & 21  & 35 & 0.991  \\ \hline
 37  & 21  & 35 & 0.993  \\ \hline
 38  & 21  & 35 & 0.990  \\ \hline
 39  & 23  & 37 & 0.990  \\ \hline
 40  & 23  & 37 & 0.991  \\ \hline
 41  & 23  & 39 & 0.997  \\ \hline
 42  & 24  & 39 & 0.994  \\ \hline
 43  & 25  & 40 & 0.993  \\ \hline
 44  & 26  & 41 & 0.992  \\ \hline
 45  & 26  & 42 & 0.995  \\ \hline
 46  & 27  & 42 & 0.990  \\ \hline
 47  & 28  & 44 & 0.993  \\ \hline
 48  & 29  & 45 & 0.991  \\ \hline
 49  & 29  & 45 & 0.993  \\ \hline
 50  & 29  & 45 & 0.990  \\ \hline
 51  & 31  & 47 & 0.990  \\ \hline
 52  & 31  & 47 & 0.991  \\ \hline
 53  & 31  & 49 & 0.996  \\ \hline
 54  & 32  & 49 & 0.993  \\ \hline
 55  & 33  & 50 & 0.993  \\ \hline
 56  & 34  & 51 & 0.992  \\ \hline
 57  & 34  & 52 & 0.995  \\ \hline
 58  & 35  & 52 & 0.991  \\ \hline
 59  & 36  & 53 & 0.990  \\ \hline
 60  & 37  & 55 & 0.992  \\ \hline
 61  & 37  & 55 & 0.993  \\ \hline
 62  & 37  & 55 & 0.991  \\ \hline
 63  & 39  & 57 & 0.991  \\ \hline
 64  & 39  & 57 & 0.991  \\ \hline
 65  & 40  & 58 & 0.991  \\ \hline
 66  & 41  & 59 & 0.990  \\ \hline
 67  & 41  & 60 & 0.993  \\ \hline
 68  & 42  & 61 & 0.993  \\ \hline
 69  & 42  & 62 & 0.995  \\ \hline
 70  & 43  & 62 & 0.992  \\ \hline
 71  & 44  & 63 & 0.991  \\ \hline
 72  & 45  & 64 & 0.991  \\ \hline
 73  & 45  & 65 & 0.994  \\ \hline
 74  & 45  & 65 & 0.992  \\ \hline
 75  & 47  & 67 & 0.992  \\ \hline
 76  & 48  & 68 & 0.991  \\ \hline
 77  & 48  & 68 & 0.992  \\ \hline
 78  & 48  & 68 & 0.991  \\ \hline
 79  & 50  & 70 & 0.991  \\ \hline
 80  & 50  & 70 & 0.991  \\ \hline
 81  & 51  & 71 & 0.990  \\ \hline
\hline $n \geq 82$ &
\multicolumn{1}{p{15mm}|}{$\approx \lfloor 0.75 n - 1.115 \sqrt{n}\rfloor$} &
\multicolumn{1}{p{15mm}|}{$\approx \lceil 0.75 n + 1 + 1.115 \sqrt{n}\rceil$} &
0.990 \\ \hline
\end{tabular}
 \hspace{2mm}
\mycaption{Quantile $q=75$\%, Confidence Levels $\gamma=95\%$ (left) and $0.99\%$ (right) }
\end{table}


\begin{table}\center \scriptsize
\hspace{2mm}
 \begin{tabular}{|c|c|c|c|}
 \hline $n$ & $j$ & $k$ & $p$ \\ \hline \hline
\multicolumn{4}{|c|}{
$n \leq 58 $: no confidence interval possible.}\\ \hline
 59  & 50  & 59 & 0.951  \\ \hline
 60  & 52  & 60 & 0.951  \\ \hline
 61  & 53  & 61 & 0.953  \\ \hline
 62  & 54  & 62 & 0.955  \\ \hline
 63  & 55  & 63 & 0.957  \\ \hline
 64  & 56  & 64 & 0.958  \\ \hline
 65  & 57  & 65 & 0.959  \\ \hline
 66  & 58  & 66 & 0.961  \\ \hline
 67  & 59  & 67 & 0.962  \\ \hline
 68  & 60  & 68 & 0.963  \\ \hline
 69  & 61  & 69 & 0.964  \\ \hline
 70  & 62  & 70 & 0.964  \\ \hline
 71  & 63  & 71 & 0.965  \\ \hline
 72  & 64  & 72 & 0.965  \\ \hline
 73  & 65  & 73 & 0.966  \\ \hline
 74  & 66  & 74 & 0.966  \\ \hline
 75  & 67  & 75 & 0.966  \\ \hline
 76  & 68  & 76 & 0.966  \\ \hline
 77  & 69  & 77 & 0.966  \\ \hline
 78  & 70  & 78 & 0.966  \\ \hline
 79  & 71  & 79 & 0.966  \\ \hline
 80  & 72  & 80 & 0.965  \\ \hline
 81  & 73  & 81 & 0.964  \\ \hline
 82  & 74  & 82 & 0.964  \\ \hline
 83  & 75  & 83 & 0.963  \\ \hline
 84  & 76  & 84 & 0.962  \\ \hline
 85  & 77  & 85 & 0.961  \\ \hline
 86  & 78  & 86 & 0.960  \\ \hline
 87  & 79  & 87 & 0.959  \\ \hline
 88  & 80  & 88 & 0.957  \\ \hline
 89  & 81  & 89 & 0.956  \\ \hline
 90  & 82  & 90 & 0.954  \\ \hline
 91  & 83  & 91 & 0.952  \\ \hline
 92  & 84  & 92 & 0.950  \\ \hline
 93  & 84  & 93 & 0.974  \\ \hline
 94  & 85  & 94 & 0.973  \\ \hline
 95  & 86  & 95 & 0.972  \\ \hline
 96  & 87  & 96 & 0.971  \\ \hline
 97  & 88  & 97 & 0.970  \\ \hline
 98  & 89  & 98 & 0.969  \\ \hline
 99  & 90  & 99 & 0.967  \\ \hline
 100  & 91  & 100 & 0.966  \\ \hline
 101  & 91  & 100 & 0.952  \\ \hline
 102  & 92  & 101 & 0.953  \\ \hline
 103  & 93  & 102 & 0.953  \\ \hline
 104  & 94  & 103 & 0.954  \\ \hline
 105  & 95  & 104 & 0.954  \\ \hline
 106  & 96  & 105 & 0.954  \\ \hline
 107  & 97  & 106 & 0.954  \\ \hline
 108  & 98  & 107 & 0.954  \\ \hline
 109  & 99  & 108 & 0.954  \\ \hline
 110  & 100  & 109 & 0.954  \\ \hline
 111  & 101  & 110 & 0.954  \\ \hline
 112  & 102  & 111 & 0.953  \\ \hline
 113  & 103  & 112 & 0.953  \\ \hline
 114  & 104  & 113 & 0.952  \\ \hline
 115  & 105  & 114 & 0.951  \\ \hline
 116  & 106  & 115 & 0.950  \\ \hline
 117  & 107  & 117 & 0.965  \\ \hline
 118  & 108  & 118 & 0.963  \\ \hline
 119  & 109  & 119 & 0.961  \\ \hline
 120  & 110  & 120 & 0.959  \\ \hline
 121  & 110  & 120 & 0.967  \\ \hline
 122  & 111  & 121 & 0.966  \\ \hline
 123  & 112  & 122 & 0.966  \\ \hline
\hline $n \geq 124$ &
\multicolumn{1}{p{15mm}|}{$\approx \lfloor 0.95 n - 0.427 \sqrt{n}\rfloor$} &
\multicolumn{1}{p{15mm}|}{$\approx \lceil 0.95 n + 1 + 0.427 \sqrt{n}\rceil$} &
0.950 \\ \hline
\end{tabular}
 \hspace{2mm}
\hspace{2mm}
 \begin{tabular}{|c|c|c|c|}
 \hline $n$ & $j$ & $k$ & $p$ \\ \hline \hline
\multicolumn{4}{|c|}{
$n \leq 89 $: no confidence interval possible.}\\ \hline
 90  & 76  & 90 & 0.990  \\ \hline
 91  & 79  & 91 & 0.990  \\ \hline
 92  & 80  & 92 & 0.990  \\ \hline
 93  & 81  & 93 & 0.991  \\ \hline
 94  & 82  & 94 & 0.991  \\ \hline
 95  & 83  & 95 & 0.991  \\ \hline
 96  & 84  & 96 & 0.992  \\ \hline
 97  & 85  & 97 & 0.992  \\ \hline
 98  & 86  & 98 & 0.992  \\ \hline
 99  & 87  & 99 & 0.992  \\ \hline
 100  & 88  & 100 & 0.993  \\ \hline
 101  & 89  & 101 & 0.993  \\ \hline
 102  & 90  & 102 & 0.993  \\ \hline
 103  & 91  & 103 & 0.993  \\ \hline
 104  & 92  & 104 & 0.993  \\ \hline
 105  & 93  & 105 & 0.993  \\ \hline
 106  & 94  & 106 & 0.993  \\ \hline
 107  & 95  & 107 & 0.993  \\ \hline
 108  & 96  & 108 & 0.993  \\ \hline
 109  & 97  & 109 & 0.993  \\ \hline
 110  & 98  & 110 & 0.993  \\ \hline
 111  & 99  & 111 & 0.993  \\ \hline
 112  & 100  & 112 & 0.993  \\ \hline
 113  & 101  & 113 & 0.993  \\ \hline
 114  & 102  & 114 & 0.992  \\ \hline
 115  & 103  & 115 & 0.992  \\ \hline
 116  & 104  & 116 & 0.992  \\ \hline
 117  & 105  & 117 & 0.992  \\ \hline
 118  & 106  & 118 & 0.991  \\ \hline
 119  & 107  & 119 & 0.991  \\ \hline
 120  & 108  & 120 & 0.991  \\ \hline
 121  & 109  & 121 & 0.990  \\ \hline
 122  & 109  & 122 & 0.995  \\ \hline
 123  & 110  & 123 & 0.995  \\ \hline
 124  & 111  & 124 & 0.995  \\ \hline
 125  & 112  & 125 & 0.994  \\ \hline
 126  & 113  & 126 & 0.994  \\ \hline
 127  & 114  & 127 & 0.994  \\ \hline
 128  & 115  & 128 & 0.994  \\ \hline
 129  & 116  & 129 & 0.993  \\ \hline
 130  & 117  & 130 & 0.993  \\ \hline
 131  & 118  & 131 & 0.993  \\ \hline
 132  & 119  & 132 & 0.992  \\ \hline
 133  & 120  & 133 & 0.992  \\ \hline
 134  & 121  & 134 & 0.992  \\ \hline
 135  & 122  & 135 & 0.991  \\ \hline
 136  & 123  & 136 & 0.991  \\ \hline
 137  & 124  & 137 & 0.990  \\ \hline
 138  & 124  & 138 & 0.995  \\ \hline
 139  & 125  & 139 & 0.995  \\ \hline
 140  & 126  & 140 & 0.995  \\ \hline
 141  & 127  & 141 & 0.994  \\ \hline
 142  & 127  & 141 & 0.992  \\ \hline
 143  & 128  & 142 & 0.992  \\ \hline
 144  & 129  & 143 & 0.992  \\ \hline
 145  & 130  & 144 & 0.992  \\ \hline
 146  & 131  & 145 & 0.992  \\ \hline
 147  & 133  & 147 & 0.992  \\ \hline
 148  & 134  & 148 & 0.992  \\ \hline
 149  & 135  & 149 & 0.992  \\ \hline
 150  & 136  & 150 & 0.991  \\ \hline
 151  & 137  & 151 & 0.991  \\ \hline
 152  & 138  & 152 & 0.990  \\ \hline
 153  & 138  & 152 & 0.992  \\ \hline
 154  & 139  & 153 & 0.992  \\ \hline
\hline $n \geq 155$ &
\multicolumn{1}{p{15mm}|}{$\approx \lfloor 0.95 n - 0.561 \sqrt{n}\rfloor$} &
\multicolumn{1}{p{15mm}|}{$\approx \lceil 0.95 n + 1 + 0.561 \sqrt{n}\rceil$} &
0.990 \\ \hline
\end{tabular}
 \hspace{2mm}
\mycaption{Quantile $q=95$\%, Confidence Levels $\gamma=95\%$
(left) and $0.99\%$ (right) }
\end{table}

%\center \footnotesize
%\begin{tabular}{|c||c|c|c|}\hline
% $n$ & $j$ & $k$ & $p$ \\ \hline \hline
%\input{binlist95.dat}
% \hline
%$n \geq 50$ & $\frac{n}{2}- 0.980\sqrt{n}$ &$\frac{n}{2}+1+
%0.980\sqrt{n}$ & 0.95 \\ \hline
%\end{tabular}
%\hspace{0.0cm}
%\begin{tabular}{|c|c|c|}\hline
%$j$ & $k$ & $p$ \\ \hline  \hline
%\input{binlist99.dat}
% \hline
% $\frac{n}{2}- 1.288\sqrt{n}$ &$\frac{n}{2}+1+ 1.288\sqrt{n}$ &
%0.99 \\ \hline
%
%\end{tabular}\mycaption{Confidence Interval
%for the median at confidence levels at least $0.95$ (left) and
%$0.99$ (right). For a sample of $n$ iid data points $x_1,...,x_n$,
%a confidence interval for the median is $[x_{(j)},x_{(k)}]$, where
%$x_{(j)}$ is the $j$th data point in increasing order: $p$ is the
%actual confidence level obtained (it is not possible to obtain a
%confidence interval at exactly the required confidence levels).
%Cells marked ``n/a" mean that no confidence interval is possible
%($n$ is too small).}
% \mylabel{tab-conf-medianci}
%\end{table}Table of confidence intervals for median:
%\nfs{Tables completes
%\begin{table}
%\begin{tabular}{|c|c|c|c||c|c|c|c|c|c|c|}
%\hline  $n$ & $j$ & $k$ & $p$ & $j$ & $k$ & $p$& $j$ & $k$ & $p$ & Best\\
%\hline \hline
%\input{binlist95-tout.dat}
%\hline
%
%\end{tabular} \mycaption{0.95}
%\end{table}
%\begin{table}
%\begin{tabular}{|c|c|c|c||c|c|c|c|c|c|c|}
%\hline  $n$ & $j$ & $k$ & $p$ & $j$ & $k$ & $p$& $j$ & $k$ & $p$ & Best\\
%\input{binlist99-tout.dat}
%\hline
%
%\end{tabular} \mycaption{0.99}
%\end{table}
%}
