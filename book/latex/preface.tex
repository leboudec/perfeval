\begin{minipage}[b]{0.6\textwidth}
\paragraph{Performance Evaluation} is often the critical part of
evaluating the results of a research project. Many of us are
familiar with simulations, but it is often difficult to address
questions like: Should I eliminate the beginning of the
simulation in order for the system to become stabilized~? I
simulate a random way point model but the average speed in my
simulation is not as expected. What happened~? The reviewers of
my study complained that I did not provide confidence
intervals. How do I go about this~? I would like to characterize the
fairness of my protocol. Should I use Jain's Fairness Index or
the Lorenz Curve Gap~? I would like to fit a distribution to
the flow sizes that I measured but all my measurements are
truncated to a maximum value; how do I account for the
truncation~?
\end{minipage}
%
\begin{minipage}[b]{0.4\textwidth}
\hfill\insfignc{scouacDoziert}{0.75}\\
\end{minipage}\\

This book groups a set of lecture notes for a course given at
EPFL. It contains all the material needed by an engineer who
wishes to evaluate the performance of a computer or
communication system. More precisely, with this book and some
accompanying practicals, you will be able to answer the above and
other questions, evaluate the performance of computer and
communication systems and master the theoretical foundations
of performance evaluation and of the corresponding software
packages.

In the past, many textbooks on performance evaluation have given
the impression that this is a complex field, with lots of baroque
queuing theory excursions, which can be exercised only by
performance evaluation experts. This is not necessarily the case. In
contrast, performance evaluation can and should be performed by
any computer engineering specialist who designs a system. When a
plumber installs pipes in our house, one expects her to properly
size their diameters; the same holds for computer engineers.

This book is not intended for the performance evaluation specialist. It
is addressed to \emph{every computer engineer or scientist} who is
active in the development or operation of software or hardware
systems. The required background is an elementary course in
probability and one in calculus.

\paragraph{The objective of this book} is therefore to make performance
evaluation usable by all computer engineers and scientists. The
foundations of performance evaluation reside in statistics and
queuing theory, therefore, \emph{some} mathematics is involved
and the text cannot be overly simplified. However, it turns out
that much of the complications are not in the general theories,
but in the exact solution of specific models. For example, some
textbooks on statistics (but none of the ones cited in the
reference list) develop various solution techniques for
specific models, the vast majority of which are encapsulated in
commercially or freely available software packages like Matlab,
S-PLUS, Excel, Scilab or R.

To avoid this pitfall, we focused first on the \emph{what}
before the \emph{how}. Indeed, the most difficult question in a
performance analysis is often ``what to do"; once you know what
to do, it is less difficult to find a way with your usual
software tools or by shopping the web. For example, what do we
do when we fit a model to data using least square
fitting~(\cref{ch-modfit})~? What is a confidence interval~?
What is a prediction interval (\cref{ch-conf})~? What is the
congestion collapse pattern (\cref{ch-metho})~? What is the
null hypothesis in a test and what does the result of a test
\emph{really} mean (\cref{ch-tests})~? What is an information
criterion~(\cref{ch-forecast})~? If no failure appears out of
$n$ experiments, what confidence interval can I give for the
failure probability (\cref{ch-conf})~?
%For example, good textbooks on simulation explain
%how to compute confidence intervals for the mean of a
%distribution, but hardly any will explain what type of confidence
%interval one should compute (for the mean or for a sample).

Second, for the \emph{how}, we looked for solution methods that as
universal as possible, i.e. that apply to many situations, whether simple or
complex. There are several reasons for this. Firstly, one
should use only methods and tools that one understands, and a
good engineer should first invest her time learning tools and
methods that she will use more often. Secondly, brute force and
a computer can do a lot more than one often seems to believe. This
philosophy is in sharp contrast to some publications on
performance evaluation. For example, computing confidence or
prediction intervals can be made simple and systematic if we
use the median and not the mean; if we have to employ the mean,
the use of likelihood ratio statistic is quite universal and
requires little intellectual sophistication regarding the model.
Thus, we focus on generic methods such as: the use of filters for
forecasting (\cref{ch-forecast}), bootstrap and Monte-Carlo simulations
for evaluating averages or prediction intervals
(\cref{ch-simul}), the likelihood ratio statistic for tests
(\cref{ch-conf}, \cref{ch-tests}), importance sampling
(\cref{ch-simul}), least square and $\ell^1$-norm minimization
methods (\cref{ch-modfit}).

When presenting solutions, we tried \emph{not} to hide their
limitations and the cases where they do not work. Indeed, some
frustrations experienced by young researchers can sometimes be
attributed to false expectations about the power of some
methods.

We give a coverage of queuing theory that attempts to strike a
balance between depth and relevance. During a performance
analysis, one is often confronted with the dilemma: should we
use an approximate model for which exact solutions exist, or
approximate solutions for a more exact model~?
%In accordance with the
%general philosophy of this book, I took the
%second option as much as possible, instead of
%dwelling on the meanders of explicit, exact
%closed forms that often apply only at the expense
%of unrealistic, restrictive assumptions.
We propose four topics (deterministic analysis, operational
laws, single queues, queuing networks) which provide a good
balance. We illustrate in a case study how the four topics can
be utilized to provide different insights on a queuing question.
For queuing networks, we give a unified treatment, which is
perhaps the first of its kind at this level of synthesis. We
show that complex topics such as queues with concurrency (MSCCC
queues) or networks with bandwidth sharing (Whittle networks)
all fit in the same framework of product form queuing networks.
Results of this kind have been traditionally presented as
separate; unifying them simplifies the student's job and
provides new insights.

We develop the topic of Palm calculus, also called ``the
importance of the viewpoint", which is so central to queuing
theory, as a topic of its own. Indeed, this topic has so many
applications to simulation and to system analysis in general,
that it is a very good time investment. Here too, we focus on
general purpose methods and results, in particular the
large-time heuristic for mapping various viewpoints
(\cref{ch-palm}).

\paragraph{\cref{ch-metho} gives a methodology} and serves
as introduction to the rest of the book.
Performance patterns are also described, i.e.
facts that repeatedly appear in various
situations, and knowledge of which considerably
helps the performance evaluation.

\cref{ch-conf} demonstrates how to summarize
experimental or simulation results, as well as how to
quantify their accuracy. It also serves as an
introduction to a scientific use of the
statistical method, i.e. pose a model and verify
its assumptions. In \cref{ch-modfit} we present
general methods for fitting an explanatory model
to data and the concept of heavy tail.
\cref{ch-tests} describes the techniques of
tests, and \cref{ch-forecast} those of forecasting.
These four chapters give a coverage of modern
statistics useful to our field.

%\imp{Part~\ref{part-tech}} focuses on evaluation techniques.
\cref{ch-simul} discusses discrete event simulation and several
important, though simple issues such as the need for transient
removal, for confidence intervals, and classical simulation
techniques. We also discuss importance sampling, which is very
useful for computing estimates of rare events; we give a
simple, though quite general and broadly applicable method.

\cref{ch-palm} describes Palm calculus, which relates the
varying viewpoints resulting from measurements done by
different operators. Here, we discuss freezing simulations, a
phenomenon which can be a problem for even simple simulations
if one is not aware of it. We also present how to perform a
perfect simulation of stochastic recurrences. \cref{ch-queuing}
discusses patterns specific to queuing, classical solution
methods for queuing networks, and, perhaps more important,
operational analysis for rapid evaluation.
%This is generally considered too
%complicated for applied textbooks, but, here too,
%I found that it is possible to convey the main
%ideas and results in a simple, accessible way.

%
%\cref{ch-fluid} provides, in a very accessible
%way, a general method for simplifying the
%analysis of a complex system, using fluid
%equations (micro-to-macro analysis), a powerful
%method often used in other contexts (such as
%chemistry and biology).

The appendix gives background information that cannot yet be easily
found elsewhere, such as a Fourier-free quick crash
course on digital filters (used in \cref{ch-forecast}) and
confidence intervals for quantiles.
%
%Sections marked with a $\star$ can be omitted or skimmed,
%depending on the reader's inclination. {\footnotesize Text in
%small font size can be skipped at first reading.}

Performance evaluation is primarily an art, and
involves using sophisticated tools such as
mathematical packages, measurement tools and
simulation tools. See the web site of the EPFL
lecture on Performance Evaluation for some
examples of \imp{practicals}, implemented in
matlab and designed around this book.

The text is intended for self-study. Proofs are not given when
there are easily accessible references (these are indicated in
the text); otherwise they can be found in appendixes at the end of the chapters.

%
%
%It contains many \imp{inline questions};  I invite the alert
%readers to try and answer the questions as they read.
% \mq{q-preface-1}
% {Where is the answer to an inline question~?}
% {In a footnote on the same page}
%Every chapter contains a \emph{review section}, which
%summarizes the main points. The \emph{exercise section} can be
%used as assignments in a lecture. The solutions are available
%on request; if time permits, a solution manual will eventually
%be available.

The \emph{Index} collects all terms and
expressions that are highlighted in the text like
{\it \blue \textsf{this}} and also serves as a
notation list.
