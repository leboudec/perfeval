%%%%%%%%%%%%%%%%%%%%%%%%%%%%%%%%%%%%%%%%%%%%%%%%%
%%%%%%%% This is Leb's usual stuff  %%%%%%%%%%%%%
%%%%%%%%                            %%%%%%%%%%%%%
%%%%%%%%%%%%%%%%%%%%%%%%%%%%%%%%%%%%%%%%%%%%%%%%%

\usepackage{lebListes}  %%% modifie les bullets selon AMS et fournit les macros suivants
\noitemsep  %%% no spacing between list items
%\doitemsep %%% retablir spacing par defaut

%%%% euro symbol by Alain Delplanque
\def\euro{\mbox{\raisebox{.25ex}{{\it =}}\hspace{-.5em}{\sf C}}}



\usepackage{ifthen}


\usepackage{algorithm,algpseudocode}

%% Bremaud's stuff
\usepackage{color,pstcol,pstricks}
%\usepackage{pstricks}
%\usepackage{color}
\newgray{backgroundgray}{0.90}
\definecolor{backgroundgray}{gray}{0.90}

%\usepackage{xspace}
%\newcounter{example}[section]
%\renewcommand{\theexample}{\arabic{section}.\arabic{example}}
%\newlength{\labelexample}
%\setlength{\labelexample}{2cm}
%\newenvironment{example}[1]
%{\refstepcounter{example}\par\noindent{\gray\rule{\labelexample}{1pt}}\par\vspace*{-.2\baselineskip}\noindent\textsc{Example~\theexample:\xspace}\ifthenelse{\equal{#1}{}}{}{{\sc\blue#1}}}
%{\nobreak\par\nobreak{\gray\hfill\rule{\labelexample}{1pt}}\par} %\vspace{-.5\baselineskip}

\newcommand{\eqframe}[1]{{\psframebox[linecolor=gray, framesep=4pt]{#1}}}

\newenvironment{preuve}
{\begin{quote}
\par\vspace{.5\baselineskip}\noindent\footnotesize\imp{Proof.}}
{\end{quote}\nobreak\hfill$\Box$\par\vspace{.5\baselineskip}}

\newenvironment{petit}
{\par\vspace{.5\baselineskip}\noindent\footnotesize}
{\nobreak\par\vspace{.5\baselineskip}}


\newenvironment{petiteNote}
{\begin{petit} \begin{quote} \imp{Note. }}
 {\end{quote}\end{petit}}



%% end Bremaud's stuff

%%%%%%%% paths, system dependent   %%%%%%%%
%\def\figuresfolder{../exoFigures/}
%\def\sourcefolder{../sourceFiles/}
\def\tempfolder{./temp/}
%\def\tempfolder{/Documents and Settings/leboudec/Desktop/evidbackup/leboudec/DocsDeCours/Common/BD d'exos/Temp/}
%%%%%%% end of system dependent part %%%%%

%%% set the flags to default values, unless they are already defined in
%% the main file

\providecommand{\imprimeTexteCache}{non}
\providecommand{\imprimeTexteSecret}{non}


%%% do not modify these lines
\newif\ifsol
\newif\ifnfs

\ifthenelse{\equal{\imprimeTexteCache}{oui}}{\soltrue}{\solfalse }
\ifthenelse{\equal{\imprimeTexteSecret}{oui}}{\nfstrue}{\nfsfalse}

%%% insert solutions or comments not for students with \sol, \ifsol...\fi
%%%  and \nfs



\ifsol
   \providecommand{\sol}[1]{\bs #1 \es}
\else
   \providecommand{\sol}[1]{}
\fi

\ifnfs
   \providecommand{\nfs}[1]{~\\{\footnotesize \sffamily \emph{\textbf{Note: } #1}~\\}}
\else
   \providecommand{\nfs}[1]{}
\fi

%% new method for solutions
%% use \bs \es instead

\newenvironment{mySolution}
{\red \it
\par\vspace{.5\baselineskip}\noindent\textbf{Solution.~}}
{\vspace{.5\baselineskip}}
\def\bs{\begin{mySolution}}
\def\es{\end{mySolution}}





\def\problemfile#1{%
\begin{filecontents}{\tempfolder#1}}

%\providecommand{\finproblemfile}{
%\end{filecontents}}

\providecommand{\m}[1]{
   \mylabel{#1}
   \input{\tempfolder#1}
   }



% \providecommand{\gradCourse}{oui}
% eventually modify to become compatible with my database of exercises
% \providecommand{\pointex}[1]{\fbox{$\rightarrow$ Exercise \ref{#1}}}


%%%% include graphics only with these commands %%%%%%%%%%%%%%%
\ifnfs
\providecommand{\insfignc}[2]{% same as insfig, no centering
 \includegraphics[width=#2\textwidth,keepaspectratio]{#1.eps}
 \footnote{\fbox{\texttt{file: #1}}}
 }

\else
\providecommand{\insfignc}[2]{% same as insfig, no centering
 \includegraphics[width=#2\textwidth,keepaspectratio]{#1.eps}
 }

\fi


\ifnfs

\providecommand{\Ifignc}[3]{% same as Ifig, no centering
 \includegraphics[width=#2\textwidth,height=#3\textwidth]{#1.eps}
 \footnote{\fbox{\texttt{file: #1}}}
 }

\else

\providecommand{\Ifignc}[3]{% same as Ifig, no centering
 \includegraphics[width=#2\textwidth,height=#3\textwidth]{#1.eps}
 }

\fi

\providecommand{\insfig}[2] {
\begin{center}
\insfignc{#1}{#2}\\
\end{center}
}

\providecommand{\Ifig}[3]{% modify width and height separately
 \begin{center}
 \Ifignc{#1}{#2}{#3}\\
 \end{center}
 }


\providecommand{\ifig}[2]{%% obsolete, for compatibility
 \insfignc{#1}{#2}
 }
%\providecommand{\ifig}[2] {\epsfig{file=#1.eps,
%width=#2\textwidth}}

% macros to be used instead of \label, \marginpar and \caption
% values of labels are printed when \imprimeTexteSecret is equal to oui
\newcommand{\mycaption}[1]{\caption{\footnotesize \sffamily #1}}
\newcommand{\mymarginpar}[1]{
    \marginpar{\sffamily \footnotesize #1}
    }
\ifnfs
\newcommand{\mylabel}[1]{
    %\fbox{\ttfamily \footnotesize #1 \label{#1}
    \label{#1}
    \footnote{\fbox{\texttt{label: #1}}}
  }
\else
\newcommand{\mylabel}[1]{
    \label{#1}
    }
\fi

\newcommand{\petitcar}[1]  {
   {\small \sffamily #1}
   }


%%%%%%%%%%%%%%%%%%% " to be done " flag %%%%%%%%%%%%%%%%%%%
%% printed only if solutions are printed
%%%%%%%%%%%%%%%%%%%%%%%%%%%%%%%%%%%%%%%%%%%%%%%%%%%%%%%%%%%
\ifnfs
\newcommand{\tbd}[1]{\textsc{To be done:\\#1}}
\else
\newcommand{\tbd}[1]{}
\fi


%%%%%%%%%%% inline questions %%%%%%%%%%%%%%%
% use mq for inline questions

\newtheorem{maquestion}{\sc Question}[section]
\newcommand\metEnFin[3]{
   %\AtEndDocument{\textbf{Question \ref{#1} on page \pageref{#1}: } #2~\\~\textbf{Answer. }#3~\\~\\}%
   \AtEndDocument{{\red{\sc\textbf{\ref{#1} (p.~\pageref{#1}). }} #2}~\\~{\sc \textbf{Answer. }}#3~\\~\\}%
   }
%\newcounter{monNumeroDeQuestion}
%\newcommand{\str}[2]{\addtocounter{strophe}{1}\thestrophe
%\newcommand\mq[3]{%%%label, question, reponse
%  \begin{maquestion}
%  \label{#1} #2 ~\\ \textbf{Solution.} #3
%  \end{maquestion}%
%  }

\ifsol
\newcommand\mq[3]{%%%label, question, reponse
  {
  \begin{maquestion}
  \label{#1} #2
  \end{maquestion}
  {\red \sc{ A. }
  }
   {\footnotesize \sffamily #3~\\}
   }
  }
\else
%\newcommand\mq[3]{%%%label, question, reponse
% {
%  \begin{maquestion}
%  \label{#1} #2
%  \end{maquestion}%
%  }
%  \metEnFin{#1}{#2}{#3}
%  }
  \newcommand\mq[3]{%%%label, question, reponse
  \begin{maquestion}
  \label{#1} #2 \footnote{#3}
  \end{maquestion}%
  }
\fi

%%%%%%% new term %%%%%%%%%%%%%%%%%%%%%%%%%%%%%
\newcommand{\nt}[1]{{\blue \it \textsf{#1}\index{#1}}}
\newcommand{\imp}[1]{{\blue \bf #1}}

%%%%%%% various shortcuts  %%%%%%%%%%%%%%%%%%%


%\newcommand{\mas}[1]{\section[#1]{\sc #1}}
%\newcommand{\mass}[1]{\subsection[#1]{\sc #1}}
%\newcommand{\masss}[1]{\subsubsection[#1]{\sc #1}}
%\newcommand{\monp}[1]{\paragraph{\sc #1}}

%%% for including pseudo code in text
\newcommand{\bp}{ % begin program
  \small \ttfamily
 \begin{tabbing}
 aaa\=aaa\=aaa\=aaa\=aaaaaaaa \= aaaaaaaaaa\= \kill
 }
\newcommand{\ep}{\end{tabbing}\normalfont\normalsize }% end program

%\newcommand{\pro}[1]{\ttfamily\small#1\normalfont\normalsize}
\newcommand{\pro}[1]{{\ttfamily\small#1}}

\newcommand{\aref}[1]{Algorithm~\ref{#1}}
\newcommand{\fref}[1]{Figure~\ref{#1}}
\newcommand{\sref}[1]{Section~\ref{#1}}
\newcommand{\cref}[1]{Chapter~\ref{#1}}
\newcommand{\coref}[1]{Corollary~\ref{#1}}
\newcommand{\eref}[1]{Eq.(\ref{#1})}
\newcommand{\tref}[1]{Table~\ref{#1}}
\newcommand{\dref}[1]{Definition~\ref{#1}}
\newcommand{\lref}[1]{Lemma~\ref{#1}}
\newcommand{\thref}[1]{Theorem~\ref{#1}}
\newcommand{\pref}[1]{Proposition~\ref{#1}}
\newcommand{\pgref}[1]{Page~\pageref{#1}}
\newcommand{\qref}[1]{Question~\ref{#1}}
\newcommand{\exref}[1]{Example~\ref{#1}}
\newcommand{\exoref}[1]{Exercise~\ref{#1}}
\newcommand{\hyporef}[1]{Hypothesis~\ref{#1}}

\newcommand{\paref}[1]{Algorithm~\ref{#1} on page~\pageref{#1}}
\newcommand{\pfref}[1]{Figure~\ref{#1} on page~\pageref{#1}}
\newcommand{\psref}[1]{Section~\ref{#1} on page~\pageref{#1}}
\newcommand{\pcref}[1]{Chapter~\ref{#1} on page~\pageref{#1}}
\newcommand{\pcoref}[1]{Corollary~\ref{#1} on page~\pageref{#1}}
\newcommand{\peref}[1]{Equation~(\ref{#1}) on page~\pageref{#1}}
\newcommand{\ptref}[1]{Table~\ref{#1} on page~\pageref{#1}}
\newcommand{\pdref}[1]{Definition~\ref{#1} on page~\pageref{#1}}
\newcommand{\plref}[1]{Lemma~\ref{#1} on page~\pageref{#1}}
\newcommand{\pthref}[1]{Theorem~\ref{#1} on page~\pageref{#1}}
\newcommand{\ppref}[1]{Proposition~\ref{#1} on page~\pageref{#1}}
\newcommand{\pqref}[1]{Question~\ref{#1} on page~\pageref{#1}}
\newcommand{\pexref}[1]{Example~\ref{#1} on page~\pageref{#1}}
\newcommand{\pexoref}[1]{Exercise~\ref{#1} on page~\pageref{#1}}

\newcommand{\maketline}{\noindent \rule{16.5cm}{0.02in} \newline }
\newcommand{\makeline}{\noindent \rule{16.5cm}{0.04in}}

%\newtheorem{st}{Student Tutorial}[section]
%\newtheorem{exo}{Exercise}[section]
%\newtheorem{theorem}{Theorem}[section]
%\newtheorem{lemma}{Lemma}[section]
%\newtheorem{proposition}{Proposition}[section]
%\newtheorem{corollary}{Corollary}[section]
%\newtheorem{definition}{Definition}[section]
%\newtheorem{conj}{Conjecture}[section]
%\newtheorem{claim}{Claim}[section]
%\newtheorem{remark}{Remark}[section]
%\newtheorem{fact}{Fact}[section]
%\newtheorem{conjecture}{Conjecture}[section]
%\newtheorem{ex}{Example}[section]


\newcommand{\dis}{\displaystyle}
\newcommand{\ie}{{\em i.e., }}
\newcommand{\eg}{{\em e.g., }}
\newcommand{\cqfd}{\hfill $\diamond$}
%\newcommand{\qed}{\hfill{$\diamond$}}
\newcommand{\la}{\lambda}
\newcommand{\calA}{\mathcal A}
\newcommand{\calB}{\mathcal B}
\newcommand{\calC}{\mathcal C}
\newcommand{\calD}{\mathcal D}
\newcommand{\calE}{\mathcal E}
\newcommand{\calF}{\mathcal F}
\newcommand{\calG}{\mathcal G}
\newcommand{\calH}{\mathcal H}
\newcommand{\calI}{\mathcal I}
\newcommand{\calJ}{\mathcal J}
\newcommand{\calK}{\mathcal K}
\newcommand{\calL}{\mathcal L}
\newcommand{\calM}{\mathcal M}
\newcommand{\calN}{\mathcal N}
\newcommand{\calQ}{\mathcal Q}
\newcommand{\calP}{\mathcal P}
\newcommand{\calR}{\mathcal R}
\newcommand{\calS}{\mathcal S}
\newcommand{\calT}{\mathcal T}
\newcommand{\calU}{\mathcal U}
\newcommand{\calV}{\mathcal V}
\newcommand{\calX}{\mathcal X}
\newcommand{\call}{\mathcal l}
\newcommand{\calZ}{\mathcal Z}

\newcommand{\lp}{\left(}
\newcommand{\lb}{\left[}
\newcommand{\lc}{\left\{}
\newcommand{\ld}{\left.}
\newcommand{\rp}{\right)}
\newcommand{\rb}{\right]}
\newcommand{\rc}{\right\}}
\newcommand{\rd}{\right.}

\newcommand{\eps}{\epsilon}
\newcommand{\esp}[1]{\E\left(#1\right)}
\newcommand{\espc}[2]{\E\left(\left.#1 \right| #2\right)}

\newcommand{\abs}[1]{\left| #1 \right|}
\newcommand{\card}{\mbox{card}}
\newcommand{\floor}[1]{\lfloor #1 \rfloor}


\newcommand{\midwor}[1]{\;\textnormal{ #1 }\;} % text in math mode

\newcommand{\nin}{\not\in} %% negation of \in
%\newcommand{\nin}{\in\!\!\!\!/} %% negation of \in
%\newcommand{\nin}{\in\!\!\/} %% negation of \in
% comb(n,p) = number of combinations of p among n
\newcommand{\comb}[2]{
 \left(
 \begin{array}{c}
    #1 \\
    #2
  \end{array}
  \right)
  }
%%% indicator functions
\newcommand{\ind}[1]{\mathbf{1}_{\{#1\}}}


 %%% norm
 \newcommand{\norm}[1]{\left\|#1\right\|}

\def\be{\begin{equation}}
\def\ee{\end{equation}}
\def\ben{\[}
\def\een{\]}
\def\bearn{\begin{eqnarray*}}
\def\eearn{\end{eqnarray*}}
\def\bear{\begin{eqnarray}}
\def\eear{\end{eqnarray}}
\def\barr{\begin{array}}
\def\earr{\end{array}}
\def\bel{\be \barr{l}}% equation array adjusted left, numbered
\def\eel{\earr\ee}
\def\beln{\ben \barr{l}}% equation array adjusted left, no number
\def\eeln{\earr\een}

\def\bfig{\begin{figure}}
\def\efig{\end{figure}}
% matrix with column formats in argument
% for example \bmat{cccc}
\def\bmat{\left(\begin{array}}
\def\emat{\end{array}\right)}

\newcommand{\erfc}{\mathrm{erfc}}
\newcommand{\cov}{\mathrm{cov}}
\newcommand{\var}{\mathrm{var}}
\newcommand{\sinc}{\mathrm{sinc}}
\newcommand{\diag}{\mathrm{diag}}
\newcommand{\limit}[2]{\lim_{#1 \rightarrow #2}}
\newcommand{\eqdef}{\stackrel{\mathrm{def}}{=}} % by definition


\newcommand{\pd}{day$^{-1}$}% per day
\newcommand{\dert}[1]{\frac{d#1}{dt}} % derivative wr t
\newcommand{\pdert}[1]{\frac{\partial#1}{\partial t}} % partial derivative wr t
\newcommand{\pder}[2]{\frac{\partial#1}{\partial#2}} % partial derivative wr 2nd arg
\newcommand{\rear}[1]{\stackrel{#1}{\rightarrow}} % reaction arrow
\newcommand{\drear}[2]{\begin{array}{c}#1\\ \rightleftarrows\\#2\end{array}} % double reaction arrow
\newcommand{\df}[1]{\textbf{#1}} % deterministic function
\newcommand{\rf}[1]{\textsf{\textbf{#1}}} % random quantity
\newcommand{\pbd}[1]{\parbox[t]{14cm}{#1}} % parbox approx 10 cm wide
%\newcommand{\eqdef}{\stackrel{\mathrm{def}}{=}} % equality as a definition

\newcommand{\vone}{\vec{1}}
\newcommand{\vzero}{\vec{0}}
\newcommand{\vu}{\vec{u}}
\newcommand{\vmu}{\vec{\mu}}
\newcommand{\vomega}{\vec{\omega}}
\newcommand{\vtheta}{\vec{\theta}}
\newcommand{\vtau}{\vec{\tau}}
\newcommand{\va}{\vec{a}}
\newcommand{\vb}{\vec{b}}
\newcommand{\vc}{\vec{c}}
\newcommand{\ve}{\vec{e}}
\newcommand{\vf}{\vec{f}}
\newcommand{\vk}{\vec{k}}
\newcommand{\vm}{\vec{m}}
\newcommand{\vn}{\vec{n}}
\newcommand{\vp}{\vec{p}}
\newcommand{\vq}{\vec{q}}
\newcommand{\vr}{\vec{r}}
\newcommand{\vs}{\vec{s}}
\newcommand{\vx}{\vec{x}}
\newcommand{\vy}{\vec{y}}
\newcommand{\vz}{\vec{z}}
\newcommand{\vv}{\vec{v}}
\newcommand{\vw}{\vec{w}}
\newcommand{\vK}{\vec{K}}
\newcommand{\vX}{\vec{X}}
\newcommand{\vY}{\vec{Y}}
\newcommand{\vZ}{\vec{Z}}
\newcommand{\vV}{\vec{V}}
\newcommand{\vepsilon}{\vec{\epsilon}}
\newcommand{\vphi}{\vec{\varphi}}

\newcommand{\mif}{\mathrm{\; if \; }}
\newcommand{\mthen} {\mathrm{\;  then \;  }}
\newcommand{\melse} {\mathrm{\;  else \;  }}
\newcommand{\mand} {\mathrm{\;  and \; }}
\newcommand{\mor} {\mathrm{\;  or \; }}
\newcommand{\mst} {\mathrm{\;  such \;  that \; }}
\newcommand{\mfa} {\mathrm{\;  for \;  all \; }}
\newcommand{\mfor} {\mathrm{\;  for \;  }}
\newcommand{\mf} {\mathrm{\;  for \; }}
\newcommand{\mfs} {\mathrm{\;  for \; some \; }}
\newcommand{\moth} {\mathrm{\;  otherwise \; }}
\newcommand{\mwith} {\mathrm{\;  with \; }}

\def\I{I\!d}
\def\Reals{\mathbb{R}}
\def\Rats{\mathbb{Q}}
\def\Ints{\mathbb{Z}}
\def\Comps{\ensuremath{\mathbb{C}}}
\def\GF{\ensuremath{\mathbb{F}}}
\newcommand{\ints}[1]{{\Ints/#1\Ints}}
\def\Nats{\mathbb{N}}
\def\E{\mathbb{E}}
\def\P{\mathbb{P}}
\def\T{\mathbb{T}}
\def\Abb{\mathbb{A}}
\def\mpc{\otimes}
\def\mpd{\oslash}
\def\Mpc{\overline{\otimes}}
\def\Mpd{\overline{\oslash}}

\newcommand{\cro}[1]{\langle#1\rangle}   % bracket for integrals etc
\newcommand{\bra}[1]{\lb#1\rb}   % modulo class
% ajoute LEB 2001.01.11 to compile netcal book
\def\mod{\bmod}
\def\hdots{\cdots}


%% for compatibility, don't use any more the \pr command;
%% use \begin{preuve} instead
\newcommand{\pr}{\paragraph{Proof: }}

%% copied from VOJ
%\def\cqfd{\mbox{\rule[0pt]{1.5ex}{1.5ex}}}
%\newenvironment{preuve}{\emph{Proof: }}{\hspace*{\fill}~\QED\par\endtrivlist\unskip}
% replaced by Bremaud's proof env.
%\newenvironment{preuve}{\emph{Proof: }}{\hspace*{\fill}~\cqfd\par\endtrivlist\unskip}

\newcommand{\bracket}[1]{ \left\{\begin{array}{l} #1 \end{array} \right.}
\newcommand{\gap}{\,\,\,}



%%% don't use these commands any more- left for compatibility

\providecommand{\idb}[1]{
   \input{#1}
   }





%%% from Andrea Ridolfi's style for Bremaud's book
%% modified LEB to fit multicolumn
\newsavebox{\coloredbox}
\newenvironment{shadethm}
{\par\noindent\begin{lrbox}{\coloredbox}\begin{minipage}{\linewidth}\begin{theorem}}
{\end{theorem}\end{minipage}\end{lrbox}\noindent\psframebox[fillstyle=solid,
fillcolor=backgroundgray, linestyle=none,
framesep=10pt]{\usebox{\coloredbox}}}


%%% shaded box
\newenvironment{sh}
{\begin{lrbox}{\coloredbox}\begin{minipage}{\linewidth}}
{\end{minipage}\end{lrbox}\noindent\psframebox[fillstyle=solid,
fillcolor=backgroundgray, linestyle=none,
framesep=10pt]{\usebox{\coloredbox}}}

%%% boxed theorem
\newsavebox{\traitbox}
\newenvironment{framethm}
{\begin{lrbox}{\traitbox}\begin{minipage}[h]{0.97\linewidth}\begin{theorem}}
{\end{theorem}\end{minipage}\end{lrbox}
\framebox[\linewidth][c]{\usebox{\traitbox}}}

%%% framedbox

\newenvironment{fr}
{\begin{lrbox}{\traitbox}\begin{minipage}[h]{0.97\linewidth}}
{\end{minipage}\end{lrbox}
\framebox[\linewidth][c]{\usebox{\traitbox}}}



%%%%%%%%%%%%%%%%%%%%%%%%%%%%%%%%%%%%%%%%%%%%%%%%%
%%%%%%%%  End of Leb's usual stuff  %%%%%%%%%%%%%
%%%%%%%%                            %%%%%%%%%%%%%
%%%%%%%%%%%%%%%%%%%%%%%%%%%%%%%%%%%%%%%%%%%%%%%%%
